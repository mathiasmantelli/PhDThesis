\chapter{Introduction}
Robots can be grouped into different classes depending on their function and the workplace they are designed for~\cite{Kumar2005Industrial, ISO2012Robots, Haidegger2013Applied}. Among all the classes of robots, there are two major ones, called \textit{industrial} and \textit{service}~\cite{UN2003World, De2011Domestic, Chiang2020Impacts, Garcia2020Service, Chen2021Human}. Industrial robots (IRs) are, according to the Robotic Industries Association (RIA), automatically controlled, reprogrammable, multipurpose manipulators programmable in three or more axes~\cite{ANSI2012American}. They can be fixed in place or mobile platforms for use in industrial automation applications. Traditionally, such robots are  designed for factories, where they are deployed for different application categories such as palletizing~\cite{Moura2018Application}, painting~\cite{Asadi2018Pictobot}, welding~\cite{Yao2019Light}, assembly~\cite{Kyrarini2019Robot}, and general handling tasks~\cite{Van2004Online, Hagele2016Industrial}. To meet the requirements for this set of applications, IRs have a wide variaty of designs regarding payload capacity, workspace volume, and number of robot axes~\cite{Hagele2016Industrial}. On the other hand, service robots (SRs) are robots that work semi or completely autonomously to perform useful services for the well-being of humans and equipment, excluding manufacturing operations~\cite{ISO2012Robots}. The SRs come in all different designs, as they may or may not be equipped with an arm structure, and currently most of them are mobile~\cite{Garcia2020Service}. The International Federation of Robotics (IFR) divides SRs into two subclasses based on their usability: \textit{professional} and \textit{personal}/\textit{domestic} service robots~\cite{Litzenberger2018Service}. Some examples of the professional service robots (PSRs) are defence robots~\cite{Martinic2014Proliferation}, farmer-assistants~\cite{Vakilian2017Farmer}, medical~\cite{Abubakar2020Arna}, and logistic~\cite{Thamrongaphichartkul2020Framework}. Domestic service robots (DSRs) include but are not limited to vaccum cleaners~\cite{Forlizzi2006Service}, lawn-mowers~\cite{Borinato2017Auto}, food and beverage waiters~\cite{Wan2020Waiter}, and elderly assistants~\cite{Hersh2015Overcoming}. The market for SRs has been regularly rising, and it is no surprise that there is an expectation that it grows even further in the next few years~\citep{De2011Domestic, Chiang2020Impacts}. The decreasing cost of hardware components (processors, motor drivers, and sensors), the increasing energy density and lower cost of batteries, and the current pandemic situation drive this expansion~\cite{Chiang2020Impacts}. 

%The industrial robots are the largest commercial application of robotics technology today, as they were the motivation for the development of the foundations for robot motion planning and control~\cite{Hagele2016Industrial}. 

\section{Scope}
According to the Population Division (PD) of the United Nations, in 2015 there were 901 million people aged 60 or over, representing 12\% of the global population~\cite{Population2015World}. Besides, the PD projects that by 2030 the number of elderly people in the world will reach 1.4 billion, and 2.1 billion by 2050. Several policies to tackle the problems of population againg have been proposed by several countries, including for example facilities for the elderly~\cite{Lin2018Qualitative, Seddigh2020Comparative}. However, placing elderly people in facilities for retirement or even in nursing homes may cause some problems, such as physically, emotionally, and psychologically dependencies~\cite{Theurer2015Need}. Aditionally, some elderly does not voluntarily stay at nursing homes, preferring to spend their remaining years at their home where they have a more positive self-image than those who live in the nursing homes~\cite{Kok2015Costs, Lin2018Qualitative}. The increasing number of older people living at home supports the need for DSRs to automate processes and tasks that may be tedious, inconvenient, or even challenging for older people~\cite{Paulius2019Survey, Torresen2020Special}. In general, these sort of robots can contribute on practical tasks for humans as robot assistants or robot companions, such as watching older adults with regard to emergency situations, reminding people to take their medicines, and searching, picking, and placing objects~\cite{Sprute2017Ambient, Torresen2018Robot, Paulius2019Survey}. 

Additionally, while some of the main motivations for the deployment of SRs have been elderly assistance and productivity improvement, the current COVID-19 pandemic has brought a more critical purpose for them~\cite{Chiang2020Impacts}. PSRs can be deployed to perform a series of applications to provide contactless services, ensuring humans can practice social distancing~\cite{Seidita2021Robots}. Besides disinfecting indoor environments~\cite{Mantelli2022Autonomous}, PSRs also have the potential to suport hospitality industry~\cite{Rosete2020Service}, and deliver medications and food~\cite{Lee2009Snackbot, Yang2020Combating}. The use of SRs in logistic applications is relevant during such unusual scenarios. In fact, some national organization from the United States identified it as one of the broad areas where robotics can make difference during outbreaks~\cite{Seidita2021Robots}.

In many example applications we listed above, it is likely that SRs have to perform some sort of searching tasks. Simple examples would be DSRs searching and picking objects for elderly with mobility restrictions and PSRs deliverying packages to an specific spot in an unknown environment. Similar to humans in the context of object searching tasks, SRs should also not rely on the assumption that the object (or regions of interest) they are searching for is already within their field of view (FoV)~\cite{Sjoo2012Topological}. Hence, they have to find the target object in large-scale environments based on primarily their visual sensors, which is known as object search (OS) problem~\cite{Aydemir2013Active}. But how does a SR find the target object that is not initially within its FoV? One way to address this problem is to make the SR to perform a brute-force OS, in which it visits the whole environment following a predefined search route. Even though this strategy seems a straightforward solution, it does not efficiently solve the problem~\cite{Rasouli2020Attention}. The SR will eventually find the target object, but the searching process may be time-consuming due to the long distances traveled by the robot. Another more efficient solution is to consider a search strategy that incorporates information from both the environment and the target object, to improve the searching performance. For the environment, such information could the shape of the room~\cite{Aydemir2011Object}, whereas for the target object it could be the color or shape~\cite{Rasouli2020Attention}, for example. The search strategy is one of the most important part of an OS approach, as it directly impacts the efficiency of an OS system~\cite{Aydemir2013Active}.

The research community has proposed valuable works related to the OS problem~\cite{Ekvall2007Object,Sjoo2009Object,Sjoo2012Topological,Aydemir2013Active,Rasouli2020Attention}. The problem is proven to be NP-Complete~\cite{Tsotsos1992Onthe,Ye2001AComplexity}, which means that the optimal search solution can be computed by approximation~\cite{Sjoo2012Topological}, minimizing the search cost as much as possible. In the case of SR performing OS tasks, such approximation could be computed with aid of strong cues provided by the semantics of both the environment and other objects in the SR's surroundings~\cite{Sjoo2012Topological}. Semantics can be regarded as the high-level information inferred (or ``perceived'') from the environment, including but not limited to names and categories of different objects, rooms and locations~\cite{Vasudevan2007Cognitive, Sjoo2012Topological, Liu2016Extracting}. In a similar way, Semantic maps encode not just the geometric and topological description of the environment, but also its semantic interpretation, providing a friendly way for robots to communicate with humans~\cite{Liu2016Extracting}. Then, when the SR processes its sensor readings to infer further knowledge about its surroundings, it increases the level of abstraction of the environment over time \cite{Barber2018Mobile}. The use of obth semantic informatio and map into robotic applications enhances the robot's autonomy and robustness in many ways, besides facilitating some challenging tasks~\cite{Cesar2016Simultaneous}.

%Escrever agora sobre os "desafios", como inferir essa informacao semantic e como usa-la no contexto de busca por objetos

\section{Objective}

This thesis aims to exploit the organization of both the objects of the environment and the environment itself to infer semantic search cues to address the OS problem. 

We claim that usually humans do not randomly organize what they build (cities, facilities, identification systems and processes, etc). It is no suprise that humans can improve their efficiency while performing daily tasks, like OS, if the environment is logically organized. Thus, they can save energy and time during such tasks. For example, most of cities have their own set of rules for numbering the properties, although there is no a unique and global rule for that. Then, the habitants can study it to understand the numbering pattern. This way, they can estimate where a certain unknown building is in the city. On the contrary, when there are no written rules to specificy the organization of the environment, humans can understand them just by observing the environment for a while. Therefore, we are interested in making the SRs to take advantage of such available organization to improve its performance in the OS problem.

We consider that semantic information could be inferred from the environment, and it could be used to help SRs in search tasks. Such semantic information is useful to OS systems because they could be used as high-level search cues. Then, with a semantic-based OS system, the SR would not need to search the whole environment to find the target object. The human reasoning process relies on several sorts of high-level search cues during searches. In our daily life, we read signs, symbols, and labels to evaluate which direction we should go to find a specific room in an unknown environment. Another example would be someone that first checks whether the car of the family is at home, to then search for the key of the car. In particular, we are focusing on the positional semantic information inferred from the organization of the environment, like the labels and signs in the first example, or the acknowledgment that objects may be moved by other people.


%O mundo não é aleatoriamente organizado. Os humanos "precisam" de lógica para resolver problemas
%"Semântica Organizacional" do mundo (ex: não existe uma regra universal sobre numeração de portas, mas todo lugar é organizado de alguma forma "intuitiva"
%Artigo das portas: como uma porta se posiciona no ambiente em relação ao seus "pares". HeatMap: como os objetos se posicionam em relação ao tempo
%Buscar aleatoriamente é exploração, e não é eficiente. O ser humano tenta prever aonde as coisas estão para fazer o menor esforço
%Se as coisas estão fixas, o humano tenta entender a organização. Se as coisas mudam, ele tenta entender o padrão da mudança
%Semântica posicional


%High-level robotics tasks demand the robots to read the environment similarly to humans, which reason over many characteristics of the room or objects and not only over the size of the free or occupied areas. The robot's geometric perception, i.e., raw sensor readings that generate standard geometric maps, is not descriptive and informative enough for providing such improvement demanded by the high-level tasks. This limitation does not mean that geometric maps are useless or irrelevant these days. On the contrary, they are still helpful and significant for the robot's safety navigation or path planning. However, there is a demand for complementing and extending the robot's geometric perception with meaningful knowledge from the environment. We then argument that the high-level information inferred from the sensor readings, also called semantic information, may be heavily exploited to complement the robot's perception when building autonomous robots.



\section{Contributions of this Thesis}

This thesis presents results \cite{Mantelli2021Semantic, Mantelli2022Semantic} showing that semantic information inferred from the organization of the environment can help SRs in the OS problem. Specially, we show that the use of organizational semantic information as search cues in the search strategy of OS systems can make the SR save resources by not visiting the whole environment. Besides, we show that the proper use of semantic information can improve the SRs' perception to perform high-level tasks, bringing them closer to humans.

The first contribution of this thesis is a long-term semantic system that searches for a target object in unknown and dynamic environments~\cite{Mantelli2022Semantic}. It assumes that some objects within the environment are not necessarily always static, and people move them around over time. In this way, its goal is to incorporate a person's routine and habits in the search strategy and then make search estimations. This work aimed to model the semantic information of how objects are organized over time within an environment. Then, it uses this information to avoid making the SRs search for the target object in not promissing regions. This idea came from observing how the objects are placed over time and that every person has their own singularities in terms of object placement.
%The contribution of the author of this thesis is in modeling the semantic information as part of the search strategy and in building a heat map with the inferred data. 

Another contribution is an OS system that seeks to find an specific room based on the organization of door labels within the environment~\cite{Mantelli2021Semantic}. Although humans heavily rely on texts for accomplishing several tasks, text as a data source is not very popular in robotics. In this work, we have argued that texts have a great potential for providing search clues and are often found in man-made environments. This idea came from the human behavior when searching for a someone's office in an unknown building, and how the door labels are used for estimating whether the current corridor is promising for finding the target office. The search strategy relies on the patterns of door labels in indoor scenarios and it reasons over them to estimate which corridor is more promising for achieving the goal.

\section{Outline}
\textcolor{myred}{The outline of this thesis is as follows. First, in Chapter~\ref{chap:2_theoretical_background}, we introduce a background on the main problems in mobile robotics, as well as the most popular approaches that deal with each problem. Besides, it also presents the general concepts of the OS problem, which is used throughout this thesis along with the basic concepts of mobile robotics. In Chapter~\ref{chap:3_text_os_system}, we provide our first semantic OS system that is based on text as the main source of semantic information. Next, in Chapter~\ref{chap:4_temporal_os_system}, we discuss our second semantic OS system, which is the one that aims to understand how the objects within the environment are moved through a period of time, to make predictions about their future positions. Lastly, in Chapter~\ref{chap:5_discussion_thesis_progress}, we conclude this thesis proposal by discussing the current contributions and draw the future directions for this PhD work.}


\section{Old version}
%============================================ OLD VERSION
The first decades of research in Mobile Robotics, from the beginning until 2004, handled the challenges of connecting efficiency and data association. They introduced probabilistic formulations to path planning, exploration, simultaneous localization and mapping (SLAM), and many other areas. Some of the approaches from these areas are still popular nowadays, such as RaoBlackwellised Particle Filters and Extended Kalman Filters. The majority of them were based on ultrasonic or lidar sensors, as these were the most popular and robust sensors at the time. Consequently, the outcome maps were mostly 2D grid ones, in which the cells represented the free, occupied (obstacles), and unknown regions~\cite{Cesar2016Simultaneous}. 

After building a solid foundation for many problems with probabilistic approaches, the research community took a forward step. They concentrated on improving the properties of the already proposed and new approaches like observability, convergence, and consistency~\cite{Cesar2016Simultaneous}.
Simultaneously in this period (2004-2015), visual sensors have been in the spotlight as an alternative to gain information about the environment. Their considerable improvement in data quality and variety (e.g., depth images, point clouds, stereo images) aided their increasing employment. In fact, building 2D and 3D maps with a visual sensor resulted in a new term, Visual SLAM~\cite{Salas2014Dense}. 

Mobile robotics have enjoyed formidable advantages in performing tasks that only expect robots to navigate through free spaces and avoid obstacles. Moving items from point A to point B or vacuuming free spaces are examples of robotics tasks with satisfactory solutions. However, the same level of success does not apply so far to many other high-level tasks that robots are supposed to dealing with nowadays. Since mobile robotics shifted its focus from factory floors and assembly lines to everyday living spaces, robots are demanded to perform human-like tasks in different scenarios that are not necessarily as strict, neat, and organized as the industrial world~\cite{Aydemir2012Exploiting}. Relying only on purely geometric maps, like 2D or 3D maps, and geometric perceptions, such as raw sensor readings, do not allow the mobile robotics going beyond basic representations, which restricts the robot to obtain a high-level understanding of the environment.  %We believe that one of the reasons for robots not prospering as much in high-level tasks is relying only on purely geometric maps and having limited perceptions that do not allow going beyond basic geometry representations to obtain a high-level understanding of the environment.  
The robots are deprived of processing the environmental data to infer or estimate extra valuable knowledge useful in various tasks. This might be the reason for robots not prosper as much in high-level tasks.

As aforementioned, 
%copiei daqui
high-level robotics tasks demand the robots to read the environment similarly to humans, which reason over many characteristics of the room or objects and not only over the size of the free or occupied areas. The robot's geometric perception, i.e., raw sensor readings that generate standard geometric maps, is not descriptive and informative enough for providing such improvement demanded by the high-level tasks. This limitation does not mean that geometric maps are useless or irrelevant these days. On the contrary, they are still helpful and significant for the robot's safety navigation or path planning. However, there is a demand for complementing and extending the robot's geometric perception with meaningful knowledge from the environment. We then argument that the high-level information inferred from the sensor readings, also called semantic information, may be heavily exploited to complement the robot's perception when building autonomous robots.
%ate aqui

The association of semantic information (or concepts) to geometric entities in the map is called semantic mapping, one of the newest topics the researchers have explored. It enhances the robot's autonomy and robustness in many ways, besides facilitating some challenging tasks~\cite{Cesar2016Simultaneous}. Autonomous cars are a great example of a robotic application that requires a human-like understanding of the environment. Figure~\ref{fig:waymo_point_cloud} depicts a point cloud from the Waymo's autonomous car while it is driving itself in the road. The raw point cloud does not differentiate the obstacles around the car since it only indicates where they are and which regions are free. If the car relies only on this point cloud, it can move by the road avoiding obstacles, but it is far from behaving like a proper driver following all traffic rules. Its geometric perception does not specify, for example, which obstacles are static (trees, curbs, or traffic plates) or dynamic (pedestrians, cyclists, or cars), which is vital for everyone's safety. However, with the aid of semantic information, the car's perception goes beyond just detecting objects such as traffic signs, cars, and people, Figure~\ref{fig:waymo_pedestrian_crossing_cars}. The car understands which traffic signs are open or closed by the color of their lights, where the cars are intended to go, and the type of the cars (regular or police), Figure~\ref{fig:waymo_pedestrian_crossing_cars_traffic_light_paths}. Human drivers naturally and quickly understand the scene in Figure~\ref{fig:waymo_clean_environment}, but the same can not be said about robots. 


\begin{figure}[h]
     \centering
     \begin{subfigure}[b]{0.493\columnwidth}
         \centering
         \includegraphics[width=\textwidth]{figs/waymo_point_cloud.png}
         \caption{}
         \label{fig:waymo_point_cloud}
     \end{subfigure}~~
     \begin{subfigure}[b]{0.49\columnwidth}
         \centering
         \includegraphics[width=\textwidth]{figs/waymo_clean_environment.png}
         \caption{}
         \label{fig:waymo_clean_environment}
     \end{subfigure}
     \\[.5em]     
     \begin{subfigure}[b]{0.49\columnwidth}
         \centering
         \includegraphics[width=\textwidth]{figs/waymo_pedestrian_crossing_cars.png}
         \caption{}
         \label{fig:waymo_pedestrian_crossing_cars}
     \end{subfigure}~~  
     \begin{subfigure}[b]{0.49\columnwidth}
         \centering
         \includegraphics[width=\textwidth]{figs/waymo_pedestrian_crossing_cars_traffic_light_paths.png}
         \caption{}
         \label{fig:waymo_pedestrian_crossing_cars_traffic_light_paths}
     \end{subfigure}
     \caption[Waymo's autonomous car driving itself in a city.]{\small Waymo's autonomous car driving itself in a city. The point cloud, (a), is read by one of its embedded sensors. At every moment, (b), the car detects all the surroundings objects, (c), and infers their meanings before making decisions, (d). Images taken from one of the videos from Waymo\footnotemark.}
     \label{fig:waymo_geometric_semantic}
 \end{figure}
%\footnote{https://www.youtube.com/watch?v=B8R148hFxPw}

\section{Hypothesis and Goals}

We hypothesize that semantic information associated with the spatial and temporal organization of the environment bridge the gap limiting mobile robotics from improving towards high-level tasks. Our idea is that the way environments are organized and how they change over time provides information that can be inferred by the robot's system, which is more informative than only the lidar, sonar, or camera sensors reading. As semantic information is more like a specific knowledge for each task and inferred from the robot's surroundings than a generic type of data from a sensor reading, several questions must be answered before using it as part of the solution for a robotic task. We see the following as noteworthy challenges\footnotetext{Waymo \ang{360} Experience: A Fully Autonomous Driving Journey. youtube.com/watch?v= B8R148hFxPw}: 
\begin{itemize}
	\item Deciding on which type of semantic information is  relevant to the task and if it is possible to infer and associate to the robot’s surroundings;
	\item How to perform the inference or estimation of the semantic information;
	\item How to use semantic information to improve the robot's performance in a particular task.
\end{itemize}

Since semantic information is relatively new in the literature, the first point is frequently discussed considering a geometric perspective, without estimations, instead of a semantic one. For example, an autonomous vacuum cleaner relies on a 2D grid map to clean the free space. Hence, the raw readings from a 2D lidar sensor are enough for the robot to build the grid map and perform its task. On the contrary, the semantic perspective for any robotic task would be: which is the information that is not explicitly available in the environment but could be inferred or estimated to improve the robot's performance? To answer this question is necessary to deeply understand the task and the general environment characteristics where the robot operates. An inspiration for answering this point is to consider how humans reason under the same circumstance and solve such a task, and how they process the environment's information to accomplish the task efficiently.

Second, depending on the needed semantic information, it may be necessary to use methods based on machine learning to estimate it. For example, training a deep learning model for estimating terrain traversability for an outdoor ground robot may provide a suitable result. However,  besides the training requirement, the quality of the solution depends on the training data, and this approach does not scale well. Probabilistic-based estimations appear as a second option, as it does not require a large set of data for training, and accepts a wide range of different models.

The third and last point, the proper use of the inferred semantic information in the robot's system, is crucial for successful task completion. As the robot gains more information from the environment, it is important to keep updating the estimations, and it is even better if the estimations become more robust over time. 

The exploitation of semantic information in robotics is an idea that has recently gained attention from researchers, and thus, most of the challenging problems are still unsolved. A simple way of pushing the limits further and investigate these problems is to study the advantages of semantic information in different areas. In this thesis, we have chosen a task with a high difficulty level that can benefit from semantic information: object search (OS) in indoor and unknown environments, a yet unsolved problem in robotics. 

In OS tasks, the robot's goal is to find a target object in the environment with a visual sensor. Usually, the environment is unknown to the robot, and the data it uses for searching are gathered with its own sensors. Since we are complementing the robot's perception with semantic information models for different OS tasks, the extra knowledge from the environment inferred by the robot has to aid the OS searching by reducing the search space. The robot plans a search strategy that estimates the most promising regions to contain the target object. This thesis exploits the improvements in OS tasks by the use of semantic information inferred from two different data sources disregarded by the research community: text and dynamic obstacles. 

%The first years of research in the field of Mobile Robotics saw the introduction of many probabilistic formulations for SLAM, path planning,  
%-At the beginning, Robotics was interested in estimating the obstacle's positions and the free space in the environment (Robotics has started with robots operating in assembly lines in factories, and now it is shifting to everyday living spaces)
%
%
%
%-However, this field has evolved and expanded the varieties of places the robots operate
%-By operating in more different environment and developing many tasks, researchers started including many sensors to the robots in order to make it more capable of acquiring data
%-However, that is not necessary if we can process the sensor readings and estimate more information besides the raw data
%
%Robotics has been changing its focus from factory floors to everyday living spaces, such as offices, houses, hospitals, airports, and etc~\cite{Aydemir2012Exploiting}. 


\section{Contributions}
Parts of this thesis have been previously published or submitted as journal articles. The following publications are the results of research carried during this PhD:

\begin{enumerate}
	\item MANTELLI, M. et al. Temporal object search system based on heat maps. \textit{Journal of intelligent \& robotic systems} (in review), Springer, v. 101, n. 2, p. 1–23, 2021.\\
\textbf{	Summary and Individual contribution:} This paper is on how to search a target object in unknown and dynamic environments efficiently. As opposed to other OS works that consider the objects' position static and ignore the human-object interaction, the idea presented in this work is to incorporate a person's routine and habits in the search strategy. This work aimed to model the semantic information of how objects are moved over time within an environment and use the inferred information to reduce the searching space. This idea came from observing how the objects are placed over time and that every person has their own singularities in terms of object placement.
The contribution of the author of this thesis is in modeling the semantic information as part of the search strategy and in building a heat map with the inferred data.  

	\item MANTELLI, M. et al. Semantic active visual search system based on text information for large and unknown environments. \textit{Journal of intelligent \& robotic systems}, Springer, v. 101, n. 2, p. 1–23, 2021.\\
\textbf{	Summary and Individual contribution:} This paper is on how to find a target door label based on text analysis. Although humans heavily rely on texts for accomplishing several tasks, text as a data source is not very popular in robotics. In this work, we have argued that texts have a great potential for providing search clues and are often found in man-made environments. This idea came from the human behavior when searching for a someone's office in an unknown building, and how the door labels are used for estimating whether the current corridor is promising for finding the target office. The search strategy relies on the patterns of door labels in indoor scenarios and it reasons over them to estimate which corridor is more promising for achieving the goal.
	
\end{enumerate}

\section{Outline}
The outline of this thesis is as follows. First, in Chapter~\ref{chap:2_theoretical_background}, we introduce a background on the main problems in mobile robotics, as well as the most popular approaches that deal with each problem. Besides, it also presents the general concepts of the OS problem, which is used throughout this thesis along with the basic concepts of mobile robotics. In Chapter~\ref{chap:3_text_os_system}, we provide our first semantic OS system that is based on text as the main source of semantic information. Next, in Chapter~\ref{chap:4_temporal_os_system}, we discuss our second semantic OS system, which is the one that aims to understand how the objects within the environment are moved through a period of time, to make predictions about their future positions. Lastly, in Chapter~\ref{chap:5_discussion_thesis_progress}, we conclude this thesis proposal by discussing the current contributions and draw the future directions for this PhD work.


%\begin{itemize}
%    \item \emph{cite}: Unicórnios são verdes \cite{Adams2009Conceptual};
%    \item \emph{citep}:Unicórnios são verdes \citep{Adams2009Conceptual};
%    \item \emph{citet}: Segundo \citet{Adams2009Conceptual}, unicórnios são
%                        verdes.
%    \item \emph{citen or citenum}: Segundo \citen{Adams2009Conceptual},
%        unicórnios são verdes.
%    \item \emph{citeauthor e citeyearpar}: Segundo artigos de
%        \citeauthor{Adams2009Conceptual} , unicórnios são verdes 
%        \citeyearpar{Adams2009Conceptual}.
%
%\end{itemize}

\chapter{Conclusion and Future Work}
\label{chap:5_discussion_thesis_progress}
In this thesis, we have explore the idea of semantic concepts to geometric entities in the robot's surroundings can be useful for mobile robotics in the context of high-level OS~\cite{Cadena2016Past}. The use of semantics in mobile robots aid to overcome the limitations of purely geometric robot's perception and maps. It aims to enhance robot's autonomy and robustness, as well as facilitate more complex tasks, like OS.

We have investigated the usage of semantic information associated with the spatial and temporal organisation of the environment in the context of OS problems. We argue that the organisational semantic information complements the robot's perception, expanding and improving it to aid in high-level tasks in everyday living spaces. The majority of the proposed solutions to the OS problems depend on geometric information only, such as the objects' 3D format or color, or even require a preparation step, like providing the association between rooms of the environments and possible objects that it may contain. The disadvantages of such proposals are that they are limited to the raw data read by the sensors or are not easy to deploy due to the preparation step they need before operating. This thesis goes beyond the geometric information and sensor readings, inferring organisational semantic information that complements the robot's geometric perception and enhances its performance in OS tasks. 

We have devised two OS systems, NSOS and LSOS, presented in Chapters~\ref{chap:3_text_os_system} and~\ref{chap:4_temporal_os_system}, respectively. They demonstrate that it is possible to model the organisation of both static and dynamic environments as semantic information, and use it for estimating objects' position. We have also demonstrated that these systems can rely on not trivial search clues, like the parity of numbers or the time an object has been detected, to efficiently deal with the OS problem. We present qualitative and quantitative results showing that the proposed OS sytems have systematically outperformed the other OS systems tested in simulation.
%based on semantic information inferred from two sources hardly explored in the mobile robotics field: text and dynamic obstacles. 

In our NSOS system, we exploited the numbers from door labels to find a target door in a large and unknown environment. This strategy is inspired by how humans behave under the same circumstances, i.e., looking for a door label in an unknown environment. Besides comparing each detected number of a door label with the target one and check whether they are equal, humans also reason over the sequence of the door labels to estimate how the door labels are organised within the corridors. The search strategy of our NSOS system combines semantic information, namely the parity and growth properties of the numbers, along with geometric information, the orientation of the corridor and the proximity between the robot and the promising regions, to estimate which region of the building is more favorable for the search. In case the robot is in a corridor that becomes less likely to contain the target door as the robot finds out more door signs, our OS system guides the robot towards another one more promising. For our NSOS system, a number from a door sign is associated to a door, which is the object that our system is actually aiming to find. Since the environment is entirely unknown to our system and no map is provided beforehand, it is doubtful that the robot will take the optimum path until the target door label while searching. Hence, more than guiding the robot straight towards the target's position, our OS system estimates when the current corridor is hopeless, and the robot should go elsewhere. The experiments in simulation and in real environment attest that our NSOS system always accomplishes the OS task, finding the target door. 

In this thesis, we also managed to come up with a way to take advantage of the changes in the organisation of the environment through a period of time. It is no surprise that robots may operate in dynamic environments, and the task of searching for objects in these scenarios becomes even more chalenging. This is the context of other contribution of our thesis, the LSOS sytem. It aims to explore the fact that the environment may change from time to time, and  observing these changes in a long-term way may produce useful information. The search strategy of our LSOS system also works in a way that mimics humans. It assumes that objects are movable in real life, and it tries to understand how they are positioned over time to estimate their future position. The semantic part of this strategy processes the data the system gathers over a while. Then, it estimates the target object's position based on the history of positions of this same object. As the estimations of LSOS system being computed based on the collected data from the same environment, our system can adapt itself to the routine and habits of the person that interact with the objects. Hence, besides being generic and can be deployed in any environment, LSOS also adapts to the local singularities. Besides, the results suggest that the more consistent and repetitive the routine is, the more confident are the estimations. Lastly, the results also shows that the LSOS system can find the target object even when the object's arrangement varies considerably over a period of time. 

The discussion about a possible deployment of both NSOS and LSOS systems to Jaci is worth to be mentioned. Besides showing in some high-level how they could be deployed to Jaci, it also explains the advantage of each system to that SR. Besides, the combination of both OS systems would bring to Jaci a considerable improvement in its autonomy. In addition to increasing Jaci's efficiency in terms of the disinfected area in a certain time, the combination would also take humans out of the look and make the disinfection process safer. The SR Jaci was not available for us to deploy our systems and carry out some experiments during the last few months. However, we still aim to do so as soon as Instor, the company that has built Jaci, gives us permission.

A drawback of our contributions is that they are too specific for their context, mainly the NSOS system. For example, our system is limited to scenarios where the doorn signs are design only with numbers. However, it was projected in a modular way, which allows the easy replacement of our current semantic planner to any other one with other heuristics. In the LSOS system, a drawback is the fact that it looks for instances of object classes. For cases in which the user wants an specific instace of object, e.g. a pair of sneakers (and not just ``shoes'') or Don Quixote (and not just ``book''), our LSOS system may finish the task finding any other instance of these examples. Nonetheless, this simplification is more associated to the limitation of the object detection algorithm, in our case YOLO, than to our LSOS system. Therefore, when the research community proposes an algorithm that is efficient in detecting such a fine level of details from objects, we could replace YOLO for it and provide a better search service. 

There are still interesting new aspects of the object search that could be addressed. Outdoor environments also have great potential for SRs to perform searching tasks, considering that such scenarios present some sort of organisation. For example, most of the cities in the world have their houses numbered according to a certain rules, resulting in a large-scale organised environment. Even better, some cities also have a street naming system along with house numbering one, which could be combined into a coarse to fine approach~\cite{Street1950American}. Thus, if an SR is tasked with finding a specific house in these well-planned cities, its searching system could rely on the street naming rules to coarsely estimate the most likely streets to contain the target house. Next, the searching system could use the house numbering rules to find the target house. In this example, the SR would depend on a computer vision algorithm and a set of visual sensors to read the street names and house numbers. This may not be an issue nowadays with the advance of OCR algorithms and robustness of visual sensors. Besides, the research community has already presented promising results towards that functionality, like~\citet{Oosterman2010Geolocation} that used an iPhone to read the street names in New Zealand a few years ago. 

Finally, using Ontology in the OS problem is also an interesting topic to consider. Ontology aims to describe a hierarchical structure composed by entities and relations for purposes of representation~\cite{Thomas2018Ontology}. It shows the properties of a subject area, along with how they are related, by defining a set of concepts and categories that represent that subject. With the aid of Ontology to formally describe parts of outdoor environments, e.g. different stores, vehicles, and parks, an OS system could improve its performance. New clues could be used by the OS system, mainly the ones that specify the relation between properties of different objects.



%For example, pharmacies may have the words ``drugs'', ``care'', and ``health'' displayed in their façade. Besides, the most popular signs associated to pharmacies are both the green cross and the snake on a staff. Then, an OS system combined with an ontology that defines pharmacy, will be aware about of the properties that can be used as search clues when when searching for a pharmacy. Detecting a green cross and the word ``care'' could be enough for suggesting that there is a pharmacy nearby

%The same can be said about an autonomous car that searchers for a parking lot, and then for a parking slot. 

%We are focusing our efforts on extending our last semantic OS system to understand better the humans' routine, which is ideal for service robots that interact with people for long periods. In a typical scenario, a person does not move an object without reason, i.e., humans do not randomly interact with objects to change their positions. Usually, there is a purpose behind a change in the object's position. We can associate this reason with an activity carried out by a person, such as taking a cup of coffee to the desk in the working area after a break. If we consider a person as the target object, as we did in one of our previous experiments, another example would be a person that moves within a house just when it is needed, as we tend to be inactive whenever possible~\cite{Lieberman2015Exercise}. The sequence of activities carried out by a person, also known as routine (or habits), has different patterns depending on the period of time that is considered. There are the activities that are executed daily, whereas others are weekly or even monthly.  

%Back to our temporal, semantic OS system, imagine a scenario where a person takes their backpack to work, which means it is not within the house during working hours, except for the weekends. In this context, if the person requests our system to find the backpack at any working hour, it would probably estimate that the object is not in the house, even during the weekends. This confusion is because our semantic OS system computes its estimations based on the hours of the day, and at this moment, the weekday concept does not exist, i.e., all weekdays are equal as if the person's routine was the same throughout the week. However, as we argued, this assumption does not hold true, and in some cases, the robot may not find the target object as efficiently as if using this new weekday concept. After incorporating this new, we believe that our temporal semantic OS system will have a more refined understanding of the placement of the objects, and by consequence, will be more accurate in its estimations. In summary, the more our system understands how the objects are moved through a period of time, the more efficient the semantic OS system becomes. 

%Lastly, due to the pandemic situation and the recommendations from the health organizations, we have not been able to perform experiments with the physical robot. We understand that such experiments are important to this research, and including their results in this thesis would be beneficial. Hence, we aim to deploy our algorithm to a robot and carry out several experiments whenever we have better and safer working conditions in the Phi Robotics Research Lab at the Federal University of Rio Grande do Sul. 

%Our semantic, temporal OS system aims to incorporate the person's routine, memorizing the past object's presence information to estimate where it will be in the future. So far, we rely only on the time of the day, but we understand that a person's routine may change depending on the weekday. The HH106 dataset presented a pattern in the human's activities, and it is possible to infer other semantic information based on the weekdays, hence improving the efficiency of our system. As future work, we aim to incorporate the days of the week into our OS system, which is another semantic information helpful in understanding the objects' position changes. Then, the OS system would estimate the most promising region for finding a target object based on both the request hour and weekday for a given request search. Despite demanding more data to start searching, we believe adding this semantic information would make the system more robust and adjustable to a person. Lastly, we were not allowed to conduct experiments with physical robots in our research lab due to the pandemic. When possible, we aim to test our OS system in a real scenario to evaluate its performance. 

%\section{Thesis Schedule}

%Since 2017, when the research work on this PhD study started, two works have been developed, and one more is for the near future. The schedule for the developed and future activities is presented in Table~\ref{chp05b_tab:activities}. The first year we focused on fulfilling most of the requirements from the PhD program, which includes the courses and the qualifying exam. The following year our focus changed to studying the OS problem and the mobile robotics problems necessary to solve it, i.e., exploration, mapping, and localization. For the next year and a half, we have developed our first semantic OS system that relies on text to infer the semantic concepts of parity and growth from door label numbers. In this last year and a half, we have developed our second OS system, based on dynamic agents. A PhD Sandwich opportunity allowed these works to be partially developed at the Robotics and Intelligent Systems (ROBIN) lab, in the University of Oslo, Norway. Lastly, we have already started working on improving our temporal OS system towards the weekday concept, and we hope to finish this study within a few months in the future.

%\pagebreak

%~

%\newcounter{RowNo} \setcounter{RowNo}{0}
%\newcounter{SubRowNo}[RowNo] \setcounter{SubRowNo}{0}
%\renewcommand{\theRowNo}{\Alph{RowNo}}
%\renewcommand{\theSubRowNo}{\theRowNo.\arabic{SubRowNo}}
%\newenvironment{RowNo}{ \refstepcounter{RowNo} \theRowNo. }{}
%\newenvironment{SubRowNo}{ \refstepcounter{SubRowNo} \hspace{20pt}\theSubRowNo. }{}

%\newcounter{magicrownumbers1}
%\newcounter{magicrownumbers2}
%\newcommand\rownumber{\Alph{magicrownumbers1}.}
%\newcommand\RowNo[1]{\refstepcounter{magicrownumbers1}\label{#1}\\[-15pt]\hline\rownumber}
%\newcommand\subrownumber{\hspace{20pt}\rownumber\arabic{magicrownumbers2}.}
%\newcommand\SubRowNo[1]{\refstepcounter{magicrownumbers2}\label{#1}\subrownumber}
%\newcommand\ResetSubRow{\setcounter{magicrownumbers2}{0}}

%\captionof{table}{Activities developed in this work}
%\label{chp05b_tab:activities}
%{
%\footnotesize
%\begin{tabularx}{\linewidth}{p{\textwidth}}
%\toprule
%    \hspace{170pt} \textbf{List of Activities}\\\hline
%    \RowNo \label{chp05b:it-SLAM} Post-Graduate Program requirements:\\
%        \ResetSubRow
%        \SubRowNo \label{chp05b:it-pf} Courses;\\
%        \SubRowNo \label{chp05b:it-graph} Qualifying Exam;\\\hline
%    \RowNo \label{chp05b:it-FSD} Investigation of the OS problem:\\
%        \ResetSubRow
%        \SubRowNo \label{chp05b:it-obs} Study and implementation of Monte Carlo localization based on laser sensors;\\
%        \SubRowNo \label{chp05b:it-kde} Study and implementation of 2D grid mapping techniques based on laser sensors;\\
%        \SubRowNo \label{chp05b:it-mcl} Study and implementation of exploration techniques based on 2D grid maps;\\\hline
%    \RowNo \label{chp05b:it-NGSlam} Semantic OS based on text information in large and unknown environments:\\
%        \ResetSubRow
%        \SubRowNo \label{chp05b:it-ngrams} Study and implementation of optical character recognition (OCR) techniques from RGB images;\\
%        \SubRowNo \label{chp05b:it-word} Development of the framework containing the basic components for our semantic OS system;\\
%        \SubRowNo \label{chp05b:it-match} Development of the search strategy based on OCR;\\
%        \SubRowNo \label{chp05b:it-exp2} Evaluation of the system through experiments in simulated scenarios and with human volunteers;\\\hline
%    \RowNo \label{chp05b:it-long} Temporal, Semantic OS system based on heat maps:\\
%        \ResetSubRow
%        \SubRowNo \label{chp05b:it-lifelong} Study of long-term robotic approaches and dynamic agents modeling;\\
%        \SubRowNo \label{chp05b:it-multi} Study of Robot Operating System (ROS);\\
%        \SubRowNo \label{chp05b:it-multimatch} Development of the framework containing the basic components for our temporal, semantic OS system;\\
%        \SubRowNo \label{chp05b:it-searchstrategy} Development of the search strategy that incorporates dynamic agents in the searching;\\        
%        \SubRowNo \label{chp05b:it-exp3} Evaluation of the system through experiments in simulated scenarios and with datasets;\\\hline
%    \RowNo \label{chp05b:it-longextend} Extending our Temporal, Semantic OS system based on heat maps:\\  
%        \ResetSubRow          
%        \SubRowNo \label{chp05b:it-lifelongext} Extending the search strategy to incorporate the weekday concept;\\        
%        \SubRowNo \label{chp05b:it-exp3ext} Evaluation of the system through experiments in simulated scenarios and with datasets;\\\hline
    
%    \RowNo \label{chp05b:it-SPhD} Doctoral Stay (PhD Sandwich).\\\hline
%    \RowNo \label{chp05b:it-TW} Thesis writing.\\\hline
%    \RowNo \label{chp05b:it-TPD} Thesis Proposal Defense;\\\hline
%    \RowNo \label{chp05b:it-TD} Thesis Defense;\\\hline
%   \RowNo \label{chp05b:it-pub1} Publications associated to this thesis.\\\hline
%    \RowNo \label{chp05b:it-pub2} Other publications.\\\bottomrule
%\end{tabularx}%\\[0.3cm]
%}

%\newcommand{\CellCA}{\cellcolor{green!70!black}}  % CA - concluded activity
%\newcommand{\CellCSA}{\cellcolor{green!50!yellow}} % CSA - concluded subactivity
%\newcommand{\CellOA}{\cellcolor{blue!80!}}  % OA - ongoing activity
%\newcommand{\CellOSA}{\cellcolor{blue!50!}} % OSA - ongoing subactivity
%\newcommand{\CellSA}{\cellcolor{red!80!}}  % SA - scheduled activity
%\newcommand{\CellSSA}{\cellcolor{red!50!}} % SSA - scheduled subactivity
%\newcommand{\CellSA}{\cellcolor{blue!80!}}  % SA - scheduled activity
%\newcommand{\CellSSA}{\cellcolor{blue!50!}} % SSA - scheduled subactivity



%\begin {table}
%    \footnotesize
%    \caption{Schedule followed in this thesis proposal.}
%    \resizebox{\textwidth}{!}{
%        \begin{tabular}{|c|c c c c|c c c c|c c c c|c c c c|c c c c|c c|}
%            \hline
%            \multirow{3}{*}{\textbf{Activities}} & \multicolumn{22}{c|}{\textbf{Period (quarters of year)}} \\ \cline{2-23}
%                             & \multicolumn{4}{c|}{2017} & \multicolumn{4}{c|}{2018} & \multicolumn{4}{c|}{2019} & \multicolumn{4}{c|}{2020} & \multicolumn{4}{c|}{2021} & \multicolumn{2}{c|}{2022}\\
 %                            & Q1 & Q2 & Q3 & Q4 & Q1 & Q2 & Q3 & Q4 & Q1 & Q2 & Q3 & Q4 & Q1 & Q2 & Q3 & Q4 & Q1 & Q2 & Q3 & Q4 & Q1 & Q2\\
%                             & 1 & 2 & 3 & 4 & 5 & 6 & 7 & 8 & 9 & 10 & 11 & 12 & 13 & 14 & 15 & 16 \\
%             \hline
%             \ref{chp05b:it-SLAM}     & \CellCA & \CellCA & \CellCA & \CellCA &  &  &   &   &   &    &    &    &    &  &  &   &   &   &   &  &   &\\
%             \ref{chp05b:it-pf}       & \CellCSA & \CellCSA & \CellCSA & \CellCSA  &   &   &   &   &   &    &    &    &    &    &    &    &   &   &   &  &   &\\
%             \ref{chp05b:it-graph}    &   &   &  &  \CellCSA &  &  &   &   &   &    &    &    &    &    &    &   &   &   &   &  &   &\\\hline

%             \ref{chp05b:it-FSD}       &   &   &   &   & \CellCA  & \CellCA  & \CellCA & \CellCA & &   &    &    &    &    &    &  &   &   & &   &   & \\
%             \ref{chp05b:it-obs}       &   &   &   &   & \CellCSA  & \CellCSA  &  &   &   &   &    &    &    &    &    &  &   &   & &    &   &\\
%             \ref{chp05b:it-kde}       &   &   &   &   &   & \CellCSA  & \CellCSA & &   &   &    &    &    &    &    &  &   &   & &    &   &\\
%             \ref{chp05b:it-mcl}       &   &   &   &   &   &   & \CellCSA & \CellCSA &   &   &    &    &    &    &    &  &   &   & &    &   &\\\hline

%             \ref{chp05b:it-NGSlam}     &   &   &   &   &   &   &   &   & \CellCA & \CellCA & \CellCA &  \CellCA  &  \CellCA  &  \CellCA  & &   &   &   & &   &   &\\
%             \ref{chp05b:it-ngrams}     &   &   &   &   &   &   &   &   & \CellCSA &  \CellCSA  &    &    &    &    &    &  &   &   & &    &   &\\
%             \ref{chp05b:it-word}       &   &   &   &   &   &   &   &   &          & \CellCSA & \CellCSA  & \CellCSA &    &    &    &   &   &   & &   &   &\\
%             \ref{chp05b:it-match}      &   &   &   &   &   &   &   &   &          &          &           & \CellCSA & \CellCSA &    &    &  &   &   & &   &   & \\
%             \ref{chp05b:it-exp2}       &   &   &   &   &   &   &   &   &          &          &           &          & \CellCSA &  \CellCSA  &  &   &   &   & &   &   &\\\hline

%             \ref{chp05b:it-long}       &   &   &   &   &   &   &   &   &          &          &          &  &  & & \CellCA & \CellCA & \CellCA & \CellCA & \CellCA & \CellCA   &   &\\
%             \ref{chp05b:it-lifelong}   &   &   &   &   &   &   &   &   &   &    &  &  &    &    &  \CellCSA  &  &   &   & &    &   &\\
%             \ref{chp05b:it-multi}      &   &   &   &   &   &   &   &   &   &    &    &  &  &    &  \CellCSA  & \CellCSA &   &   & &   &   & \\
%             \ref{chp05b:it-multimatch} &   &   &   &   &   &   &   &   &   &    &    &  &  &    &    & \CellCSA & \CellCSA  & \CellCSA  & &   &   & \\
%             \ref{chp05b:it-searchstrategy} &   &   &   &   &   &   &   &   &   &    &    &  &  &    &    &  &   &  \CellCSA & \CellCSA &   &   & \\             
%             \ref{chp05b:it-exp3}       &   &   &   &   &   &   &   &   &   &    &    &    &  &  &  &  &   &   & \CellCSA & \CellCSA   &   & \\\hline

%             \ref{chp05b:it-longextend} &   &   &   &   &   &   &   &   &   &    &    &  &  &    &    &  & &  & & \CellSA  & \CellSA  & \CellSA \\
%             \ref{chp05b:it-lifelongext} &   &   &   &   &   &   &   &   &   &    &    &  &  &    &    &  &   &   & & \CellSSA  & \CellSSA   & \\             
%             \ref{chp05b:it-exp3ext}       &   &   &   &   &   &   &   &   &   &    &    &    &  &  &  &  &   &   & & & \CellSSA  & \CellSSA \\\hline

%			 \ref{chp05b:it-SPhD}    &   &   &   &   &   &   &   &   &   &    &  \CellCA  & \CellCA   & \CellCA & \CellCA & &   &   &   &   &    &    &   \\\hline
 %            \ref{chp05b:it-TW}      &   &   &   &   &   &   &   &   &   &    &    &    &  & & &   &   &   & \CellCA & \CellCA  & \CellSA  & \CellSA \\\hline
%             \ref{chp05b:it-TPD}     &   &   &   &   &   &   &   &   &   &    &    &    &    &   &   &   &   &   & & \CellSA  &   & \\\hline
%             \ref{chp05b:it-TD}      &   &   &   &   &   &   &   &   &   &    &    &    &    &    &    &   &   &   & &   &   & \CellSA \\\hline
%             \ref{chp05b:it-pub1}    &   &   &   &   &   &   &   &   &   &  &   &  &    &   &  &  & \CellCA  &   & & \CellSA &   & \\\hline
%             \ref{chp05b:it-pub2}    &   &   &   & \CellCA  &   & \CellCA &  & & \CellCA  &  & & \CellCA &    & \CellCA   & \CellCA  &  \CellCA &   & \CellCA & & \CellCA &   & \\\hline
%        \end{tabular}
%    }
%    \vspace{0.2cm}
%    \\
%    \centering
%    \resizebox{0.65\textwidth}{!}{
%        \begin{tabular}{ c c c c c }
%          \CellCA & Concluded activity & & \CellSA & Ongoing/Remaining activity \vspace{0.1cm}\\ 
%          \CellCSA  & Concluded sub-activity & & \CellSSA & Ongoing/Remaining sub-activity \vspace{0.1cm} \\
%        \end{tabular}
%    }
%    \resizebox{0.85\textwidth}{!}{
%        \begin{tabular}{ c c c c c c c c }
%          \CellCA & Concluded activity & & \CellOA & Ongoing activity & & \CellSA & Scheduled activity \vspace{0.1cm}\\ 
%          \CellCSA  & Concluded sub-activity & & \CellOSA & Ongoing sub-activity & & \CellSSA & Scheduled sub-activity \vspace{0.1cm} \\
%        \end{tabular}
%    }
%    \label{chp05b_tab:schedule}
%\end{table}
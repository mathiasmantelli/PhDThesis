\chapter{Discussion on thesis progress}

In this thesis proposal, we have investigated the usage of semantic information in the context of OS problems. Our focus is on expanding and improving the robots' perception to perform high-level tasks in everyday living spaces. The majority of the proposed solutions to the OS problems depend on geometric information only, such as the object's 3D format or color, or require a preparation step, like informing which room is more likely to contain which object. The disadvantages of such proposals are that they are limited to the raw data read by the sensors and are not easy to deploy due to the preparation step they need before operating. Our approaches go beyond the geometric information and just the sensor readings, inferring semantic information that complements the robot's perception and enhances its performance in OS tasks. 

We proposed two OS systems based on semantic information inferred from two sources hardly explored in the mobile robotics field, text and dynamism. In the first system, we exploited the text of door labels, which were numbers, to find a target door in a large and unknown environment. 
This strategy is inspired by how humans behave under the same circumstances, i.e., looking for a door label in an unknown environment. Besides comparing each detected door label with the target one to check whether they are equal, humans also reason over the sequence of the door labels to estimate if the current corridor is promising to bring them to the target door. The search strategy of our first system combines semantic information (the parity and growth properties of the numbers) and geometric information (the corridor's orientation and the distance between the robot and the promising regions) to estimate which corridor of the building is more favorable. In case the robot is in a corridor that becomes less likely to contain the target door as the robot moves through it, the OS system guides the robot towards another one. Since the environment is entirely unknown to the system and no map is provided beforehand, it is doubtful that the robot will take the optimum path until the target door label while searching. Hence, more than guiding the robot straight towards the target's position, our OS system estimates when the current corridor is hopeless, and the robot should go elsewhere. The experiments in simulation and the physical world attest to that behavior from our semantic OS system, in which it always accomplishes the OS task. 

The second OS system aims to explore another source of information that promises to be helpful but neglected by the research community, the dynamic agents in the environment. The search strategy of this system also works in a way that mimics humans, because it assumes that objects are movable in real life, and it tries to understand how they move to estimate their future position. The semantic part of this strategy processes the data the system gathers over a while and calculates the target object's position based on its history of positions. As the estimations of this system are computed based on the collected data from the same environment, it allows our system to adapt to the person's routine and habits that interact with the objects. Hence, even though our system is generic and can be deployed in any environment, it adapts to the local singularities. Besides, the results of this system suggest that the more constant and repetitive the routine is, the more confident are the estimations. However, as also shown by the results, our system finds the target object even when the object's placement has not a pattern. 

\section{Ongoing and Future Work}

Our semantic, temporal OS system aims to incorporate the person's routine, memorizing the past object's presence information to estimate where it will be in the future. So far, we rely only on the time of the day, but we understand that a person's routine may change depending on the weekday. The HH106 dataset presented a pattern in the human's activities, and it is possible to infer other semantic information based on the weekdays, hence improving the efficiency of our system. As future work, we aim to incorporate the days of the week into our OS system, which is another semantic information helpful in understanding the objects' position changes. Then, the OS system would estimate the most promising region for finding a target object based on both the request hour and weekday for a given request search. Despite demanding more data to start searching, we believe adding this semantic information would make the system more robust and adjustable to a person. Lastly, we were not allowed to conduct experiments with physical robots in our research lab due to the pandemic. When possible, we aim to test our OS system in a real scenario to evaluate its performance. 

\section{Thesis Schedule}

The works on this PhD study started back in 2013. All developed, ongoing and future activities are listed in Table \ref{chp05b_tab:activities}. The schedule for these activities is presented in Table \ref{chp05b_tab:schedule}.
In short, we spent the initial one and a half years of the period studying robot state estimation, with focus on graph-based SLAM approaches, culminating in the realization of the qualifying exam. During the following two years we have developed our approach for translating raw sensor readings obtained by a robot into a simplified text and solving the robot state estimation problem by finding matches of words in texts. In this last year, we are working on the long-term place recognition strategy. % quickly described in previous section
%and also preparing the thesis.

\pagebreak

~

\newcounter{RowNo} \setcounter{RowNo}{0}
\newcounter{SubRowNo}[RowNo] \setcounter{SubRowNo}{0}
\renewcommand{\theRowNo}{\Alph{RowNo}}
\renewcommand{\theSubRowNo}{\theRowNo.\arabic{SubRowNo}}
\newenvironment{RowNo}{ \refstepcounter{RowNo} \theRowNo. }{}
\newenvironment{SubRowNo}{ \refstepcounter{SubRowNo} \hspace{20pt}\theSubRowNo. }{}

%\newcounter{magicrownumbers1}
\newcounter{magicrownumbers2}
%\newcommand\rownumber{\Alph{magicrownumbers1}.}
%\newcommand\RowNo[1]{\refstepcounter{magicrownumbers1}\label{#1}\\[-15pt]\hline\rownumber}
%\newcommand\subrownumber{\hspace{20pt}\rownumber\arabic{magicrownumbers2}.}
%\newcommand\SubRowNo[1]{\refstepcounter{magicrownumbers2}\label{#1}\subrownumber}
\newcommand\ResetSubRow{\setcounter{magicrownumbers2}{0}}

\captionof{table}{Activities developed in this work}
\label{chp05b_tab:activities}
{
\footnotesize
\begin{tabularx}{\linewidth}{p{\textwidth}}
\toprule
    \hspace{170pt} \textbf{List of Activities}\\\hline
    \RowNo \label{chp05b:it-SLAM} Thorough studies of mobile robotics foundations:\\
        \ResetSubRow
        %\SubRowNo \label{chp05b:it-kf} Study and implementation of approaches based on Kalman filter: Extended (EKF-SLAM) and Unscented (UKF-SLAM);\\
        \SubRowNo \label{chp05b:it-pf} Study and implementation of approaches based on particle filter: Monte Carlo Localization (MCL) and GMapping ;\\
        \SubRowNo \label{chp05b:it-graph} Study of graph-based SLAM, including front-end techniques (place recognition) and back-end techniques (graph optimization);\\
        \SubRowNo \label{chp05b:it-gauss} Study and implementation of the Gauss-Newton back-end for graph-based SLAM approaches;\\\hline
    \RowNo \label{chp05b:it-QE} Qualifying Exam;\\\hline
    \RowNo \label{chp05b:it-FSD} Free-Space Density (FSD) - translating sensor measurements into simple observation values:\\
        \ResetSubRow
        \SubRowNo \label{chp05b:it-obs} Study of observation model techniques for robots equipped with laser sensors;\\
        \SubRowNo \label{chp05b:it-kde} Development of a novel observation model (FSD) based on kernel density estimation;\\
        \SubRowNo \label{chp05b:it-mcl} Application of the FSD in Monte Carlo Localization;\\
        \SubRowNo \label{chp05b:it-exp1} Evaluation of the method through experiments in simulated and real scenarios;\\\hline
    \RowNo \label{chp05b:it-NGSlam} N-Gram SLAM - Translating simple observation values into words:\\
        \ResetSubRow
        \SubRowNo \label{chp05b:it-ngrams} Study of techniques from shallow linguistic processing, focusing on $n$-grams;\\
        \SubRowNo \label{chp05b:it-word} Development of a compact word representation for contiguous regions;\\
        \SubRowNo \label{chp05b:it-match} Development of a SLAM front-end based on the matching of words using $n$-grams;\\
        \SubRowNo \label{chp05b:it-exp2} Evaluation of the method through experiments in simulated and real scenarios;\\\hline
    \RowNo \label{chp05b:it-long} Extending the N-Gram SLAM for lifelong operation (i.e. dealing with semi-static environments):\\
        \ResetSubRow
        \SubRowNo \label{chp05b:it-lifelong} Study of long-term SLAM approaches;\\
        \SubRowNo \label{chp05b:it-multi} Development of a multi-level word representation to describe semi-static regions;\\
        \SubRowNo \label{chp05b:it-multimatch} Extending the N-Gram SLAM for matching multi-level words;\\
        \SubRowNo \label{chp05b:it-exp3} Evaluation of the method through experiments in simulated and real scenarios;\\\hline
    \RowNo \label{chp05b:it-TW} Thesis writing.\\\hline
    \RowNo \label{chp05b:it-TPD} Thesis Proposal Defense;\\\hline
    \RowNo \label{chp05b:it-TD} Thesis Defense;\\\hline
    \RowNo \label{chp05b:it-pub1} Publications associated to this thesis.\\\hline
    \RowNo \label{chp05b:it-pub2} Other publications.\\\bottomrule
\end{tabularx}%\\[0.3cm]
}

\newcommand{\CellCA}{\cellcolor{green!70!black}}  % CA - concluded activity
\newcommand{\CellCSA}{\cellcolor{green!50!yellow}} % CSA - concluded subactivity
%\newcommand{\CellOA}{\cellcolor{blue!80!}}  % OA - ongoing activity
%\newcommand{\CellOSA}{\cellcolor{blue!50!}} % OSA - ongoing subactivity
%\newcommand{\CellSA}{\cellcolor{red!80!}}  % SA - scheduled activity
%\newcommand{\CellSSA}{\cellcolor{red!50!}} % SSA - scheduled subactivity
\newcommand{\CellSA}{\cellcolor{blue!80!}}  % SA - scheduled activity
\newcommand{\CellSSA}{\cellcolor{blue!50!}} % SSA - scheduled subactivity


\begin {table}
    \footnotesize
    \caption{Schedule followed in this thesis proposal.}
    \resizebox{\textwidth}{!}{
        \begin{tabular}{|c|c c c c|c c c c|c c c c|c c c c|c c c c|}
            \hline
            \multirow{3}{*}{\textbf{Activities}} & \multicolumn{20}{c|}{\textbf{Period (quarters of year)}} \\ \cline{2-21}
                             & \multicolumn{4}{c|}{2017} & \multicolumn{4}{c|}{2018} & \multicolumn{4}{c|}{2019} & \multicolumn{4}{c|}{2020} & \multicolumn{4}{c|}{2021}\\
                             & Q1 & Q2 & Q3 & Q4 & Q1 & Q2 & Q3 & Q4 & Q1 & Q2 & Q3 & Q4 & Q1 & Q2 & Q3 & Q4 & Q1 & Q2 & Q3 & Q4\\
%                             & 1 & 2 & 3 & 4 & 5 & 6 & 7 & 8 & 9 & 10 & 11 & 12 & 13 & 14 & 15 & 16 \\
             \hline
             \ref{chp05b:it-SLAM}     & \CellCA & \CellCA & \CellCA & \CellCA & \CellCA & \CellCA &   &   &   &    &    &    &    &  &  &   &   &   &   &\\
             \ref{chp05b:it-kf}       & \CellCSA & \CellCSA & \CellCSA &   &   &   &   &   &   &    &    &    &    &    &    &    &   &   &   &\\
             \ref{chp05b:it-pf}       & \CellCSA & \CellCSA & \CellCSA &   &   &   &   &   &   &    &    &    &    &    &    &    &   &   &   &\\
             \ref{chp05b:it-graph}    &   &   & \CellCSA & \CellCSA & \CellCSA & \CellCSA &   &   &   &    &    &    &    &    &    &   &   &   &   &\\
             \ref{chp05b:it-gauss}    &   &   &   &   & \CellCSA & \CellCSA &   &   &   &    &    &    &    &    &    &    &   &   &   &\\\hline

             \ref{chp05b:it-QE}      &   &   &   &   &  & \CellCA  &   &   &   &    &    &    &    &    &    &  &   &   & &  \\\hline

             \ref{chp05b:it-FSD}       &   &   &   &   &   &   & \CellCA & \CellCA & \CellCA &   &    &    &    &    &    &  &   &   & &  \\
             \ref{chp05b:it-obs}       &   &   &   &   &   &   & \CellCSA &   &   &   &    &    &    &    &    &  &   &   & &  \\
             \ref{chp05b:it-kde}       &   &   &   &   &   &   & \CellCSA & \CellCSA &   &   &    &    &    &    &    &  &   &   & &  \\
             \ref{chp05b:it-mcl}       &   &   &   &   &   &   & \CellCSA & \CellCSA &   &   &    &    &    &    &    &  &   &   & &  \\
             \ref{chp05b:it-exp1}      &   &   &   &   &   &   &   & \CellCSA & \CellCSA &  &    &    &    &    &    &  &   &   & & \\\hline

             \ref{chp05b:it-NGSlam}     &   &   &   &   &   &   &   &   & \CellCA & \CellCA & \CellCA &    &    &    & \CellSA &   &   &   & & \\
             \ref{chp05b:it-ngrams}     &   &   &   &   &   &   &   &   & \CellCSA &    &    &    &    &    &    &  &   &   & &  \\
             \ref{chp05b:it-word}       &   &   &   &   &   &   &   &   & \CellCSA & \CellCSA &   &    &    &    &    &   &   &   & & \\
             \ref{chp05b:it-match}      &   &   &   &   &   &   &   &   & \CellCSA & \CellCSA &   &    &    &    &    &  &   &   & &  \\
             \ref{chp05b:it-exp2}       &   &   &   &   &   &   &   &   &   & \CellCSA & \CellCSA &    &    &    & \CellSSA &   &   &   & & \\\hline

             \ref{chp05b:it-long}       &   &   &   &   &   &   &   &   &   &    & \CellCA & \CellCA & \CellCA & \CellCA & \CellSA &  &   &   & &  \\
             \ref{chp05b:it-lifelong}   &   &   &   &   &   &   &   &   &   &    & \CellCSA & \CellCSA &    &    &    &  &   &   & &  \\
             \ref{chp05b:it-multi}      &   &   &   &   &   &   &   &   &   &    &    & \CellCSA & \CellCSA &    &    &  &   &   & &  \\
             \ref{chp05b:it-multimatch} &   &   &   &   &   &   &   &   &   &    &    & \CellCSA & \CellCSA &    &    &  &   &   & &  \\
             \ref{chp05b:it-exp3}       &   &   &   &   &   &   &   &   &   &    &    &    & \CellCSA & \CellCSA & \CellSSA &  &   &   & & \\\hline

             \ref{chp05b:it-TW}      &   &   &   &   &   &   &   &   &   &    &    &    & \CellCA  & \CellCA  &  \CellSA  & \CellSA  &   &   & & \\\hline
             \ref{chp05b:it-TPD}     &   &   &   &   &   &   &   &   &   &    &    &    &    &   & \CellSA  &   &   &   & & \\\hline
             \ref{chp05b:it-TD}      &   &   &   &   &   &   &   &   &   &    &    &    &    &    &    &  \CellSA &   &   & & \\\hline
             \ref{chp05b:it-pub1}    &   &   &   &   &   &   &   &   &   & \CellCA &   & \CellCA &    &   &  \CellSA  &  \CellSA &   &   & & \\\hline
             \ref{chp05b:it-pub2}    &   &   & \CellCA &   &   & \CellCA & \CellCA & \CellCA &   & \CellCA & \CellCA &    &    &    &   &  &   &   & & \\\hline
        \end{tabular}
    }
    \vspace{0.2cm}
    \\
    \centering
    \resizebox{0.65\textwidth}{!}{
        \begin{tabular}{ c c c c c }
          \CellCA & Concluded activity & & \CellSA & Ongoing/Remaining activity \vspace{0.1cm}\\ 
          \CellCSA  & Concluded sub-activity & & \CellSSA & Ongoing/Remaining sub-activity \vspace{0.1cm} \\
        \end{tabular}
    }
%    \resizebox{0.85\textwidth}{!}{
%        \begin{tabular}{ c c c c c c c c }
%          \CellCA & Concluded activity & & \CellOA & Ongoing activity & & \CellSA & Scheduled activity \vspace{0.1cm}\\ 
%          \CellCSA  & Concluded sub-activity & & \CellOSA & Ongoing sub-activity & & \CellSSA & Scheduled sub-activity \vspace{0.1cm} \\
%        \end{tabular}
%    }
    \label{chp05b_tab:schedule}
\end{table}
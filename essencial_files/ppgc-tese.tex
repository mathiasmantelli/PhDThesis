 %
% exemplo genérico de uso da classe iiufrgs.cls
% $Id: iiufrgs.tex,v 1.1.1.1 2005/01/18 23:54:42 avila Exp $
%
% This is an example file and is hereby explicitly put in the
% public domain.
%
\documentclass[ppgc,prop-tese, english]{iiufrgs}
% Para usar o modelo, deve-se informar o programa e o tipo de documento.
% Programas :
%   * cic       -- Graduação em Ciência da Computação
%   * ecp       -- Graduação em Ciência da Computação
%   * ppgc      -- Programa de Pós Graduação em Computação
%   * pgmigro   -- Programa de Pós Graduação em Microeletrônica
%   
% Tipos de Documento:
%   * tc                -- Trabalhos de Conclusão (apenas cic e ecp)
%   * diss ou mestrado  -- Dissertações de Mestrado (ppgc e pgmicro)
%   * tese ou doutorado -- Teses de Doutorado (ppgc e pgmicro)
%   * ti                -- Trabalho Individual (ppgc e pgmicro)
% 
% Outras Opções:
%   * english    -- para textos em inglês
%   * openright  -- Força início de capítulos em páginas ímpares (padrão da
%                   biblioteca)
%   * oneside    -- Desliga frente-e-verso
%   * nominatalocal -- Lê os dados da nominata do arquivo nominatalocal.def

\usepackage{amsmath,amssymb,amsfonts}
\usepackage{amsmath}
\usepackage{siunitx}

% Use unicode
\usepackage[utf8]{inputenc}   % pacote para acentuação

% Necessário para incluir figuras
\usepackage{graphicx}           % pacote para importar figuras
\usepackage{subcaption}
\usepackage{mathtools}
\captionsetup{compatibility=false}

\usepackage{times}              % pacote para usar fonte Adobe Times
% \usepackage{palatino}
% \usepackage{mathptmx}          % p/ usar fonte Adobe Times nas fórmulas

\usepackage[alf,abnt-emphasize=bf]{abntex2cite}	% pacote para usar citações abnt


\hyphenation{ro-bo-tics}
\DeclareMathOperator*{\argmax}{arg\,max}
\DeclareMathOperator*{\argmin}{arg\,min}
\newcommand{\bs}{\boldsymbol}

%
% Informações gerais
%
\title{Exploiting semantic information in indoor environments}

\author{Mantelli}{Mathias Fassini}
% alguns documentos podem ter varios autores:
%\author{Flaumann}{Frida Gutenberg}
%\author{Flaumann}{Klaus Gutenberg}

% orientador e co-orientador são opcionais (não diga isso pra eles :))
\advisor[Profa.~Dra.]{Kolberg}{Mariana Luderitz}
\coadvisor[Prof.~Dr.]{Maffei}{Renan de Queiroz}

% a data deve ser a da defesa; se nao especificada, são gerados
% mes e ano correntes
%\date{maio}{2001}

% o local de realização do trabalho pode ser especificado (ex. para TCs)
% com o comando \location:
%\location{Itaquaquecetuba}{SP}

% itens individuais da nominata podem ser redefinidos com os comandos
% abaixo:
% \renewcommand{\nominataReit}{Prof\textsuperscript{a}.~Wrana Maria Panizzi}
% \renewcommand{\nominataReitname}{Reitora}
% \renewcommand{\nominataPRE}{Prof.~Jos{\'e} Carlos Ferraz Hennemann}
% \renewcommand{\nominataPREname}{Pr{\'o}-Reitor de Ensino}
% \renewcommand{\nominataPRAPG}{Prof\textsuperscript{a}.~Joc{\'e}lia Grazia}
% \renewcommand{\nominataPRAPGname}{Pr{\'o}-Reitora Adjunta de P{\'o}s-Gradua{\c{c}}{\~a}o}
% \renewcommand{\nominataDir}{Prof.~Philippe Olivier Alexandre Navaux}
% \renewcommand{\nominataDirname}{Diretor do Instituto de Inform{\'a}tica}
% \renewcommand{\nominataCoord}{Prof.~Carlos Alberto Heuser}
% \renewcommand{\nominataCoordname}{Coordenador do PPGC}
% \renewcommand{\nominataBibchefe}{Beatriz Regina Bastos Haro}
% \renewcommand{\nominataBibchefename}{Bibliotec{\'a}ria-chefe do Instituto de Inform{\'a}tica}
% \renewcommand{\nominataChefeINA}{Prof.~Jos{\'e} Valdeni de Lima}
% \renewcommand{\nominataChefeINAname}{Chefe do \deptINA}
% \renewcommand{\nominataChefeINT}{Prof.~Leila Ribeiro}
% \renewcommand{\nominataChefeINTname}{Chefe do \deptINT}

% A seguir são apresentados comandos específicos para alguns
% tipos de documentos.

% Relatório de Pesquisa [rp]:
% \rp{123}             % numero do rp
% \financ{CNPq, CAPES} % orgaos financiadores

% Trabalho Individual [ti]:
% \ti{123}     % numero do TI
% \ti[II]{456} % no caso de ser o segundo TI

% Monografias de Especialização [espec]:
% \espec{Redes e Sistemas Distribuídos}      % nome do curso
% \coord[Profa.~Dra.]{Weber}{Taisy da Silva} % coordenador do curso
% \dept{INA}                                 % departamento relacionado

%
% Palavras-chave
% A primeira palavra de cada palavra-chave (que, ironicamente, pode ser várias palavras) deve sempre
% começar em maíuscula (novo padrão da biblioteca), as demais palavras começam em minúscula, exceto
% se são nomes próprios.
%
\keyword{Formatação eletrônica de documentos}
\keyword{Padronização de documentos}
\keyword{Instituto de Informática da UFRGS}
\keyword{\LaTeX}
\keyword{ABNT}
\keyword{UFRGS}

%
% inicio do documento
%
\begin{document}

% folha de rosto
% às vezes é necessário redefinir algum comando logo antes de produzir
% a folha de rosto:
% \renewcommand{\coordname}{Coordenadora do Curso}
\maketitle

% DEDICATORIA
\clearpage
\begin{flushright}
\mbox{}\vfill
{\sffamily\itshape
``If I have seen farther than others,\\
it is because I stood on the shoulders of giants.''\\}
--- \textsc{Sir~Isaac Newton}
\end{flushright}

% AGRADECIMENTOS
%\chapter*{Agradecimentos}
%Agradeço ao \LaTeX\ por não ter vírus de macro\ldots



% Resumo na língua do documento.
\begin{abstract}
Este documento é um exemplo de como formatar documentos para o
Instituto de Informática da UFRGS usando as classes \LaTeX\
disponibilizadas pelo UTUG\@. Ao mesmo tempo, pode servir de consulta
para comandos mais genéricos. \emph{O texto do resumo não deve
conter mais do que 500 palavras.}
\end{abstract}

% Resumo na outra língua, se o documento estiver em português, esse estará em inglês,
% se o documento estiver em inglês, este deve estar em português.
% Devem ser passados como parâmetros: o título e as palavras-chave
% na outra língua, separadas por pontos, sem ponto após a última palavra-chave.
% O mesmo padrão de capitalizar a primeira palavra de cada palavra-chave deve ser seguido aqui.
% Seguindo a ideia de não traduzir nomes próprios, nem UFRGS, nem ABNT, nem "Instituto de Informática
% da UFRGS" são traduzidos.
\begin{englishabstract}{Using \LaTeX\ to Prepare Documents at II/UFRGS}{Electronic formatting of documents. Instituto de Informática da UFRGS. \LaTeX. ABNT. UFRGS}
This document is an example on how to prepare documents at II/UFRGS
using the \LaTeX\ classes provided by the UTUG\@. At the same time, it
may serve as a guide for general-purpose commands. \emph{The text in
the abstract should not contain more than 500~words.}
\end{englishabstract}

% lista de figuras
\listoffigures

% lista de tabelas
\listoftables

% lista de abreviaturas e siglas
% o parametro deve ser a abreviatura mais longa
% As abreviações devem estar em ordem alfabética (levando em conta só a abreviação mesmo,
% não o que significa cada palavra).
\begin{listofabbrv}{SPMD}
        \item[OS] Object Search
        \item[SLAM] Simultaneous Localization and Mapping

\end{listofabbrv}

% idem para a lista de símbolos
%\begin{listofsymbols}{$\alpha\beta\pi\omega$}
%       \item[$\sum{\frac{a}{b}}$] Somatório do produtório
%       \item[$\alpha\beta\pi\omega$] Fator de inconstância do resultado
%\end{listofsymbols}

% sumario
\tableofcontents

% aqui comeca o texto propriamente dito

% introducao
\chapter{Introduction}
The first decades of research in Mobile Robotics, from the beginning until 2004, handled the challenges of connecting efficiency and data association. They introduced probabilistic formulations to path planning, exploration, simultaneous localization and mapping (SLAM), and many other areas. Some of the approaches from these areas are still popular nowadays, such as RaoBlackwellised Particle Filters and Extended Kalman Filters. The majority of them were based on ultrasonic or lidar sensors, as they were the most popular and robust sensors at the time. Consequently, the outcome maps were mostly 2D grid ones, in which the cells represented the free, occupied, and unknown regions. 

After building a solid foundation, the research community moved forward, concentrating on improving the properties like observability, convergence, and consistency of the already proposed and the new approaches.
Using visual sensors as one of the main ways to read the environment is another highlight for this period (2004-2015), given the considerable improvement in such sensors regarding the data quality and cameras' size and price. In fact, building 2D and 3D maps from the environment with a visual sensor resulted in a new term, Visual SLAM. 

Simultaneously to algorithmic advances, mobile robotics shifted its focus from factory floors and assembly lines to everyday living spaces. Mobile robotics is increasingly demanded in our daily lives, whether with simple vacuum cleaners or complex autonomous cars. However, this demand for robots to perform high-level tasks in different scenarios, such as a service robot interacting with the objects within the environment or a ground robot avoiding mud terrains, revealed the geometric robots' perception weaknesses.

Despite the progress on the software and hardware fronts, the researchers realized the limitations of purely geometric maps and that the robot's perception should be improved. For example, a vacuum cleaner robot a few years ago would be asked to clean all the free spaces within the environment and avoid obstacles. A 2D grid lidar-based SLAM would be enough for this task, as the robot would map the free space and avoid the obstacles. In contrast, now it has to clean the kitchen on Mondays and the living room on Wednesdays, which brings the question: "what is a kitchen for the robot? Is there a sensor that reads house rooms and informs what each room is?". Hence, the difference between the two versions of robots in this example is the capability of going beyond basic geometry representations to obtain a high-level understanding of the environment. 

The association of semantic information (or concepts) to geometric entities in the map is called semantic mapping, one of the newest topics the researchers have explored. It enhances the robot's autonomy and robustness in many ways, besides facilitating some high-level tasks.
Fig.~\ref{fig:zoox_semantic} is an image from Zoox's autonomous car, and it illustrates the advantage of using semantic information in robotics tasks. The car would probably map this scene with its geometric perception as four obstacles in its front, and two are closer than the other two. Differently, with a semantic perception, the car estimates three people and a truck within the scene. Most importantly, it estimates that one person is distracted using his phone, and another is holding a stop sign. Combining the detection of a walking person and a phone allows the car to estimate the semantic information that this person is likely distracted. Hence, the car should drive itself even more carefully. This whole process is natural for human drivers, but the same can not be said about robots. 


\begin{figure}    
    \centering
    \begin{subfigure}[b]{0.9\columnwidth}
    \includegraphics[width=\textwidth]{figs/zoox_semantic.png}
    \end{subfigure}
    \caption{\small  plane.}
    \label{fig:zoox_semantic}
\end{figure}

As semantic information is more like a specific knowledge inferred from the robot's surroundings than a specific type of data from a sensor reading, several questions need to be answered before using it in a robotic task. We see the following as noteworthy challenges: 
\begin{itemize}
	\item Deciding on what type of semantic information is possible to infer or estimate from the robot's surroundings that is relevant to the task
	\item How to perform the inferring or estimate the semantic information
	\item How to use the semantic information to improve the robot's performance in a given task
\end{itemize}

The first point is frequently discussed in its geometric version, as semantic information is relatively new in the literature. Briefly, for the context of a given robotic task, what information is not explicitly in the environment but could be inferred or estimated to improve the robot's performance? This demands a deep understanding of the task and the general environment characteristics where the robot operates. An inspiration for answering this point is to consider how humans behave and solve such a task and how we connect and process the environment's information to accomplish the task efficiently. 

Second, depending on the needed semantic information, it may be necessary to use methods based on machine learning to estimate it. For example, training a deep learning model for estimating terrain traversability for an outdoor ground robot may provide a suitable result. However,  besides the training requirement, the solution's quality depends on the training data, and this approach does not scale well. Probabilistic-based estimations appear as a second option, as it does not require a large set of data for training, and accepts a wide range of different models.

The third and last point, the proper use of the inferred semantic information in the robot's system, is crucial for successful task completion. As the robot gains more information from the environment, it is important to keep updating the estimations, and it is even better if the estimations become more robust over time. 

The exploitation of semantic information in robotics is an idea that has recently gained attention from researchers, and thus, most of the challenging problems are still unsolved. A simply way of pushing the limits further and exploring these problems is to study the advantages of semantic information in different areas. We have chosen a task with a high difficulty level that can benefit from semantic information, object search (OS) in indoor and unknown environments, a yet unsolved problem in robotics. 

In OS tasks, the robot's goal is to find a target object in the environment with a visual sensor. Usually, the environment is unknown to the robot, and it only guides its moves with the clues it finds out during the search by its sensor readings. 
\\------
\\
The first years of research in the field of Mobile Robotics saw the introduction of many probabilistic formulations for SLAM, path planning,  
-At the beginning, Robotics was interested in estimating the obstacle's positions and the free space in the environment (Robotics has started with robots operating in assembly lines in factories, and now it is shifting to everyday living spaces)



-However, this field has evolved and expanded the varieties of places the robots operate
-By operating in more different environment and developing many tasks, researchers started including many sensors to the robots in order to make it more capable of acquiring data
-However, that is not necessary if we can process the sensor readings and estimate more information besides the raw data

Robotics has been changing its focus from factory floors to everyday living spaces, such as offices, houses, hospitals, airports, and etc~\cite{Aydemir2012Exploiting}. 

%\begin{itemize}
%    \item \emph{cite}: Unicórnios são verdes \cite{Adams2009Conceptual};
%    \item \emph{citep}:Unicórnios são verdes \citep{Adams2009Conceptual};
%    \item \emph{citet}: Segundo \citet{Adams2009Conceptual}, unicórnios são
%                        verdes.
%    \item \emph{citen or citenum}: Segundo \citen{Adams2009Conceptual},
%        unicórnios são verdes.
%    \item \emph{citeauthor e citeyearpar}: Segundo artigos de
%        \citeauthor{Adams2009Conceptual} , unicórnios são verdes 
%        \citeyearpar{Adams2009Conceptual}.
%
%\end{itemize}

\chapter{Theoretical Background}
In the previous chapter, we have argued that multiple robotic tasks would benefit from exploiting the semantic information inferred from everyday environments that surrounds the robot. We have chosen the object search (OS) problem to explore this idea, which aims to estimate a target object's location in a large unknown environment, usually with a camera attached to a mobile robot. We believe investigating this problem can expand our understanding regarding the benefits of employing semantic information to improve the robot's perception. 

This chapter presents a theoretical background detailing techniques used throughout this thesis. The OS problem requires the robot to map the unknown environment and to estimate its position simultaneously. SLAM systems fulfill these requirements, and hence, we address the basic concepts of such systems and other basic concepts in mobile robotics. Besides, we cover the generic and central formulation of OS problems, which is the basis for the works presented in Chapters X and Y [TO DO].

\section{The Basics of Mobile Robotics}
Mobile robots perform several tasks that require them to be aware of their positions in the environment and obstacles' positions to avoid collisions. In most realistic scenarios where the robots are deployed, such information is not directly available. Hence, the robots have to estimate it with their sensors, which provide noisy and partial data from the environment (CITE PROB. ROBOTICS).

The state estimation in mobile robotics can be summarized in four variables: 
\begin{itemize}
	\item $\bs{x}_t$: robot's pose at time step $t$. It is composed by a three dimensional vector containing $(x, y, \theta)^T$, in which $x,y$  represent the position and $\theta$ the orientation. A sequence of robot's poses from time step $0$ to time step $t$ is defined as $\bs{x}_{0:t}= \{ \bs{x}_0, \bs{x}_1, \cdots, \bs{x}_t\}$.
	\item $\bs{m}_i$: object $i$'s position in the environment. A list of $N$ objects, with $1 \leq n \leq N$, in the environment along with their properties is given by the vector $\bs{m} = (\bs{m}_1, \bs{m}_2, \cdots, \bs{m}_N)^T$.
	\item $\bs{u}_t$: control data at instant $t$, and it corresponds to the change of state in the time interval $(t - 1;t]$. The sequence of control data that takes the robot from the initial position to $\bs{x}_t$ is denoted by $\bs{u}_{1:t} = \{\bs{u}_1, \bs{u}_2, \cdots, \bs{u}_t\}$.
	\item $\bs{z}^i_t$: the $i$-th measurement made by the robot at instant $t$. The vector of all of them acquired at the same instant $t$ is $\bs{z}_t = (\bs{z}^1_t, \bs{z}^2_t, \cdots, \bs{z}^K_t)^T$, whereas $\bs{z}_{1:t} = \{\bs{z}_1, \bs{z}_2, \cdots, \bs{z}_t\}$ expresses the history of all observations.	
\end{itemize}

After defining the four variables that are the basic foundation for state estimation in mobile robotics, it is worthing to explain their role in different estimation problems.  The set of controls $\bs{u}_{1:t}$ and measurements $\bs{z}_{1:t}$ are always known since the robot's sensor provides them. Inertial measurement units and wheel encoders are sensors that provide control data, whereas lidars, sonars, and cameras measure the environment. The other two variables, robot's pose $\bs{x}_{0:t}$ and environmental map $\bs{m}$, are not necessarily known. Depending on the estimation problem, it is necessary to estimate different variables, like the three examples depicted in Fig. [REF THE FIG]. In \textit{Localization}, the map is known in advance, and hence, only the robot's pose is estimated. The opposite happens in \textit{Mapping}, as the map is built based on the robot's pose. Lastly, in \textit{SLAM}, which combines the two previous problems, none of them is given a priori, and therefore, both are estimated simultaneously. 

Localization is the most basic perceptual problem in robotics. It aims to determine the robot's pose relative to a given map of the environment. Localization can also be seen as a problem of coordinate transformation, in which it is established a correspondence between the map coordinate system and the robot's local coordinate system. (CITE PROB. ROBOTICS).  There are multiple localization problems, and not each of them is equally difficult. One characteristic that divides this problem into local and global localization is the awareness of the robot's initial pose. The former assumes that the initial robot's pose is known. Therefore, the problem becomes a sort of position tracking in which the noise is adjusted in robot motion commonly by a Gaussian distribution. On the other hand, the latter is unaware of the initial pose, making it perform the localization globally (where the name comes from) in the map. The global localization has a higher difficulty level than the local one, but one of its variations is even more challenging, called the kidnapped robot problem. It addresses the problem of a localized robot being teleported to some other location in that the robot might believe it knows where it is while it does not.  Although a robot is rarely kidnapped in practice, recovering from localization failures is essential for autonomous robots. 

The formulation of the global localization problem is presented in Figure [REF THE FIG], which depicts a few iterations of the robot's pose estimation and how the variables are used. The map $\bs{m} = (\bs{m}_1, \bs{m}_2, \bs{m}_3, \bs{m}_4)^T$ is already known, whereas the $\bs{x}_{0:t}$ must be estimated based on the controls $\bs{u}_{1:t}$ and the measurements $\bs{z}_{1:t}$. For the case of local localization, the $\bs{x}_0$ is known and hence, does not need to be estimated. Markov localization is a probabilistic algorithm that addresses all the localization problems mentioned earlier. It applies the Bayes filter, $p(\bs{x}_t \mid \bs{u}_{1:t}, \bs{z}_{1:t}, \bs{m})$, to transform a probabilistic belief at time $t-1$ into a belief at time $t$.

Many other localization algorithms implement Markov localization in mobile robotics. Three of them have been in the spotlight for a long time and are prevalent in this field: Kalman filter, grid-based filter, and particle filter. The former filters and predicts in linear dynamics and measurement functions (CITE KALMAN), whereas the grid-based filter approximates the estimations by decomposing the state space into finitely many regions of the grid map (CITE GRID). The key idea of the latter, particle filter, is to represent the estimation by a set of random state samples, called particles, drawn from the previous estimation. It can represent a much broader space of distribution, in contrast to the Kalman filter that is more strict to Gaussians (CITE PARTICLE). The particle filter implementation for mobile robotics is also known as Monte Carlo Localization (MCL), widely used in many different robotics applications for multiple robot types. 

Mapping, for the case of the robot's poses are known, is the problem of generating consistent maps from noisy and imprecise measurement data (CITE PROB ROB). The estimated belief of the map, $p(\bs{m} \mid \bs{x}_{1:t}, \bs{z}_{1:t})$, considers the set of all measurements up to time $t$, $\bs{z}_{1:t}$, along with the robot's path defined by its history of all poses, $\bs{x}_{1:t}$, as shown in Fig [REF FIG]. Comparing the graphical models of the localization and mapping problems, [REF FIGS], one can say that they are opposite each other in terms of which variable each estimates. This thought makes sense, since whereas the former relies on $\bs{m}$ to estimate $\bs{x}_{0:t}$, the latter relies on $\bs{x}_{0:t}$ to estimate $\bs{m}$. It is important to mention that the controls $\bs{u}_{1:t}$ play no role in this context, as the path is already known. Besides, the robot's initial pose $\bs{x}_0$ is omitted from the map estimation because no measures are taken when the robot is at that pose.

Similar to the localization problem that groups multiple localization types, the mapping problem also represents a general idea implemented by different map types. The feature-based maps represent the cartesian location of features, which are distinct objects in the physical world, extracted from the measurements, such as (CITE EXAMPLES images from visual sensors or a vector of distances from a 2D lidar.). The advantage of such a map type is the reduction of computational complexity, as the feature space has a lower dimension than the raw measurement. For example, the eight 3D edges of a boudingbox encircling a car are computationally cheaper to process than a point cloud from a 3D lidar. Another map type within the mapping problem is called location-based. It represents in each map component $\bs{m}_i$ the regions from the environment, regardless of whether they contain objects. This way, any location in the world has a label on the map, not only features. Occupancy grid maps are often considered the most popular location-based map (CITE PROB ROB). They discretize the environment into small portions called grid cells, which store information about the area it covers. In general, this information in each cell is a single value representing the probability that an obstacle occupies this cell. The size of the cells defines the map resolution, which brings a tradeoff between the level of details and the demand for memory resources. 
\input{chapters/03_text_semantic_information}


% e aqui vai a parte principal
%
% \chapter{Estado da arte}
% \chapter{Mais estado da arte}
% \chapter{A minha contribuição}
% \chapter{Prova de que a minha contribuição é válida}
% \chapter{Conclusão}

% referencias
% aqui será usado o environment padrao `thebibliography'; porém, sugere-se
% seriamente o uso de BibTeX e do estilo abnt.bst (veja na página do
% UTUG)
%
% observe também o estilo meio estranho de alguns labels; isso é
% devido ao uso do pacote `natbib', que permite fazer citações de
% autores, ano, e diversas combinações desses

\bibliographystyle{abntex2-alf}
\bibliography{biblio}


\appendix

\chapter{Resumo expandido}

\noindent
\textbf{Resolução 02/2021 -- Redação de Teses e Dissertações em Inglês}
Dissertações de Mestrado e Teses de Doutorado do PPGC, bem como outros
trabalhos escritos tais como Proposta de Tese e PEP, poderão ser
redigidas em inglês desde que contenham um título e resumo expandido
redigidos em português. O resumo expandido deve conter no mínimo duas
páginas inteiras, deve aparecer como apêndice e deve conter as
principais contribuições e resultados do trabalho.


\end{document}

%\begin{frame}
%\centering
%\textcolor{red}{\textit{\textbf{NSOS}: using \textbf{numbers} to \textbf{interpret} the \textbf{organisation of the environments}}}
%\end{frame}
%========================================== ############# ======================================================
%========================================== ############# ======================================================
%========================================== ############# ====================================================== 
\begin{frame}{Humans in search tasks}
	\begin{columns}[T]
		\column{.55\textwidth}
		\begin{itemize}
			\item \textbf{Humans} \textbf{searching} in \textbf{unknown environments}
		\end{itemize}			
		\column{.45\textwidth}
			\centering
			\begin{figure}
				 \includegraphics[width=.7\linewidth]{figs/airport_sign.jpg}
				 \caption{\tiny Signs and departure information screen​ in an airport.\footnotemark[11]}			 		 
	         \end{figure}			
	\end{columns}
	\footnotetext[11]{\tiny Figure extracted from \url{https://commons.wikimedia.org/wiki/File:BIAL_signs_and_departure_information_screen.jpg}.}						
\end{frame}
%========================================== ############# ======================================================
%========================================== ############# ======================================================
%========================================== ############# ====================================================== 
\begin{frame}[noframenumbering]{Humans in search tasks}
	\begin{columns}[T]
		\column{.55\textwidth}
		\begin{itemize}
			\item \textbf{Humans} \textbf{searching} in \textbf{unknown environments}		
			\item \textbf{Text} provides \textbf{meaningful information}
		\end{itemize}			
		\column{.45\textwidth}
			\centering
			\begin{figure}
				 \includegraphics[width=.7\linewidth]{figs/airport_sign.jpg}
				 \caption{\tiny Signs and departure information screen​ in an airport.\footnotemark[11]}				 		 
	         \end{figure}			
	\end{columns}
	\footnotetext[11]{\tiny Figure extracted from \url{https://commons.wikimedia.org/wiki/File:BIAL_signs_and_departure_information_screen.jpg}.}						
\end{frame}
%========================================== ############# ======================================================
%========================================== ############# ======================================================
%========================================== ############# ====================================================== 
\begin{frame}[noframenumbering]{Humans in search tasks}
	\begin{columns}[T]
		\column{.55\textwidth}
		\begin{itemize}
			\item \textbf{Humans} \textbf{searching} in \textbf{unknown environments}
			\item \textbf{Text} provides \textbf{meaningful information}
			\item Example of SRs in the context of OS: \textbf{courier SR} 
		\end{itemize}		 %
		\column{.4\textwidth}
			\centering	
			\begin{figure}
			    \begin{subfigure}[b]{.45\columnwidth}
			    	\centering			
			    	\includegraphics[width=.95\linewidth]{figs/corridor_door_signs.jpg}
			    	\caption{\tiny Corridor and door signs.\footnotemark[12]}			 		 
			    \end{subfigure}
			    \begin{subfigure}[b]{.45\columnwidth}
			    	\centering					    	
			    	\includegraphics[width=1.07\linewidth]{figs/lift_robot.png}	         
			    	\caption{\tiny Mobile robot in the lift.\footnotemark[13]}					 
			    \end{subfigure}					 
			    \caption{\tiny Initial works on SLAM.}			 
	         \end{figure}	         		
	\end{columns}
	\footnotetext[12]{\tiny Figure extracted from \url{br.pinterest.com/pin/305611524696049040/}.}
	\footnotetext[13]{\tiny Figure extracted from \url{youtube.com/watch?v=kYmnS_3iSdk}.}
\end{frame}
%========================================== ############# ======================================================
%========================================== ############# ======================================================
%========================================== ############# ====================================================== 
\begin{frame}[noframenumbering]{Humans in search tasks}
	\begin{columns}[T]
		\column{.55\textwidth}
		\begin{itemize}
			\item \textbf{Humans} \textbf{searching} in \textbf{unknown environments}
			\item \textbf{Text} provides \textbf{meaningful information}
			\item Example of SRs in the context of OS: \textbf{courier SR}  
			\item \textbf{Search strategy} in a large-scale unknown building \textbf{plays an important role}
		\end{itemize}			
		\column{.4\textwidth}
			\centering	
			\begin{figure}
			    \begin{subfigure}[b]{.45\columnwidth}
			    		\centering			
					 \includegraphics[width=.95\linewidth]{figs/corridor_door_signs.jpg}
     				 \caption{\tiny Corridor and door signs.\footnotemark[12]}			 		 
			    	\end{subfigure}
			    \begin{subfigure}[b]{.45\columnwidth}
			    		\centering					    	
			    		\includegraphics[width=1.07\linewidth]{figs/lift_robot.png}	         
			    		\caption{\tiny Mobile robot in the lift.\footnotemark[13]}					 
			    	\end{subfigure}					 
				 \caption{\tiny Initial works on SLAM.}			 
	         \end{figure}		         		
	\end{columns}
	\footnotetext[12]{\tiny Figure extracted from \url{br.pinterest.com/pin/305611524696049040/}.}
	\footnotetext[13]{\tiny Figure extracted from \url{youtube.com/watch?v=kYmnS_3iSdk}.}
\end{frame}
%========================================== ############# ======================================================
%========================================== ############# ======================================================
%========================================== ############# ====================================================== 
\begin{frame}[noframenumbering]{Humans in search tasks}
	\begin{columns}[T]
		\column{.55\textwidth}
		\begin{itemize}
			\item \textbf{Humans} \textbf{searching} in \textbf{unknown environments}
			\item \textbf{Text} provides \textbf{meaningful information}
			\item Example of SRs in the context of OS: \textbf{courier SR} 
			\item \textbf{Search strategy} in a large-scale unknown building \textbf{plays an important role}
			\item \textbf{Organisational semantic information} may be useful: \textbf{search cues} inferred from text signs
		\end{itemize}			
		\column{.4\textwidth}
			\centering	
			\begin{figure}
			    \begin{subfigure}[b]{.45\columnwidth}
			    		\centering			
					 \includegraphics[width=.95\linewidth]{figs/corridor_door_signs.jpg}
     				 \caption{\tiny Corridor and door signs.\footnotemark[12]}			 		 
			    	\end{subfigure}
			    \begin{subfigure}[b]{.45\columnwidth}
			    		\centering					    	
			    		\includegraphics[width=1.07\linewidth]{figs/lift_robot.png}	         
			    		\caption{\tiny Mobile robot in the lift.\footnotemark[13]}					 
			    	\end{subfigure}					 
				 \caption{\tiny Initial works on SLAM.}			 
	         \end{figure}		         		
	\end{columns}
	\footnotetext[12]{\tiny Figure extracted from \url{br.pinterest.com/pin/305611524696049040/}.}
	\footnotetext[13]{\tiny Figure extracted from \url{youtube.com/watch?v=kYmnS_3iSdk}.}
\end{frame}
%========================================== ############# ======================================================
%========================================== ############# ======================================================
%========================================== ############# ====================================================== 
%\begin{frame}[noframenumbering]{Introduction}
%	\begin{columns}[T]
%		\column{.55\textwidth}
%		\begin{itemize}
%			\item \textbf{Humans} in \textbf{search tasks}					
%			\item Importance of \textbf{texts} in the \textbf{human visual recognition}
%			\item Robots in the context of OS: \textbf{courier robot} 
%			\item \textbf{Search strategy} in a large-scale unknown building \textbf{plays an important role}
%			\item \textbf{Semantic information} may be useful: \textbf{search cues} inferred from texts
%			\item Level of \textbf{abstraction} of the \textbf{environment}
%		\end{itemize}			
%		\column{.4\textwidth}
%			\centering	
%			\begin{figure}
%			    \begin{subfigure}[b]{.5\columnwidth}
%			    		\centering			
%					 \includegraphics[width=1\linewidth]{figs/corridor_door_signs.jpg}
%     				 \caption{\tiny Corridor and door signs.\footnotemark[12]}			 		 
%			    	\end{subfigure}\\
%			    \begin{subfigure}[b]{.5\columnwidth}
%			    		\centering					    	
%			    		\includegraphics[width=1\linewidth]{figs/lift_robot.png}	         
%			    		\caption{\tiny Mobile robot in the lift.\footnotemark[13]}					 
%			    	\end{subfigure}					 
%				 \caption{\tiny Initial works on SLAM.}			 
%	         \end{figure}	         		
%	\end{columns}
%	\footnotetext[12]{\tiny Figure extracted from \url{br.pinterest.com/pin/305611524696049040/}.}
%	\footnotetext[13]{\tiny Figure extracted from \url{youtube.com/watch?v=kYmnS_3iSdk}.}
%\end{frame}
%========================================== ############# ======================================================
%========================================== ############# ======================================================
%========================================== ############# ======================================================
\begin{frame}{Number-based Semantic OS system (NSOS)}
	\begin{itemize}[<+->]
	
%		\item Number-based Semantic OS system (\textbf{NSOS})
		\item It searches for a specific door sign (\textbf{goal-door})
		\item It was \textbf{evaluated} in \textbf{simulated} and \textbf{real indoor environments}, \textbf{compared} against \textbf{other OS systems} and \textbf{humans participants}
		\item The \textbf{novelties} of NSOS:
		\begin{itemize}[<+->]
			\item \textbf{Target object} and scenario
			\item \textbf{Numbers} and \textbf{organisation of the environment}
			\item Analysis that highlights the \textbf{advantages} of the use of \textbf{organisational semantic information}
		\end{itemize}
	\end{itemize}
\end{frame}
%========================================== ############# ======================================================
%========================================== ############# ======================================================
%========================================== ############# ======================================================
\begin{frame}{NSOS system overview}
	\begin{columns}[T]
		\column{.35\textwidth}
		\begin{itemize}
			\item Our system is composed by:
			\begin{itemize}
				\item Mapping 
				\item Map Segmentation
				\item Image Processing
				\item Semantic Planner
			\end{itemize}
		\end{itemize}
		\column{.65\textwidth}
			\vspace{1.1cm}
			$\Bigg\}$\textit{\textbf{Foundation} of our NSOS system}\\
			$\}$ \textit{\textbf{Core} of our NSOS system}
	\end{columns}
\end{frame} 
%AAAAAAAAAAAAAAAAAAAAAAAAAAAAAAAAAAAAAAAAAAAAAAAAAAAAAAAAAAAAAAAAAAAAAAAAAAAAAAAAAAA
%========================================== ############# ======================================================
%========================================== ############# ======================================================
%========================================== ############# ======================================================
\begin{frame}{NSOS system overview}
	\begin{columns}[T]
		\column{.5\textwidth}
		\begin{itemize}
			\item Foundation of the NSOS system:
			\begin{itemize}
				\item \textbf{Mapping}: 
				\begin{itemize}
					\item It \textbf{builds} and \textbf{updates} a \textbf{2D grid map}
				\end{itemize}
				\item Map Segmentation
%				\begin{itemize}
%					\item It recognizes numbers in door signs and include them into the map
%				\end{itemize}
				\item Image Processing
%				\begin{itemize}
%					\item It segments the free space of the 2D grid map
%				\end{itemize}
%				\item Semantic Planner
%				\begin{itemize}
%					\item It estimates where the SR should during the OS
%				\end{itemize}
			\end{itemize}
		\end{itemize}
		\column{.5\textwidth}
			\begin{figure}
			    \begin{subfigure}[b]{.5\columnwidth}
			    		\centering			
					 \includegraphics[width=.67\linewidth]{figs/2d_map_simulation.png}
     				 \caption{\tiny Mapping in simulation with HIMM.\footnotemark[14]}				 		 
			    	\end{subfigure}\\
			    \begin{subfigure}[b]{.5\columnwidth}
			    		\centering					    	
			    		\includegraphics[width=.75\linewidth]{figs/2d_map_real.png}	         
			    		\caption{\tiny Mapping the real world with Gmapping.\footnotemark[15]}					 
			    	\end{subfigure}					 
				 \caption{\tiny Initial works on SLAM.}			 
	         \end{figure}	
	\end{columns}
	\footnotetext[14]{\tiny Borenstein, J., and Yoram K. ``\textit{Histogramic in-motion mapping for mobile robot obstacle avoidance}.'' Transactions on robotics and automation. 1991.}
	\footnotetext[15]{\tiny Grisetti, G., Stachniss, C., and Burgard, W. ``\textit{Improving grid-based slam with rao-blackwellized particle filters by adaptive proposals and selective resampling.}'' ICRA. 2005.}	
\end{frame} 
%========================================== ############# ======================================================
%========================================== ############# ======================================================
%========================================== ############# ======================================================
\begin{frame}{NSOS system overview}
	\begin{columns}[T]
		\column{.5\textwidth}
		\begin{itemize}
			\item Foundation of the NSOS system:
			\begin{itemize}
				\item Mapping 
%				\begin{itemize}
%					\item It builds and segments a 2D grid map
%				\end{itemize}
				\item \textbf{Map Segmentation}:
				\begin{itemize}
					\item It \textbf{segments} the \textbf{free space} of the \textbf{2D grid map}
				\end{itemize}
				\item Image Processing
%				\begin{itemize}
%					\item It recognizes numbers in door signs and include them into the map
%				\end{itemize}
	%			\item Semantic Planner
%				\begin{itemize}
%					\item It estimates where the SR should during the OS
%				\end{itemize}
			\end{itemize}
		\end{itemize}
		\column{.5\textwidth}
			\begin{figure}
			    \begin{subfigure}{.4\columnwidth}		
					 \includegraphics[width=.85\linewidth]{figs/33.png}
     				 \caption{\tiny The 2D grid map segmented.}			 		 
			    	\end{subfigure}
			    \begin{subfigure}{.4\columnwidth}
			    		\includegraphics[width=.85\linewidth]{figs/44.png}	         
			    		\caption{\tiny The different segments identified by different colors.}					 
			    	\end{subfigure}					 
				 \caption{\tiny The 2D grid maps used in this work.}		 
	         \end{figure}	
	\end{columns}
%	\footnotetext[14]{\tiny Borenstein, J., and Yoram K. ``\textit{Histogramic in-motion mapping for mobile robot obstacle avoidance}.'' Transactions on robotics and automation. 1991.}
%	\footnotetext[15]{\tiny Grisetti, G., Stachniss, C., and Burgard, W. ``\textit{Improving grid-based slam with rao-blackwellized particle filters by adaptive proposals and selective resampling.}'' ICRA. 2005.}	
\end{frame} 
%========================================== ############# ======================================================
%========================================== ############# ======================================================
%========================================== ############# ======================================================
\begin{frame}{NSOS system overview}
	\begin{columns}[T]
		\column{.5\textwidth}
		\begin{itemize}
			\item Foundation of the NSOS system:
			\begin{itemize}
				\item Mapping 
%				\begin{itemize}
%					\item It builds and segments a 2D grid map
%				\end{itemize}
				\item Map Segmentation
%				\begin{itemize}
%					\item It segments the free space of the 2D grid map
%				\end{itemize}
				\item \textbf{Image Processing}:
				\begin{itemize}
					\item It \textbf{recognizes numbers} in \textbf{door signs} and \textbf{include} them into the \textbf{map}
				\end{itemize}
%				\item Semantic Planner
%				\begin{itemize}
%					\item It estimates where the SR should during the OS
%				\end{itemize}
			\end{itemize}
		\end{itemize}
		\column{.5\textwidth}
			\begin{figure}
			    		\centering			
			    		\includegraphics[width=.65\linewidth]{figs/image_processing.png}
			    		\caption{\tiny Illustration of the Image Processing module detecting a door label.}			 
	         \end{figure}	
	\end{columns}
%	\footnotetext[14]{\tiny Borenstein, J., and Yoram K. ``\textit{Histogramic in-motion mapping for mobile robot obstacle avoidance}.'' Transactions on robotics and automation. 1991.}
%	\footnotetext[15]{\tiny Grisetti, G., Stachniss, C., and Burgard, W. ``\textit{Improving grid-based slam with rao-blackwellized particle filters by adaptive proposals and selective resampling.}'' ICRA. 2005.}	
\end{frame} 
%========================================== ############# ======================================================
%========================================== ############# ======================================================
%========================================== ############# ======================================================
%\begin{frame}{NSOS system overview}
%	\begin{columns}[T]
%		\column{.7\textwidth}
%		\begin{itemize}
%			\item Foundation of the NSOS system:
%			\begin{itemize}
%				\item Mapping 
%%				\begin{itemize}
%%					\item It builds and segments a 2D grid map
%%				\end{itemize}
%				\item Image Processing
%%				\begin{itemize}
%%					\item It recognizes numbers in door signs and include them into the map
%%				\end{itemize}
%				\item Map Segmentation
%%				\begin{itemize}
%%					\item It segments the free space of the 2D grid map
%%				\end{itemize}
%%				\item Semantic Planner
%%				\begin{itemize}
%%					\item It estimates where the SR should during the OS
%%				\end{itemize}
%			\end{itemize}
%		\end{itemize}	
%		\column{.3\textwidth}
%	\end{columns}
%	\footnotetext[14]{\tiny Borenstein, J., and Yoram K. ``\textit{Histogramic in-motion mapping for mobile robot obstacle avoidance}.'' Transactions on robotics and automation. 1991.}
%	\footnotetext[15]{\tiny Grisetti, G., Stachniss, C., and Burgard, W. ``\textit{Improving grid-based slam with rao-blackwellized particle filters by adaptive proposals and selective resampling.}'' ICRA. 2005.}	
%\end{frame} 
%AAAAAAAAAAAAAAAAAAAAAAAAAAAAAAAAAAAAAAAAAAAAAAAAAAAAAAAAAAAAAAAAAAAAAAAAAAAAAAAAAAAAAAAAAAAAAAAAAAAAAAAAAAAAAAAAAAAAAAAAAAAAAAAAAAAAAAAAAAA
%========================================== ############# ======================================================        
%========================================== ############# ======================================================
%========================================== ############# ======================================================
%\begin{frame}{Mapping module}
%	\begin{columns}[T]
%		\column{.5\textwidth}
%		\vspace{.95cm}
%		\begin{itemize}
%			\item A \textbf{2D grid} map is build ({\small Simulation and real worlds})
%		\end{itemize}
%		\column{.5\textwidth}	
%			\centering	
%			\begin{figure}
%			    \begin{subfigure}[b]{.6\columnwidth}
%			    		\centering			
%					 \includegraphics[width=.6\linewidth]{figs/2d_map_simulation.png}
%     				 \caption{\tiny Mapping in simulation with HIMM.\footnotemark[14]}		 
%			    	\end{subfigure}
%%			    	\vspace{-.3cm}\\
%			    \begin{subfigure}[b]{.6\columnwidth}
%			    		\centering					    	
%			    		\includegraphics[width=.6\linewidth]{figs/2d_map_real.png}	         
%			    		\caption{\tiny Mapping the real world with Gmapping.\footnotemark[15]}				 
%			    	\end{subfigure}				
%%			    	\vspace{-.7cm}	 
%				 \caption{\tiny The 2D grid maps used in this work.}			 
%	         \end{figure}		         		
%	\end{columns}		
%	\footnotetext[14]{\tiny Borenstein, J., and Yoram K. ``\textit{Histogramic in-motion mapping for mobile robot obstacle avoidance}.'' Transactions on robotics and automation. 1991.}
%	\footnotetext[15]{\tiny Grisetti, G., Stachniss, C., and Burgard, W. ``\textit{Improving grid-based slam with rao-blackwellized particle filters by adaptive proposals and selective resampling.}'' ICRA. 2005.}		
%\end{frame}
%========================================== ############# ======================================================        
%========================================== ############# ======================================================
%========================================== ############# ======================================================
%\begin{frame}[noframenumbering]{Mapping module}
%	\begin{columns}[T]
%		\column{.5\textwidth}
%		\vspace{1.2cm}		
%		\begin{itemize}
%			\item A \textbf{2D grid} map is build ({\small Simulation and real worlds})
%			\item \textbf{Voronoi} diagram 
%		\end{itemize}
%		\column{.5\textwidth}	
%			\centering	
%			\begin{figure}
%			    \begin{subfigure}[b]{.6\columnwidth}
%			    		\centering			
%					 \includegraphics[width=.6\linewidth]{figs/2d_map_voronoi_simulation.png}
%     				 \caption{\tiny Mapping in simulation with HIMM.\footnotemark[14]}			 
%			    	\end{subfigure}
%%			    	\vspace{-.3cm}\\
%			    \begin{subfigure}[b]{.6\columnwidth}
%			    		\centering					    	
%			    		\includegraphics[width=.6\linewidth]{figs/2d_map_voronoi_real.png}	         
%			    		\caption{\tiny Mapping the real world with Gmapping.\footnotemark[15]}					 
%			    	\end{subfigure}				
%%			    	\vspace{-.7cm}	 
%				 \caption{\tiny The 2D grid maps used in this work.}			 
%	         \end{figure}		         		
%	\end{columns}		
%	\footnotetext[14]{\tiny Borenstein, J., and Yoram K. ``\textit{Histogramic in-motion mapping for mobile robot obstacle avoidance}.'' Transactions on robotics and automation. 1991.}
%	\footnotetext[15]{\tiny Grisetti, G., Stachniss, C., and Burgard, W. ``\textit{Improving grid-based slam with rao-blackwellized particle filters by adaptive proposals and selective resampling.}'' ICRA. 2005.}		
%\end{frame}		
%========================================== ############# ======================================================        
%========================================== ############# ======================================================
%========================================== ############# ======================================================
%\begin{frame}[noframenumbering]{Mapping module}
%	\begin{columns}[T]
%		\column{.5\textwidth}
%		\vspace{1.2cm}		
%		\begin{itemize}
%			\item A \textbf{2D grid} map is build ({\small Simulation and real worlds})
%			\item \textbf{Voronoi} diagram 
%			\item \textbf{Visited} regions
%		\end{itemize}
%		\column{.5\textwidth}	
%			\centering	
%			\begin{figure}
%			    \begin{subfigure}[b]{.6\columnwidth}
%			    		\centering			
%					 \includegraphics[width=.6\linewidth]{figs/2d_map_visited_simulation.png}
%     				 \caption{\tiny Mapping in simulation with HIMM.\footnotemark[14]}			 
%			    	\end{subfigure}
%%			    	\vspace{-.3cm}\\
%			    \begin{subfigure}[b]{.6\columnwidth}
%			    		\centering					    	
%			    		\includegraphics[width=.6\linewidth]{figs/2d_map_visited_real.png}	         
%			    		\caption{\tiny Mapping the real world with Gmapping.\footnotemark[15]}					 
%			    	\end{subfigure}				
%%			    	\vspace{-.7cm}	 
%				 \caption{\tiny The 2D grid maps used in this work.}			 
%	         \end{figure}		         		
%	\end{columns}		
%	\footnotetext[14]{\tiny Borenstein, J., and Yoram K. ``\textit{Histogramic in-motion mapping for mobile robot obstacle avoidance}.'' Transactions on robotics and automation. 1991.}
%	\footnotetext[15]{\tiny Grisetti, G., Stachniss, C., and Burgard, W. ``\textit{Improving grid-based slam with rao-blackwellized particle filters by adaptive proposals and selective resampling.}'' ICRA. 2005.}		
%\end{frame}			
%========================================== ############# ======================================================        
%========================================== ############# ======================================================
%========================================== ############# ======================================================
%\begin{frame}[noframenumbering]{Mapping module}
%	\begin{columns}[T]
%		\column{.5\textwidth}
%		\vspace{1.2cm}		
%		\begin{itemize}
%			\item A \textbf{2D grid} map is build ({\small Simulation and real worlds})
%			\item \textbf{Voronoi} diagram 
%			\item \textbf{Visited} regions
%			\item \textbf{Frontiers} (visited and unvisited regions)
%		\end{itemize}
%		\column{.5\textwidth}	
%			\centering	
%			\begin{figure}
%			    \begin{subfigure}[b]{.6\columnwidth}
%			    		\centering			
%					 \includegraphics[width=.6\linewidth]{figs/2d_map_visited_simulation.png}
%     				 \caption{\tiny Mapping in simulation with HIMM.\footnotemark[14]}			 
%			    	\end{subfigure}
%%			    	\vspace{-.3cm}\\
%			    \begin{subfigure}[b]{.6\columnwidth}
%			    		\centering					    	
%			    		\includegraphics[width=.6\linewidth]{figs/2d_map_visited_real.png}	         
%			    		\caption{\tiny Mapping the real world with Gmapping.\footnotemark[15]}					 
%			    	\end{subfigure}				
%%			    	\vspace{-.7cm}	 
%				 \caption{\tiny The 2D grid maps used in this work.}			 
%	         \end{figure}		         		
%	\end{columns}		
%	\footnotetext[14]{\tiny Borenstein, J., and Yoram K. ``\textit{Histogramic in-motion mapping for mobile robot obstacle avoidance}.'' Transactions on robotics and automation. 1991.}
%	\footnotetext[15]{\tiny Grisetti, G., Stachniss, C., and Burgard, W. ``\textit{Improving grid-based slam with rao-blackwellized particle filters by adaptive proposals and selective resampling.}'' ICRA. 2005.}		
%\end{frame}	
%========================================== ############# ======================================================
%========================================== ############# ======================================================
%========================================== ############# ======================================================
%\begin{frame}{Map Segmentation module}
%	\begin{columns}[T]
%		\column{.5\textwidth}
%		\begin{itemize}
%			\item \textbf{Divides} the map based on the \textbf{free space}	
%		\end{itemize}
%		\column{.5\textwidth}	
%			\centering	
%			\begin{figure}
%			    \begin{subfigure}{.4\columnwidth}		
%					 \includegraphics[width=1\linewidth]{figs/33.png}
%     				 \caption{\tiny The 2D grid map segmented.}			 		 
%			    	\end{subfigure}
%			    \begin{subfigure}{.4\columnwidth}
%			    		\includegraphics[width=1\linewidth]{figs/44.png}	         
%			    		\caption{\tiny The different segments identified by different colors.}					 
%			    	\end{subfigure}					 
%				 \caption{\tiny The 2D grid maps used in this work.}				 
%	         \end{figure}		         		
%	\end{columns}		
%\end{frame}
%========================================== ############# ======================================================
%========================================== ############# ======================================================
%========================================== ############# ======================================================
%\begin{frame}[noframenumbering]{Map Segmentation module}
%	\begin{columns}[T]
%		\column{.5\textwidth}
%		\begin{itemize}
%			\item \textbf{Divides} the map based on the \textbf{free space}	
%			\item \textbf{Simultaneously} to the \textbf{Mapping module}
%		\end{itemize}
%		\column{.5\textwidth}	
%			\centering	
%			\begin{figure}
%			    \begin{subfigure}{.4\columnwidth}		
%					 \includegraphics[width=1\linewidth]{figs/33.png}
%     				 \caption{\tiny The 2D grid map segmented.}			 		 
%			    	\end{subfigure}
%			    \begin{subfigure}{.4\columnwidth}
%			    		\includegraphics[width=1\linewidth]{figs/44.png}	         
%			    		\caption{\tiny The different segments identified by different colors.}					 
%			    	\end{subfigure}					 
%				 \caption{\tiny The 2D grid maps used in this work.}				 
%	         \end{figure}			         		
%	\end{columns}		
%\end{frame}
%========================================== ############# ======================================================
%========================================== ############# ======================================================
%========================================== ############# ======================================================
%\begin{frame}[noframenumbering]{Map Segmentation module}
%	\begin{columns}[T]
%		\column{.5\textwidth}
%		\begin{itemize}
%			\item \textbf{Divides} the map based on the \textbf{free space}	
%			\item \textbf{Simultaneously} to the \textbf{Mapping module}
%			\item Kernel Density Estimation (\textbf{KDE})\footnotemark[16]
%		\end{itemize}
%		\column{.5\textwidth}	
%			\centering	
%			\begin{figure}
%			    \begin{subfigure}{.4\columnwidth}		
%					 \includegraphics[width=1\linewidth]{figs/33.png}
%     				 \caption{\tiny The 2D grid map segmented.}			 		 
%			    	\end{subfigure}
%			    \begin{subfigure}{.4\columnwidth}
%			    		\includegraphics[width=1\linewidth]{figs/44.png}	         
%			    		\caption{\tiny The different segments identified by different colors.}					 
%			    	\end{subfigure}					 
%				 \caption{\tiny The 2D grid maps used in this work.}				 
%	         \end{figure}		         		
%	\end{columns}		
%	\footnotetext[16]{\tiny Maffei, Renan, et al. ``\textit{Fast monte carlo localization using spatial density information.}'', ICRA. 2015}
%\end{frame}
%========================================== ############# ======================================================
%========================================== ############# ======================================================
%========================================== ############# ======================================================
%\begin{frame}[noframenumbering]{Map Segmentation module}
%	\begin{columns}[T]
%		\column{.5\textwidth}
%		\begin{itemize}
%			\item \textbf{Divides} the map based on the \textbf{free space}	
%			\item \textbf{Simultaneously} to the \textbf{Mapping module}
%			\item Kernel Density Estimation (\textbf{KDE})\footnotemark[16]
%			\item An \textbf{abstraction} of the \textbf{2D grid map} for \textbf{grouping} the \textbf{detected door signs}
%		\end{itemize}
%		\column{.5\textwidth}	
%			\centering	
%			\begin{figure}
%			    \begin{subfigure}{.4\columnwidth}		
%					 \includegraphics[width=1\linewidth]{figs/33.png}
%     				 \caption{\tiny The 2D grid map segmented.}			 		 
%			    	\end{subfigure}
%			    \begin{subfigure}{.4\columnwidth}
%			    		\includegraphics[width=1\linewidth]{figs/44.png}	         
%			    		\caption{\tiny The different segments identified by different colors.}					 
%			    	\end{subfigure}					 
%				 \caption{\tiny The 2D grid maps used in this work.}				 
%	         \end{figure}	         		
%	\end{columns}		
%	\footnotetext[16]{\tiny Maffei, Renan, et al. ``\textit{Fast monte carlo localization using spatial density information.}'', ICRA. 2015}	
%\end{frame}
%========================================== ############# ======================================================
%========================================== ############# ======================================================
%========================================== ############# ======================================================
%\begin{frame}{Image Processing module}
%	\begin{columns}[T]
%		\column{0.5\textwidth}
%		\begin{itemize}
%			\item Optical character recognition (\textbf{OCR})
%		\end{itemize}
%		\column{0.5\textwidth}
%			\centering
%			\begin{figure}
%				 \includegraphics[width=.9\linewidth]{figs/image_processing.png}
%				 \caption{\tiny Illustration of the Image Processing module detecting a door label.}			 		 
%	        \end{figure}			
%	\end{columns}	
%\end{frame}
%========================================== ############# ======================================================
%========================================== ############# ======================================================
%========================================== ############# ======================================================
%\begin{frame}[noframenumbering]{Image Processing module}
%	\begin{columns}[T]
%		\column{0.5\textwidth}
%		\begin{itemize}
%			\item Optical character recognition (\textbf{OCR})
%			\item \textbf{Recognize} the number of \textbf{door signs} in \textbf{RGB images}
%		\end{itemize}
%		\column{0.5\textwidth}
%			\centering
%			\begin{figure}
%				 \includegraphics[width=.9\linewidth]{figs/image_processing.png}
%				 \caption{\tiny Illustration of the Image Processing module detecting a door label.}		
%	        \end{figure}			
%	\end{columns}	
%\end{frame}
%========================================== ############# ======================================================
%========================================== ############# ======================================================
%========================================== ############# ======================================================
%\begin{frame}[noframenumbering]{Image Processing module}
%	\begin{columns}[T]
%		\column{0.5\textwidth}
%		\begin{itemize}
%			\item Optical character recognition (\textbf{OCR})
%			\item \textbf{Recognize} the number of \textbf{door signs} in \textbf{RGB images}
%			\item We use the work proposed by Neumann and Matas (2012)\footnotemark[9]
%		\end{itemize}
%		\column{0.5\textwidth}
%			\centering
%			\begin{figure}
%				 \includegraphics[width=.9\linewidth]{figs/image_processing.png}
%				 \caption{\tiny Illustration of the Image Processing module detecting a door label.}				 		 
%	        \end{figure}			
%	\end{columns}	
%	\footnotetext[9]{\tiny Neumann, Lukáš, and Jiří Matas. ``\textit{Real-time scene text localization and recognition}.'',Conference on Computer Vision and Pattern Recognition. 2012.}	
%\end{frame}
%========================================== ############# ======================================================
%========================================== ############# ======================================================
%========================================== ############# ======================================================
%\begin{frame}[noframenumbering]{Image Processing module}
%	\begin{columns}[T]
%		\column{0.5\textwidth}
%		\begin{itemize}
%			\item Optical character recognition (\textbf{OCR})
%			\item \textbf{Recognize} the number of \textbf{door signs} in \textbf{RGB images}
%			\item We use the work proposed by Neumann and Matas (2012)\footnotemark[9]
%			\item The \textbf{recognized numbers} are \textbf{associated} to the \textbf{map segments}
%		\end{itemize}
%		\column{0.5\textwidth}
%			\centering
%			\begin{figure}
%				 \includegraphics[width=.9\linewidth]{figs/image_processing.png}
%				 \caption{\tiny Illustration of the Image Processing module detecting a door label.}		
%	        \end{figure}			
%	\end{columns}	
%	\footnotetext[9]{\tiny Neumann, Lukáš, and Jiří Matas. ``\textit{Real-time scene text localization and recognition}.'',Conference on Computer Vision and Pattern Recognition. 2012.}		
%\end{frame}
%========================================== ############# ======================================================
%========================================== ############# ======================================================
%========================================== ############# ======================================================
%\begin{frame}{Semantic Planner module}
%	\vspace{-1.4cm}	
%	\begin{itemize}[<+->]
%		\item It \textbf{estimates} which \textbf{regions} are \textbf{more promising} to find the goal-door
%		\item It is composed of \textbf{five} different \textbf{factors}:
%	\end{itemize}
%\end{frame}
%========================================== ############# ======================================================
%========================================== ############# ======================================================
%========================================== ############# ======================================================
%\begin{frame}[noframenumbering]{Semantic Planner module}
%	\begin{itemize}
%		\item It \textbf{estimates} which \textbf{regions} are \textbf{more promising} to find the goal-door
%		\item It is composed of \textbf{five} different \textbf{factors}:
%		\begin{itemize}
%			\item Parity		
%			\item Growing Direction
%			\item Doors Orientation
%			\item Robot Orientation
%			\item Distance			
%		\end{itemize}
%	\end{itemize}
%\end{frame}
%========================================== ############# ======================================================
%========================================== ############# ======================================================
%========================================== ############# ======================================================
\begin{frame}{Semantic Planner module}
		\vspace{.8cm}		
%		\hspace{.15cm}			
	\begin{columns}
		\column{0.755\textwidth}		
		\begin{itemize}
			\item It \textbf{estimates} which \textbf{regions} are \textbf{more promising} to the SR \textbf{find} the \textbf{goal-door}
			\item It is composed of five different factors:	
			\begin{itemize}
				\item Parity
				\item Growing Direction
				\item Doors Orientation
				\item Robot Orientation
				\item Closeness			
			\end{itemize}
		\end{itemize}
		\vspace{1.85cm}		
		\column{0.245\textwidth}		
			\hspace{-7.6cm}					
			$\Big\}$\textit{\textbf{Semantic} factors}\\
			\vspace{0.15cm}
			\hspace{-7.6cm}						
			$\Bigg\}$\textit{\textbf{Geometric} factors}		
	\end{columns}	
\end{frame}
%========================================== ############# ======================================================
%========================================== ############# ======================================================
%========================================== ############# ======================================================
\begin{frame}{Parity}
	\begin{columns}[T]
		\column{.5\textwidth}
		\begin{itemize}
			\item \textbf{Goal}: estimate whether the segment is promising regarding its door signs' parity		
			\item \textbf{Analysis} based on the \textbf{door sign numbers} 
			\item \textbf{Parity} of the \textbf{segment}
			\item Comparison with the \textbf{goal-door's parity}
		\end{itemize}
		\column{.65\textwidth}	
			\centering	
			\begin{figure}
			    \begin{subfigure}{.6\columnwidth}
			    		\centering
					 \includegraphics[width=.85\linewidth]{figs/parity_even.png}
     				 \caption{\tiny Door signs and goal-door with same parity. $\varphi_p({c}) = 1$}			 		 
			    	\end{subfigure}\\
			    \begin{subfigure}{.6\columnwidth}
			    		\centering				    
			    		\includegraphics[width=.85\linewidth]{figs/parity_odd.png}	         
			    		\caption{\tiny Door signs and goal-door with different parity. $\varphi_p({c}) = 0$}					 
			    	\end{subfigure}					 
				 \caption{\tiny Example to illustrate the parity factor.}			 
	         \end{figure}	         		
	\end{columns}
\end{frame}
%========================================== ############# ======================================================
%========================================== ############# ======================================================
%========================================== ############# ======================================================
%\begin{frame}{Parity} %
%	\begin{figure}
%%		\includegraphics[width=.5\linewidth]{figs/parity_equation.png}	         
%			    \begin{subfigure}{.4\columnwidth}
%					 \includegraphics[width=1\linewidth]{figs/parity_even.png}
%     				 \caption{\tiny Door signs and goal-door with same parity. $\varphi_p({c}) = 1$}			 		 
%			    	\end{subfigure}
%			    \begin{subfigure}{.4\columnwidth}
%			    		\includegraphics[width=1\linewidth]{figs/parity_odd.png}	         
%			    		\caption{\tiny Door signs and goal-door with different parity. $\varphi_p({c}) = 0$}	 
%			    	\end{subfigure}					 
%				 \caption{\tiny Example to illustrate the parity factor.}			 
%	\end{figure}					
%\end{frame}
%========================================== ############# ======================================================
%========================================== ############# ======================================================
%========================================== ############# ======================================================
\begin{frame}{Growing Direction}
	\begin{columns}[T]
		\column{0.5\textwidth}
			\centering
			\begin{itemize}
				\item \textbf{Goal}: \textbf{estimate} whether the \textbf{segment is promising} regarding the \textbf{sequence of door signs}
				\item Find out the direction that the numbers of the segment is growing
				\item \textbf{Angle} of the \textbf{growing direction} and the \textbf{numbers of the segment}
%				\item The position of the detected door signs is considered
%				\item All detected door signs of the same segment
			\end{itemize}
		\column{0.5\textwidth}
			\begin{figure}
				\includegraphics[width=.7\linewidth]{figs/growing_unknown.png}	
				 \caption{\tiny Example of a partial 2D map and the detected door signs without numbers.}			 		 				         
			\end{figure}					
	\end{columns}	
\end{frame}
%========================================== ############# ======================================================
%========================================== ############# ======================================================
%========================================== ############# ======================================================
\begin{frame}{Angle of the Growing Direction}
	\begin{columns}[T]
		\centering
		\column{1\textwidth}
		\begin{itemize}
			\item Compute the \textbf{angle} in which the door sign \textbf{sequence is increasing}
			\item For all possible \textbf{pairs of two different door signs} (one is larger)
			\item Comparison with the angle of the Voronoi cell in the frontier
		\end{itemize}
	\end{columns}	
	\begin{columns}
		\centering
		\column{1\textwidth}
			\begin{figure}
			
				\includegraphics[width=.23\linewidth]{figs/growing_1.png}
				\includegraphics[width=.23\linewidth]{figs/growing_2.png}	         
				\includegraphics[width=.23\linewidth]{figs/growing_4.png}				
				\includegraphics[width=.23\linewidth]{figs/growing_3.png}	         				
				\caption{\tiny Step by step of how the angle of the growing direction is computed.}
	         \end{figure}			
%		\column{0.5\textwidth}
%			\begin{figure}
%				
%			\end{figure}					
	\end{columns}
\end{frame}
%========================================== ############# ======================================================
%========================================== ############# ======================================================
%========================================== ############# ======================================================
\begin{frame}{Numbers of the segment}
	\begin{columns}[T]
		\centering
		\column{.3\textwidth}
		\begin{itemize}
			\item The door sign \textbf{numbers} of the \textbf{segment matters}
			\item \textbf{Comparison} with the \textbf{goal-door} (\textbf{smaller} and \textbf{larger} numbers are counted)
		\end{itemize}	
		\column{0.8\textwidth}
			\centering	
			\begin{figure}
			    \begin{subfigure}{.45\columnwidth}
			    		\centering
					 \includegraphics[width=.83\linewidth]{figs/growing_direction_same_orient_higher.png}
     				 \caption{\tiny $\gamma(\theta_f({c})) = 1$ and $\zeta\big(S({c})\big) = 1$, $\varphi_g({c}) = 1$.}			 		 
			    	\end{subfigure}
			    \begin{subfigure}{.45\columnwidth}
			    		\centering			    
			    		\includegraphics[width=.83\linewidth]{figs/growing_direction_same_orient_smaller.png}
			    		\caption{\tiny $\gamma(\theta_f({c})) = 1$ and $\zeta\big(S({c})\big) = -1$, $\varphi_g({c}) = 0$.}					 
			    	\end{subfigure}	\\	
			    \begin{subfigure}{.45\columnwidth}
			    		\centering			    
					 \includegraphics[width=.83\linewidth]{figs/growing_direction_different_orient_smaller.png}
     				 \caption{\tiny $\gamma(\theta_f({c})) = -1$ and $\zeta\big(S({c})\big) = -1$, $\varphi_g({c}) = 1$.}			 		 
			    	\end{subfigure}
			    \begin{subfigure}{.45\columnwidth}
			    		\centering			    
			    		\includegraphics[width=.83\linewidth]{figs/growing_direction_different_orient_higher.png}			    	
			    		\caption{\tiny $\gamma(\theta_f({c})) = -1$ and $\zeta\big(S({c})\big) = 1$, $\varphi_g({c}) = 0$.}					 
			    	\end{subfigure}				    				 
				 \caption{\tiny Maps used in the simulated experiments.}			 
	         \end{figure}				
	\end{columns}
\end{frame}
%---------------------------------------------------------------------------------------
%\begin{frame}{Numbers of the segment}
%	\begin{columns}[T]
%		\centering
%		\column{1\textwidth}
%		\begin{itemize}
%			\item The door sign \textbf{numbers} of the \textbf{segment matters}
%			\item \textbf{Comparison} with the \textbf{goal-door} (\textbf{smaller} and \textbf{larger} numbers are counted)
%		\end{itemize}
%	\end{columns}	
%	\begin{columns}
%		\centering
%		\column{1\textwidth}
%			\begin{figure}
%				\includegraphics[width=.3\linewidth]{figs/growing_goaldoor_2.png}
%				\includegraphics[width=.3\linewidth]{figs/growing_goaldoor_32.png}	         
%				\includegraphics[width=.3\linewidth]{figs/growing_goaldoor_15.png}				
%	         \end{figure}			
%%		\column{0.5\textwidth}
%%			\begin{figure}
%%				
%%			\end{figure}					
%	\end{columns}
%\end{frame}
%========================================== ############# ======================================================
%========================================== ############# ======================================================
%========================================== ############# ======================================================
%\begin{frame}{Growing Direction}
%	\begin{columns}					
%		\column{0.8\textwidth}
%			\centering	
%			\begin{figure}
%			    \begin{subfigure}{.45\columnwidth}
%			    		\centering
%					 \includegraphics[width=.9\linewidth]{figs/growing_direction_same_orient_higher.png}
%     				 \caption{\tiny $\gamma(\theta_f({c})) = 1$ and $\zeta\big(S({c})\big) = 1$, $\varphi_g({c}) = 1$.}			 		 
%			    	\end{subfigure}
%			    \begin{subfigure}{.45\columnwidth}
%			    		\centering			    
%			    		\includegraphics[width=.9\linewidth]{figs/growing_direction_same_orient_smaller.png}
%			    		\caption{\tiny $\gamma(\theta_f({c})) = 1$ and $\zeta\big(S({c})\big) = -1$, $\varphi_g({c}) = 0$.}					 
%			    	\end{subfigure}	\\	
%			    \begin{subfigure}{.45\columnwidth}
%			    		\centering			    
%					 \includegraphics[width=.9\linewidth]{figs/growing_direction_different_orient_smaller.png}
%     				 \caption{\tiny $\gamma(\theta_f({c})) = -1$ and $\zeta\big(S({c})\big) = -1$, $\varphi_g({c}) = 1$.}			 		 
%			    	\end{subfigure}
%			    \begin{subfigure}{.45\columnwidth}
%			    		\centering			    
%			    		\includegraphics[width=.9\linewidth]{figs/growing_direction_different_orient_higher.png}			    	
%			    		\caption{\tiny $\gamma(\theta_f({c})) = -1$ and $\zeta\big(S({c})\big) = 1$, $\varphi_g({c}) = 0$.}					 
%			    	\end{subfigure}				    				 
%				 \caption{\tiny Maps used in the simulated experiments.}			 
%	         \end{figure}	
%	\end{columns}					
%\end{frame}
%========================================== ############# ======================================================
%========================================== ############# ======================================================
%========================================== ############# ======================================================
\begin{frame}{Door and Robot Orientations}
	\begin{columns}
		\column{0.55\textwidth}
		\begin{itemize}[<+->]
			\item \textbf{Goal:} \textbf{estimate} how the \textbf{door signs are organised} in the environment
			\item \textbf{History} of a certain amount of the \textbf{most recent robot’s orientations} when a \textbf{door sign has been recognized} (Door orientation)			
			\item \textbf{History} of a certain amount of the \textbf{most recent robot’s orientations} (Robot orientation)
		\end{itemize}
		\column{0.55\textwidth}
			\centering	
			\begin{figure}
%			    \begin{subfigure}{.45\columnwidth} , angle=90 
					 \includegraphics[width=.75\linewidth]{figs/all_orientation_robots.png}
%     				 \caption{\tiny Corridor and door signs.}			 		 
%			    	\end{subfigure}
%			    	\\
%			    \begin{subfigure}{.45\columnwidth}
%			    		\includegraphics[width=1\linewidth]{figs/door_orientation.png}
%			    		\caption{\tiny Mobile robot in the lift.}					 
%			    	\end{subfigure}					 
				 \caption{\tiny Initial works on SLAM}			 
	         \end{figure}		
	\end{columns}
\end{frame}
%========================================== ############# ======================================================
%========================================== ############# ======================================================
%========================================== ############# ======================================================
\begin{frame}[noframenumbering]{Door and Robot Orientations}
	\begin{columns}
		\column{0.55\textwidth}
		\begin{itemize}
			\item \textbf{Histograms} are computed (each bin corresponds to an angle):
			\begin{itemize}
				\item $\varphi_o({c}) = H_d[\theta_f({c})]$
				\item $\varphi_r({c}) = H_r[\theta_f({c})]$
			\end{itemize}
			\item \textbf{Door orientation prioritize orientations} in which the \textbf{robot} has \textbf{recognized} most of the \textbf{door signs}
			\item \textbf{Robot orientation} makes the \textbf{robot} \textbf{consider} \textbf{other paths} that may connect to other corridors
		\end{itemize}
		\column{0.55\textwidth}
			\centering	
			\begin{figure}
%			    \begin{subfigure}{.45\columnwidth} , angle=90
					 \includegraphics[width=.75\linewidth]{figs/all_orientation_robots.png}
%     				 \caption{\tiny Corridor and door signs.}			 		 
%			    	\end{subfigure}
%			    	\\
%			    \begin{subfigure}{.45\columnwidth}
%			    		\includegraphics[width=1\linewidth]{figs/door_orientation.png}
%			    		\caption{\tiny Mobile robot in the lift.}					 
%			    	\end{subfigure}					 
				 \caption{\tiny Initial works on SLAM}			 
	         \end{figure}		
	\end{columns}
\end{frame}
%========================================== ############# ======================================================
%========================================== ############# ======================================================
%========================================== ############# ======================================================
\begin{frame}{Closeness}
	\begin{columns}[T]
		\column{1\textwidth}
		\begin{itemize}[<+->]
			\item \textbf{Voronoi} diagram
			\item \textbf{Smallest number} of \textbf{Voronoi cells} between the \textbf{robot's position} and each \textbf{frontier}
			\item It guides the robot towards the \textbf{closest frontier}, $\varphi_c({c}) = 1$
			\item It \textbf{saves} the \textbf{robot's resources}
		\end{itemize}
%		\column{0.55\textwidth}
%			\centering	
%			\begin{figure}
%				\includegraphics[width=.4\linewidth]{figs/distance_1.png}\\
%				\includegraphics[width=.4\linewidth]{figs/distance_2.png}\\
%				\includegraphics[width=.4\linewidth]{figs/distance_3.png}			    	
%				 \caption{\tiny Initial works on SLAM}			 
%	         \end{figure}
	\end{columns}
\end{frame}
%========================================== ############# ======================================================
%========================================== ############# ======================================================
%========================================== ############# ======================================================
%\begin{frame}{Distance}
%	\begin{figure}
%		\includegraphics[width=.5\linewidth]{figs/distance_equation.png}	         
%	\end{figure}					
%\end{frame}
%========================================== ############# ======================================================
%========================================== ############# ======================================================
%========================================== ############# ======================================================
\begin{frame}{Estimating where to search}
	\begin{columns}[T]
		\column{0.85\textwidth}
		\begin{itemize}
			\item Combining all the factors:
			\begin{itemize}
				\item \textbf{Semantic}:\\$\mathtt{SF}(\mathbf{m}_i) = \varphi_g(\mathbf{m}_i) \varphi_p(\mathbf{m}_i)$			
			\end{itemize}
			\begin{itemize}
				\item \textbf{Geometric}:\\$\mathtt{GF}(\mathbf{m}_i) = \frac{\big(\varphi_o(\mathbf{m}_i) + \varphi_r(\mathbf{m}_i)\big)\varphi_c(\mathbf{m}_i) + \varphi_c(\mathbf{m}_i)}{3.0}$
			\end{itemize}	
%I SHOULD UNCOMMENT THIS ########################################################################################				
			\item Find the \textbf{most promising frontier}: $\varphi(\mathbf{m}_i) = \mathtt{SF}(\mathbf{m}_i)\alpha + \mathtt{GF}(\mathbf{m}_i)(1.0 - \alpha)$	\\ ~~~~~~~~~~~~~~~~~~~~~~~~~~~~~~~~~~~~~~~~~~~~~~~~~${\mathbf{m}_i}^* = \underset{\mathbf{m}_i \in \mathbf{C}}{\textup{\textup{arg max}}}(\varphi(\mathbf{m}_i))$			
%I SHOULD UNCOMMENT THIS ########################################################################################
		\end{itemize}
		\column{0.15\textwidth}
	\end{columns}
\end{frame}
%========================================== ############# ======================================================
%========================================== ############# ======================================================
%========================================== ############# ======================================================
%\begin{frame}{Final formula}
%	\begin{figure}
%		\includegraphics[width=.5\linewidth]{figs/semantic_factor_equation.png}\\
%		\includegraphics[width=.5\linewidth]{figs/geometric_factor_equation.png}\\
%		\includegraphics[width=.5\linewidth]{figs/final_formula_equation.png}\\
%		\includegraphics[width=.5\linewidth]{figs/max_final_equation.png}
%	\end{figure}					
%\end{frame}
%%========================================== ############# ======================================================
%========================================== ############# ======================================================
%========================================== ############# ======================================================
\begin{frame}{Experiments and Results}
%	\begin{columns}[T]
%		\column{.4\textwidth}
%		\vspace{-1.5cm}
		\begin{itemize}
			\item Simulated experiment setup
		\end{itemize}
%		\column{.6\textwidth}
%		\centering		
		\hspace{-7cm}
		\begin{figure}
			\centering
			\includegraphics[width=.6\linewidth]{figs/setup_simulation_text.png}	   
			\caption{\tiny System overview of our setup for the experiments in simulation.}											
		\end{figure}
%	\end{columns}	
\end{frame}
%========================================== ############# ======================================================
%========================================== ############# ======================================================
%========================================== ############# ======================================================
\begin{frame}[noframenumbering]{Experiments and Results}
	\begin{columns}[T]
		\column{0.4\textwidth}
		\begin{itemize}
			\item Simulated experiment setup
			\item Maps:
			\begin{itemize}
				\item Normal
				\item Inverse
				\item Hotel
				\item KTH
			\end{itemize}
		\end{itemize}
		\column{0.6\textwidth}
			\centering	
			\begin{figure}
			    \begin{subfigure}{.45\columnwidth}
			    		\centering
					 \includegraphics[width=.7\linewidth]{figs/map_normal_redcircles.png}			    		
     				 \caption{\tiny Normal.}			 		 
			    	\end{subfigure}
			    \begin{subfigure}{.45\columnwidth}
			    		\centering		
			    		\includegraphics[width=.7\linewidth]{figs/map_inverse_redcircles.png}						    			    
			    		\caption{\tiny Inverse.}					 
			    	\end{subfigure}	\\	
			    \begin{subfigure}{.45\columnwidth}
			    		\centering			    
			    		\includegraphics[width=.8\linewidth]{figs/hotel2_PAPER.png}
     				 \caption{\tiny Hotel.}			 		 
			    	\end{subfigure}
			    \begin{subfigure}{.45\columnwidth}
			    		\centering			    
					 \includegraphics[width=.52\linewidth]{figs/KTH_CampusValhallava_finished.png}			    		    	
			    		\caption{\tiny KTH.}					 
			    	\end{subfigure}				    				 
				 \caption{\tiny Maps used in the simulated experiments.}					 
	         \end{figure}				
	\end{columns}
\end{frame}
%========================================== ############# ======================================================
%========================================== ############# ======================================================
%========================================== ############# ======================================================
\begin{frame}[noframenumbering]{Experiments and Results}
	\begin{columns}[T]
		\column{0.4\textwidth}
		\begin{itemize}
			\item Simulated experiment setup
			\item Maps:
			\begin{itemize}
				\item Normal
				\item Inverse
				\item Hotel
				\item KTH
			\end{itemize}
			\item Greedy search
		\end{itemize}
		\column{0.6\textwidth}
			\centering	
			\begin{figure}
			    \begin{subfigure}{.45\columnwidth}
			    		\centering
					 \includegraphics[width=.6\linewidth]{figs/map_normal_redcircles.png}			    		
     				 \caption{\tiny Normal.}			 		 
			    	\end{subfigure}
			    \begin{subfigure}{.45\columnwidth}
			    		\centering		
			    		\includegraphics[width=.6\linewidth]{figs/map_inverse_redcircles.png}						    			    
			    		\caption{\tiny Inverse.}					 
			    	\end{subfigure}	\\	
			    \begin{subfigure}{.45\columnwidth}
			    		\centering			    
			    		\includegraphics[width=.7\linewidth]{figs/hotel2_PAPER.png}
     				 \caption{\tiny Hotel.}			 		 
			    	\end{subfigure}
			    \begin{subfigure}{.45\columnwidth}
			    		\centering			    
					 \includegraphics[width=.45\linewidth]{figs/KTH_CampusValhallava_finished.png}			    		    	
			    		\caption{\tiny KTH.}					 
			    	\end{subfigure}				    				 
				 \caption{\tiny Maps used in the simulated experiments.}					 
	         \end{figure}					
	\end{columns}
\end{frame}
%========================================== ############# ======================================================
%========================================== ############# ======================================================
%========================================== ############# ======================================================
\begin{frame}[noframenumbering]{Experiments and Results}
	\begin{columns}[T]
		\column{0.4\textwidth}
		\begin{itemize}
			\item Simulated experiment setup
			\item Maps:
			\begin{itemize}
				\item Normal
				\item Inverse
				\item Hotel
				\item KTH
			\end{itemize}
			\item Greedy search
			\item The \textbf{shorter} the \textbf{traveled distance}, the \textbf{better}
		\end{itemize}
		\column{0.6\textwidth}
			\centering	
			\begin{figure}
			    \begin{subfigure}{.45\columnwidth}
			    		\centering
					 \includegraphics[width=.6\linewidth]{figs/map_normal_redcircles.png}			    		
     				 \caption{\tiny Normal.}			 		 
			    	\end{subfigure}
			    \begin{subfigure}{.45\columnwidth}
			    		\centering		
			    		\includegraphics[width=.6\linewidth]{figs/map_inverse_redcircles.png}						    			    
			    		\caption{\tiny Inverse.}					 
			    	\end{subfigure}	\\	
			    \begin{subfigure}{.45\columnwidth}
			    		\centering			    
			    		\includegraphics[width=.7\linewidth]{figs/hotel2_PAPER.png}
     				 \caption{\tiny Hotel.}			 		 
			    	\end{subfigure}
			    \begin{subfigure}{.45\columnwidth}
			    		\centering			    
					 \includegraphics[width=.45\linewidth]{figs/KTH_CampusValhallava_finished.png}			    		    	
			    		\caption{\tiny KTH.}					 
			    	\end{subfigure}				    				 
				 \caption{\tiny Maps used in the simulated experiments.}					 
	         \end{figure}					
	\end{columns}
\end{frame}
%========================================== ############# ======================================================
%========================================== ############# ======================================================
%========================================== ############# ======================================================
\begin{frame}{Experiments and Results}
			\centering	
			\begin{figure}
			    \begin{subfigure}{.45\columnwidth}
			    		\centering
					 \includegraphics[width=.85\linewidth]{figs/normal_avrg_text.png}
     				 \caption{\tiny Average of the traveled distance in the Normal map.}			 		 
			    	\end{subfigure}
			    \begin{subfigure}{.45\columnwidth}
			    		\centering			    
			    		\includegraphics[width=.85\linewidth]{figs/inverse_avrg_text.png}
     				 \caption{\tiny Average of the traveled distance in the Inverse map.}	
			    	\end{subfigure}	\\	
			    \begin{subfigure}{.45\columnwidth}
			    		\centering			    
					 \includegraphics[width=.85\linewidth]{figs/hotel_avrg_text.png}
     				 \caption{\tiny Average of the traveled distance in the Hotel map.}		 
			    	\end{subfigure}
			    \begin{subfigure}{.45\columnwidth}
			    		\centering			    
			    		\includegraphics[width=.85\linewidth]{figs/kth_avrg_text.png}			    	
     				 \caption{\tiny Average of the traveled distance in the KTH map.}				 
			    	\end{subfigure}	
			    \begin{subfigure}{.45\columnwidth}
			    		\centering			    
			    		\includegraphics[width=.55\linewidth]{figs/scale_colors.png}			    	
			    		\caption{\tiny Meaning of the cell colors.}					 
			    	\end{subfigure}				    				    				 
				 \caption{\tiny Maps used in the simulated experiments.}					 
	         \end{figure}	
\end{frame}

%========================================== ############# ======================================================
%========================================== ############# ======================================================
%========================================== ############# ======================================================
%\begin{frame}{Experiments and Results}
%	\begin{columns}[T]
%		\column{0.2\textwidth}
%		\begin{itemize}
%			\item Normal map
%		\end{itemize}		
%		\column{0.4\textwidth}					
%				\begin{figure}
%				    \begin{subfigure}{1\columnwidth}	
%				    		\centering								
%						\includegraphics[width=1\linewidth]{figs/normal_avrg_text.png}%RESULTS_NORMAL_SHORTEST_NEW_T-eps-converted-to}	
%				    	\end{subfigure}
%				    \begin{subfigure}{1\columnwidth}												
%				    		\centering		
%				    		\vspace{.15cm}		    
%						\includegraphics[width=.8\linewidth]{figs/map_normal_redcircles.png}					
%				    	\end{subfigure}						
%				\end{figure}			
%		\column{0.4\textwidth}
%				\begin{figure}		
%					\includegraphics[width=.8\linewidth]{figs/RESULTS_NORMAL_NEW_T-eps-converted-to}		
%				\end{figure}			    				         
%	\end{columns}
%\end{frame}
%========================================== ############# ======================================================
%========================================== ############# ======================================================
%========================================== ############# ======================================================
%\begin{frame}{Experiments and Results}
%	\begin{columns}[T]
%		\column{0.2\textwidth}
%		\begin{itemize}
%			\item Inverse map
%		\end{itemize}
%		\column{0.4\textwidth}					
%				\begin{figure}
%				    \begin{subfigure}{1\columnwidth}	
%				    		\centering								
%						\includegraphics[width=1\linewidth]{figs/inverse_avrg_text.png}%RESULTS_INVERSE_SHORTEST_NEW_T-eps-converted-to}	
%				    	\end{subfigure}
%				    \begin{subfigure}{1\columnwidth}												
%				    		\centering		
%				    		\vspace{.15cm}		    
%						\includegraphics[width=.8\linewidth]{figs/map_inverse_redcircles.png}			
%				    	\end{subfigure}						
%				\end{figure}			
%		\column{0.4\textwidth}
%				\begin{figure}		
%					\includegraphics[width=.8\linewidth]{figs/RESULTS_INVERSE_NEW_T-eps-converted-to}
%				\end{figure}	         			
%	\end{columns}
%\end{frame}
%========================================== ############# ======================================================
%========================================== ############# ======================================================
%========================================== ############# ======================================================
%\begin{frame}{Experiments and Results}
%	\begin{columns}[T]
%		\column{0.2\textwidth}
%		\begin{itemize}
%			\item Hotel map
%		\end{itemize}
%		\column{0.4\textwidth}					
%				\begin{figure}
%				    \begin{subfigure}{1\columnwidth}	
%				    		\centering								
%						\includegraphics[width=1\linewidth]{figs/hotel_avrg_text.png}%RESULTS_HOTEL_SHORTEST_NEW_T-eps-converted-to	}
%				    	\end{subfigure}
%				    \begin{subfigure}{1\columnwidth}												
%				    		\centering		
%				    		\vspace{.15cm}		    
%						\includegraphics[width=.8\linewidth]{figs/hotel2_PAPER.png}
%				    	\end{subfigure}						
%				\end{figure}			
%		\column{0.4\textwidth}
%				\begin{figure}		
%					\includegraphics[width=.8\linewidth]{figs/RESULTS_HOTEL_NEW_T-eps-converted-to}
%				\end{figure}				         			
%	\end{columns}
%\end{frame}
%========================================== ############# ======================================================
%========================================== ############# ======================================================
%========================================== ############# ======================================================
%\begin{frame}{Experiments and Results}
%	\begin{columns}[T]
%		\column{0.2\textwidth}
%		\begin{itemize}
%			\item KTH map
%		\end{itemize}
%		\column{0.4\textwidth}					
%				\begin{figure}
%				    \begin{subfigure}{1\columnwidth}	
%				    		\centering								
%						\includegraphics[width=1\linewidth]{figs/kth_avrg_text.png}%RESULTS_KTH_SHORTEST_NEW_T-eps-converted-to}	
%				    	\end{subfigure}
%				    \begin{subfigure}{1\columnwidth}												
%				    		\centering		
%				    		\vspace{.15cm}		    
%						\includegraphics[width=.53\linewidth]{figs/KTH_CampusValhallava_finished.png}
%				    	\end{subfigure}						
%				\end{figure}				
%		\column{0.4\textwidth}
%				\begin{figure}		
%					\includegraphics[width=.8\linewidth]{figs/RESULTS_KTH_NEW_T-eps-converted-to}
%				\end{figure}	         		
%	\end{columns}
%\end{frame}
%========================================== ############# ======================================================
%========================================== ############# ======================================================
%========================================== ############# ======================================================
%\begin{frame}{Experiments and Results}
%
%			\centering	
%			\begin{figure}
%			    \begin{subfigure}{.45\columnwidth}
%			    		\centering
%					 \includegraphics[width=.7\linewidth]{figs/4_111_0.8_inverse_map/1222.png}
%     				 \caption{\tiny Inverse map, goal-door 111, $\alpha = 80\%$.}
%			    	\end{subfigure}
%			    \begin{subfigure}{.45\columnwidth}
%			    		\centering			    
%			    		\includegraphics[width=.7\linewidth]{figs/7_111_0.9_inverse_map/1222.png}
%     				 \caption{\tiny Inverse map, goal-door 111, $\alpha = 90\%$.}
%			    	\end{subfigure}						    				    				 
%				 \caption{\tiny Comparison of our NSOS system searching for goal-door 111 in the Inverse map with different $\alpha$ values.}					
%	         \end{figure}	
%\end{frame}
%========================================== ############# ======================================================
%========================================== ############# ======================================================
%========================================== ############# ======================================================
%\begin{frame}{Experiments and Results}
%
%			\centering	
%			\begin{figure}
%			    \begin{subfigure}{.45\columnwidth}
%			    		\centering
%					 \includegraphics[width=.7\linewidth]{figs/4_111_0.8_inverse_map/22.png}
%     				 \caption{\tiny Inverse map, goal-door 111, $\alpha = 80\%$.}
%			    	\end{subfigure}
%			    \begin{subfigure}{.45\columnwidth}
%			    		\centering			    
%			    		\includegraphics[width=.7\linewidth]{figs/7_111_0.9_inverse_map/22.png}
%     				 \caption{\tiny Inverse map, goal-door 111, $\alpha = 90\%$.}
%			    	\end{subfigure}						    				    				 
%				 \caption{\tiny Comparison of our NSOS system searching for goal-door 111 in the Inverse map with different $\alpha$ values.}					
%	         \end{figure}	
%\end{frame}
%========================================== ############# ======================================================
%========================================== ############# ======================================================
%========================================== ############# ======================================================
%\begin{frame}{Experiments and Results}
%
%			\centering	
%			\begin{figure}
%			    \begin{subfigure}{.45\columnwidth}
%			    		\centering
%					 \includegraphics[width=.7\linewidth]{figs/4_111_0.8_inverse_map/32.png}
%     				 \caption{\tiny Inverse map, goal-door 111, $\alpha = 80\%$.}
%			    	\end{subfigure}
%			    \begin{subfigure}{.45\columnwidth}
%			    		\centering			    
%			    		\includegraphics[width=.7\linewidth]{figs/7_111_0.9_inverse_map/32.png}
%     				 \caption{\tiny Inverse map, goal-door 111, $\alpha = 90\%$.}
%			    	\end{subfigure}						    				    				 
%				 \caption{\tiny Comparison of our NSOS system searching for goal-door 111 in the Inverse map with different $\alpha$ values.}					
%	         \end{figure}	
%\end{frame}
%========================================== ############# ======================================================
%========================================== ############# ======================================================
%========================================== ############# ======================================================
%\begin{frame}{Experiments and Results}
%
%			\centering	
%			\begin{figure}
%			    \begin{subfigure}{.45\columnwidth}
%			    		\centering
%					 \includegraphics[width=.7\linewidth]{figs/4_111_0.8_inverse_map/42.png}
%     				 \caption{\tiny Inverse map, goal-door 111, $\alpha = 80\%$.}
%			    	\end{subfigure}
%			    \begin{subfigure}{.45\columnwidth}
%			    		\centering			    
%			    		\includegraphics[width=.7\linewidth]{figs/7_111_0.9_inverse_map/42.png}
%     				 \caption{\tiny Inverse map, goal-door 111, $\alpha = 90\%$.}
%			    	\end{subfigure}						    				    				 
%				 \caption{\tiny Comparison of our NSOS system searching for goal-door 111 in the Inverse map with different $\alpha$ values.}					
%	         \end{figure}	
%\end{frame}
%========================================== ############# ======================================================
%========================================== ############# ======================================================
%========================================== ############# ======================================================
%\begin{frame}{Experiments and Results}
%
%			\centering	
%			\begin{figure}
%			    \begin{subfigure}{.45\columnwidth}
%			    		\centering
%					 \includegraphics[width=.7\linewidth]{figs/4_111_0.8_inverse_map/52.png}
%     				 \caption{\tiny Inverse map, goal-door 111, $\alpha = 80\%$.}
%			    	\end{subfigure}
%			    \begin{subfigure}{.45\columnwidth}
%			    		\centering			    
%			    		\includegraphics[width=.7\linewidth]{figs/7_111_0.9_inverse_map/52.png}
%     				 \caption{\tiny Inverse map, goal-door 111, $\alpha = 90\%$.}
%			    	\end{subfigure}						    				    				 
%				 \caption{\tiny Comparison of our NSOS system searching for goal-door 111 in the Inverse map with different $\alpha$ values.}					
%	         \end{figure}	
%\end{frame}
%========================================== ############# ======================================================
%========================================== ############# ======================================================
%========================================== ############# ======================================================
%\begin{frame}{Experiments and Results}
%
%			\centering	
%			\begin{figure}
%			    \begin{subfigure}{.45\columnwidth}
%			    		\centering
%					 \includegraphics[width=.7\linewidth]{figs/4_111_0.8_inverse_map/62.png}
%     				 \caption{\tiny Inverse map, goal-door 111, $\alpha = 80\%$.}
%			    	\end{subfigure}
%			    \begin{subfigure}{.45\columnwidth}
%			    		\centering			    
%			    		\includegraphics[width=.7\linewidth]{figs/7_111_0.9_inverse_map/62.png}
%     				 \caption{\tiny Inverse map, goal-door 111, $\alpha = 90\%$.}
%			    	\end{subfigure}						    				    				 
%				 \caption{\tiny Comparison of our NSOS system searching for goal-door 111 in the Inverse map with different $\alpha$ values.}					
%	         \end{figure}	
%\end{frame}
%========================================== ############# ======================================================
%========================================== ############# ======================================================
%========================================== ############# ======================================================
%\begin{frame}{Experiments and Results}
%
%			\centering	
%			\begin{figure}
%			    \begin{subfigure}{.45\columnwidth}
%			    		\centering
%					 \includegraphics[width=.7\linewidth]{figs/4_111_0.8_inverse_map/72.png}
%     				 \caption{\tiny Inverse map, goal-door 111, $\alpha = 80\%$.}
%			    	\end{subfigure}
%			    \begin{subfigure}{.45\columnwidth}
%			    		\centering			    
%			    		\includegraphics[width=.7\linewidth]{figs/7_111_0.9_inverse_map/72.png}
%     				 \caption{\tiny Inverse map, goal-door 111, $\alpha = 90\%$.}
%			    	\end{subfigure}						    				    				 
%				 \caption{\tiny Comparison of our NSOS system searching for goal-door 111 in the Inverse map with different $\alpha$ values.}					
%	         \end{figure}	
%\end{frame}
%========================================== ############# ======================================================
%========================================== ############# ======================================================
%========================================== ############# ======================================================
%\begin{frame}{Experiments and Results}
%
%			\centering	
%			\begin{figure}
%			    \begin{subfigure}{.45\columnwidth}
%			    		\centering
%					 \includegraphics[width=.7\linewidth]{figs/4_111_0.8_inverse_map/82.png}
%     				 \caption{\tiny Inverse map, goal-door 111, $\alpha = 80\%$.}
%			    	\end{subfigure}
%			    \begin{subfigure}{.45\columnwidth}
%			    		\centering			    
%			    		\includegraphics[width=.7\linewidth]{figs/7_111_0.9_inverse_map/82.png}
%     				 \caption{\tiny Inverse map, goal-door 111, $\alpha = 90\%$.}
%			    	\end{subfigure}						    				    				 
%				 \caption{\tiny Comparison of our NSOS system searching for goal-door 111 in the Inverse map with different $\alpha$ values.}					
%	         \end{figure}	
%\end{frame}
%========================================== ############# ======================================================
%========================================== ############# ======================================================
%========================================== ############# ======================================================
%\begin{frame}{Experiments and Results}
%
%			\centering	
%			\begin{figure}
%			    \begin{subfigure}{.45\columnwidth}
%			    		\centering
%					 \includegraphics[width=.7\linewidth]{figs/4_111_0.8_inverse_map/92.png}
%     				 \caption{\tiny Inverse map, goal-door 111, $\alpha = 80\%$.}
%			    	\end{subfigure}
%			    \begin{subfigure}{.45\columnwidth}
%			    		\centering			    
%			    		\includegraphics[width=.7\linewidth]{figs/7_111_0.9_inverse_map/92.png}
%     				 \caption{\tiny Inverse map, goal-door 111, $\alpha = 90\%$.}
%			    	\end{subfigure}						    				    				 
%				 \caption{\tiny Comparison of our NSOS system searching for goal-door 111 in the Inverse map with different $\alpha$ values.}					
%	         \end{figure}	
%\end{frame}
%========================================== ############# ======================================================
%========================================== ############# ======================================================
%========================================== ############# ======================================================
%\begin{frame}{Experiments and Results}
%
%			\centering	
%			\begin{figure}
%			    \begin{subfigure}{.45\columnwidth}
%			    		\centering
%					 \includegraphics[width=.7\linewidth]{figs/4_111_0.8_inverse_map/102.png}
%     				 \caption{\tiny Inverse map, goal-door 111, $\alpha = 80\%$.}
%			    	\end{subfigure}
%			    \begin{subfigure}{.45\columnwidth}
%			    		\centering			    
%			    		\includegraphics[width=.7\linewidth]{figs/7_111_0.9_inverse_map/102.png}
%     				 \caption{\tiny Inverse map, goal-door 111, $\alpha = 90\%$.}
%			    	\end{subfigure}						    				    				 
%				 \caption{\tiny Comparison of our NSOS system searching for goal-door 111 in the Inverse map with different $\alpha$ values.}					
%	         \end{figure}	
%\end{frame}
%========================================== ############# ======================================================
%========================================== ############# ======================================================
%========================================== ############# ======================================================
%\begin{frame}{Experiments and Results}
%
%			\centering	
%			\begin{figure}
%			    \begin{subfigure}{.45\columnwidth}
%			    		\centering
%					 \includegraphics[width=.7\linewidth]{figs/4_111_0.8_inverse_map/112.png}
%     				 \caption{\tiny Inverse map, goal-door 111, $\alpha = 80\%$.}
%			    	\end{subfigure}
%			    \begin{subfigure}{.45\columnwidth}
%			    		\centering			    
%			    		\includegraphics[width=.7\linewidth]{figs/7_111_0.9_inverse_map/112.png}
%     				 \caption{\tiny Inverse map, goal-door 111, $\alpha = 90\%$.}
%			    	\end{subfigure}						    				    				 
%				 \caption{\tiny Comparison of our NSOS system searching for goal-door 111 in the Inverse map with different $\alpha$ values.}					
%	         \end{figure}	
%\end{frame}
%========================================== ############# ======================================================
%========================================== ############# ======================================================
%========================================== ############# ======================================================
%\begin{frame}{Experiments and Results}
%
%			\centering	
%			\begin{figure}
%			    \begin{subfigure}{.45\columnwidth}
%			    		\centering
%					 \includegraphics[width=.7\linewidth]{figs/4_111_0.8_inverse_map/122.png}
%     				 \caption{\tiny Inverse map, goal-door 111, $\alpha = 80\%$.}
%			    	\end{subfigure}
%			    \begin{subfigure}{.45\columnwidth}
%			    		\centering			    
%			    		\includegraphics[width=.7\linewidth]{figs/7_111_0.9_inverse_map/112.png}
%     				 \caption{\tiny Inverse map, goal-door 111, $\alpha = 90\%$.}
%			    	\end{subfigure}						    				    				 
%				 \caption{\tiny Comparison of our NSOS system searching for goal-door 111 in the Inverse map with different $\alpha$ values.}					
%	         \end{figure}	
%\end{frame}
%========================================== ############# ======================================================
%========================================== ############# ======================================================
%========================================== ############# ======================================================
%\begin{frame}{Experiments and Results}
%
%			\centering	
%			\begin{figure}
%			    \begin{subfigure}{.45\columnwidth}
%			    		\centering
%					 \includegraphics[width=.7\linewidth]{figs/4_111_0.8_inverse_map/132.png}
%     				 \caption{\tiny Inverse map, goal-door 111, $\alpha = 80\%$.}
%			    	\end{subfigure}
%			    \begin{subfigure}{.45\columnwidth}
%			    		\centering			    
%			    		\includegraphics[width=.7\linewidth]{figs/7_111_0.9_inverse_map/112.png}
%     				 \caption{\tiny Inverse map, goal-door 111, $\alpha = 90\%$.}
%			    	\end{subfigure}						    				    				 
%				 \caption{\tiny Comparison of our NSOS system searching for goal-door 111 in the Inverse map with different $\alpha$ values.}					
%	         \end{figure}	
%\end{frame}
%========================================== ############# ======================================================
%========================================== ############# ======================================================
%========================================== ############# ======================================================
\begin{frame}{Experiments and Results}

			\centering	
			\begin{figure}
			    \begin{subfigure}{.45\columnwidth}
			    		\centering
					 \includegraphics[width=.7\linewidth]{figs/4_111_0.8_inverse_map/142.png}
     				 \caption{\tiny Inverse map, goal-door 111, $\alpha = 80\%$.}
			    	\end{subfigure}
			    \begin{subfigure}{.45\columnwidth}
			    		\centering			    
			    		\includegraphics[width=.7\linewidth]{figs/7_111_0.9_inverse_map/112.png}
     				 \caption{\tiny Inverse map, goal-door 111, $\alpha = 90\%$.}
			    	\end{subfigure}						    				    				 
				 \caption{\tiny Comparison of our NSOS system searching for goal-door 111 in the Inverse map with different $\alpha$ values.}					
	         \end{figure}	
\end{frame}
%========================================== ############# ======================================================
%========================================== ############# ======================================================
%========================================== ############# ======================================================
\begin{frame}{Experiments and Results}

			\centering	
			\begin{figure}
			    \begin{subfigure}{.3\columnwidth}
			    		\centering
					 \includegraphics[width=1\linewidth]{figs/normal_shortest_path_percentage.png}
     				 \caption{\tiny Normal map.}			 		 
			    	\end{subfigure}
			    \begin{subfigure}{.3\columnwidth}
			    		\centering			    
			    		\includegraphics[width=1\linewidth]{figs/inverse_shortest_path_percentage.png}
     				 \caption{\tiny Inverse map.}	
			    	\end{subfigure}		
			    \begin{subfigure}{.3\columnwidth}
			    		\centering			    
					 \includegraphics[width=1\linewidth]{figs/hotel_shortest_path_percentage.png}
     				 \caption{\tiny Hotel map.}		 
			    	\end{subfigure}				    				    				 
				 \caption{\tiny Comparison of the traveled distances for each goal-door with its respective shortest distance.}					 
	         \end{figure}	
\end{frame}
%========================================== ############# ======================================================
%========================================== ############# ======================================================
%========================================== ############# ======================================================
\begin{frame}{Experiments and Results}
	\begin{columns}[T]
		\column{0.3\textwidth}
		\begin{itemize}
			\item Human participants in the OS task
		\end{itemize}
		\column{0.35\textwidth}
				\begin{figure}	
					\centering	
					\includegraphics[width=.8\linewidth]{figs/RESULTS_HUMANS_NEW_T-eps-converted-to}
					\caption{Comparison of the traveled distances for two goal-doors.}
				\end{figure}
		\column{0.4\textwidth}					
				\hspace{-1cm}
				\begin{figure}
					\centering
				    \begin{subfigure}{1\columnwidth}	
				    		\centering								
						\includegraphics[width=.56\linewidth]{figs/map_normal_redcircles.png}
						\caption{\tiny Normal map.}						
				    	\end{subfigure}
				    \begin{subfigure}{1\columnwidth}												
				    		\centering		
				    		\vspace{.15cm}		    
						\includegraphics[width=.5\linewidth]{figs/hotel2_PAPER.png}
						\caption{\tiny Hotel map.}						
				    	\end{subfigure}						
						\caption{\tiny Maps used for the experiments with the participants.}										    	
				\end{figure}	
	\end{columns}
\end{frame}
%========================================== ############# ======================================================
%========================================== ############# ======================================================
%========================================== ############# ======================================================
\begin{frame}{Experiments and Results}
	\begin{columns}[T]
		\column{0.45\textwidth}
		\begin{itemize}
			\item Experiments with the physical Robot
			\begin{itemize}
				\item Pioneer 3DX
				\item 2 cameras (left and right)
				\item Laptop
				\item Inf, UFRGS
			\end{itemize}			
		\end{itemize}
		\column{0.55\textwidth}					
				\begin{figure}
				    \begin{subfigure}{1\columnwidth}	
				    		\centering								
						\includegraphics[width=.4\linewidth]{figs/real_environment_map.png}	
						\caption{\tiny Map of the environment used for this experiment.}												
				    	\end{subfigure}
				    \begin{subfigure}{1\columnwidth}												
				    		\centering		
				    		\vspace{.15cm}		    
						\includegraphics[width=.4\linewidth]{figs/real_scenario_labeled2.png}
						\caption{\tiny Pioneer 3DX, sensors and the door signs.}												
				    	\end{subfigure}						
					\caption{\tiny Experiments with the physical robot.}										    	
				\end{figure}	
	\end{columns}
\end{frame}
%========================================== ############# ======================================================
%========================================== ############# ======================================================
%========================================== ############# ======================================================
\begin{frame}[noframenumbering]{Experiments and Results}
	\begin{columns}[T]
		\column{0.45\textwidth}
		\vspace{-.47cm}
		\begin{itemize}
			\item Experiments with the physical Robot
			\begin{itemize}
				\item Pioneer 3DX
				\item 2 cameras (left and right)
				\item Laptop
				\item Inf, UFRGS
			\end{itemize}			
			\item Goal-door: \textbf{232}
		\end{itemize}
		\column{0.55\textwidth}					
				\begin{figure}
				    	\centering								
					\includegraphics[width=.75\linewidth]{figs/experiment_physical_robot.png}						
					\caption{\tiny Step by step of the experiment with the physical robot.}
				\end{figure}	
	\end{columns}
\end{frame}
%========================================== ############# ======================================================
%========================================== ############# ======================================================
%========================================== ############# ======================================================
\begin{frame}{Organisational semantic information from numbers}
	\begin{columns}[T]
		\column{1\textwidth}
		\begin{itemize}[<+->]
			\item It is \textbf{possible} to \textbf{expand} the \textbf{robot's perception} with \textbf{organisational semantic information} inferred from numbers
			\item The \textbf{organisational semantic information} helps in the estimations
			\item \textbf{NSOS system} that:
			\begin{itemize}
	%			\item  Relies on \textbf{organisational semantic information} within the environment
				\item  Does \textbf{not require} that the \textbf{door signs} are set according to a \textbf{specific pattern}
				\item  \textbf{Performs better} than the \textbf{greedy search} and \textbf{humans participants}
%				\item  Demonstrates the \textbf{difficulty} of the \textbf{OS problem} by the experiments with human participants
			\end{itemize}
			\item \textbf{Door signs} and \textbf{corridors} do \textbf{not change} very \textbf{often}
			\item What about \textbf{OS} in \textbf{semi-dynamic environments}?
		\end{itemize}
%		\column{0.55\textwidth}
	\end{columns}
\end{frame}
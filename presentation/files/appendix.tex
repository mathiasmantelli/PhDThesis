\begin{frame}[noframenumbering]{Mapping module}
	\begin{columns}[T]
		\column{.5\textwidth}
		\vspace{.95cm}
		\begin{itemize}
			\item A \textbf{2D grid} map is build ({\small Simulation and real worlds})
		\end{itemize}
		\column{.5\textwidth}	
			\centering	
			\begin{figure}
			    \begin{subfigure}[b]{.6\columnwidth}
			    		\centering			
					 \includegraphics[width=.6\linewidth]{figs/2d_map_simulation.png}
     				 \caption{\tiny Mapping in simulation with HIMM.\footnotemark[14]}		 
			    	\end{subfigure}
%			    	\vspace{-.3cm}\\
			    \begin{subfigure}[b]{.6\columnwidth}
			    		\centering					    	
			    		\includegraphics[width=.6\linewidth]{figs/2d_map_real.png}	         
			    		\caption{\tiny Mapping the real world with Gmapping.\footnotemark[15]}				 
			    	\end{subfigure}				
%			    	\vspace{-.7cm}	 
				 \caption{\tiny The 2D grid maps used in this work.}			 
	         \end{figure}		         		
	\end{columns}		
	\footnotetext[14]{\tiny Borenstein, J., and Yoram K. ``\textit{Histogramic in-motion mapping for mobile robot obstacle avoidance}.'' Transactions on robotics and automation. 1991.}
	\footnotetext[15]{\tiny Grisetti, G., Stachniss, C., and Burgard, W. ``\textit{Improving grid-based slam with rao-blackwellized particle filters by adaptive proposals and selective resampling.}'' ICRA. 2005.}		
\end{frame}
%========================================== ############# ======================================================        
%========================================== ############# ======================================================
%========================================== ############# ======================================================
\begin{frame}[noframenumbering]{Mapping module}
	\begin{columns}[T]
		\column{.5\textwidth}
		\vspace{1.2cm}		
		\begin{itemize}
			\item A \textbf{2D grid} map is build ({\small Simulation and real worlds})
			\item \textbf{Voronoi} diagram 
		\end{itemize}
		\column{.5\textwidth}	
			\centering	
			\begin{figure}
			    \begin{subfigure}[b]{.6\columnwidth}
			    		\centering			
					 \includegraphics[width=.6\linewidth]{figs/2d_map_voronoi_simulation.png}
     				 \caption{\tiny Mapping in simulation with HIMM.\footnotemark[14]}			 
			    	\end{subfigure}
%			    	\vspace{-.3cm}\\
			    \begin{subfigure}[b]{.6\columnwidth}
			    		\centering					    	
			    		\includegraphics[width=.6\linewidth]{figs/2d_map_voronoi_real.png}	         
			    		\caption{\tiny Mapping the real world with Gmapping.\footnotemark[15]}					 
			    	\end{subfigure}				
%			    	\vspace{-.7cm}	 
				 \caption{\tiny The 2D grid maps used in this work.}			 
	         \end{figure}		         		
	\end{columns}		
	\footnotetext[14]{\tiny Borenstein, J., and Yoram K. ``\textit{Histogramic in-motion mapping for mobile robot obstacle avoidance}.'' Transactions on robotics and automation. 1991.}
	\footnotetext[15]{\tiny Grisetti, G., Stachniss, C., and Burgard, W. ``\textit{Improving grid-based slam with rao-blackwellized particle filters by adaptive proposals and selective resampling.}'' ICRA. 2005.}		
\end{frame}		
%========================================== ############# ======================================================        
%========================================== ############# ======================================================
%========================================== ############# ======================================================
\begin{frame}[noframenumbering]{Mapping module}
	\begin{columns}[T]
		\column{.5\textwidth}
		\vspace{1.2cm}		
		\begin{itemize}
			\item A \textbf{2D grid} map is build ({\small Simulation and real worlds})
			\item \textbf{Voronoi} diagram 
			\item \textbf{Visited} regions
		\end{itemize}
		\column{.5\textwidth}	
			\centering	
			\begin{figure}
			    \begin{subfigure}[b]{.6\columnwidth}
			    		\centering			
					 \includegraphics[width=.6\linewidth]{figs/2d_map_visited_simulation.png}
     				 \caption{\tiny Mapping in simulation with HIMM.\footnotemark[14]}			 
			    	\end{subfigure}
%			    	\vspace{-.3cm}\\
			    \begin{subfigure}[b]{.6\columnwidth}
			    		\centering					    	
			    		\includegraphics[width=.6\linewidth]{figs/2d_map_visited_real.png}	         
			    		\caption{\tiny Mapping the real world with Gmapping.\footnotemark[15]}					 
			    	\end{subfigure}				
%			    	\vspace{-.7cm}	 
				 \caption{\tiny The 2D grid maps used in this work.}			 
	         \end{figure}		         		
	\end{columns}		
	\footnotetext[14]{\tiny Borenstein, J., and Yoram K. ``\textit{Histogramic in-motion mapping for mobile robot obstacle avoidance}.'' Transactions on robotics and automation. 1991.}
	\footnotetext[15]{\tiny Grisetti, G., Stachniss, C., and Burgard, W. ``\textit{Improving grid-based slam with rao-blackwellized particle filters by adaptive proposals and selective resampling.}'' ICRA. 2005.}		
\end{frame}			
%========================================== ############# ======================================================        
%========================================== ############# ======================================================
%========================================== ############# ======================================================
\begin{frame}[noframenumbering]{Mapping module}
	\begin{columns}[T]
		\column{.5\textwidth}
		\vspace{1.2cm}		
		\begin{itemize}
			\item A \textbf{2D grid} map is build ({\small Simulation and real worlds})
			\item \textbf{Voronoi} diagram 
			\item \textbf{Visited} regions
			\item \textbf{Frontiers} (visited and unvisited regions)
		\end{itemize}
		\column{.5\textwidth}	
			\centering	
			\begin{figure}
			    \begin{subfigure}[b]{.6\columnwidth}
			    		\centering			
					 \includegraphics[width=.6\linewidth]{figs/2d_map_visited_simulation.png}
     				 \caption{\tiny Mapping in simulation with HIMM.\footnotemark[14]}			 
			    	\end{subfigure}
%			    	\vspace{-.3cm}\\
			    \begin{subfigure}[b]{.6\columnwidth}
			    		\centering					    	
			    		\includegraphics[width=.6\linewidth]{figs/2d_map_visited_real.png}	         
			    		\caption{\tiny Mapping the real world with Gmapping.\footnotemark[15]}					 
			    	\end{subfigure}				
%			    	\vspace{-.7cm}	 
				 \caption{\tiny The 2D grid maps used in this work.}			 
	         \end{figure}		         		
	\end{columns}		
	\footnotetext[14]{\tiny Borenstein, J., and Yoram K. ``\textit{Histogramic in-motion mapping for mobile robot obstacle avoidance}.'' Transactions on robotics and automation. 1991.}
	\footnotetext[15]{\tiny Grisetti, G., Stachniss, C., and Burgard, W. ``\textit{Improving grid-based slam with rao-blackwellized particle filters by adaptive proposals and selective resampling.}'' ICRA. 2005.}		
\end{frame}	
%========================================== ############# ======================================================
%========================================== ############# ======================================================
%========================================== ############# ======================================================
\begin{frame}[noframenumbering]{Map Segmentation module}
	\begin{columns}[T]
		\column{.5\textwidth}
		\begin{itemize}
			\item \textbf{Divides} the map based on the \textbf{free space}	
		\end{itemize}
		\column{.5\textwidth}	
			\centering	
			\begin{figure}
			    \begin{subfigure}{.4\columnwidth}		
					 \includegraphics[width=1\linewidth]{figs/33.png}
     				 \caption{\tiny The 2D grid map segmented.}			 		 
			    	\end{subfigure}
			    \begin{subfigure}{.4\columnwidth}
			    		\includegraphics[width=1\linewidth]{figs/44.png}	         
			    		\caption{\tiny The different segments identified by different colors.}					 
			    	\end{subfigure}					 
				 \caption{\tiny The 2D grid maps used in this work.}				 
	         \end{figure}		         		
	\end{columns}		
\end{frame}
%========================================== ############# ======================================================
%========================================== ############# ======================================================
%========================================== ############# ======================================================
\begin{frame}[noframenumbering]{Map Segmentation module}
	\begin{columns}[T]
		\column{.5\textwidth}
		\begin{itemize}
			\item \textbf{Divides} the map based on the \textbf{free space}	
			\item \textbf{Simultaneously} to the \textbf{Mapping module}
		\end{itemize}
		\column{.5\textwidth}	
			\centering	
			\begin{figure}
			    \begin{subfigure}{.4\columnwidth}		
					 \includegraphics[width=1\linewidth]{figs/33.png}
     				 \caption{\tiny The 2D grid map segmented.}			 		 
			    	\end{subfigure}
			    \begin{subfigure}{.4\columnwidth}
			    		\includegraphics[width=1\linewidth]{figs/44.png}	         
			    		\caption{\tiny The different segments identified by different colors.}					 
			    	\end{subfigure}					 
				 \caption{\tiny The 2D grid maps used in this work.}				 
	         \end{figure}			         		
	\end{columns}		
\end{frame}
%========================================== ############# ======================================================
%========================================== ############# ======================================================
%========================================== ############# ======================================================
\begin{frame}[noframenumbering]{Map Segmentation module}
	\begin{columns}[T]
		\column{.5\textwidth}
		\begin{itemize}
			\item \textbf{Divides} the map based on the \textbf{free space}	
			\item \textbf{Simultaneously} to the \textbf{Mapping module}
			\item Kernel Density Estimation (\textbf{KDE})\footnotemark[16]
		\end{itemize}
		\column{.5\textwidth}	
			\centering	
			\begin{figure}
			    \begin{subfigure}{.4\columnwidth}		
					 \includegraphics[width=1\linewidth]{figs/33.png}
     				 \caption{\tiny The 2D grid map segmented.}			 		 
			    	\end{subfigure}
			    \begin{subfigure}{.4\columnwidth}
			    		\includegraphics[width=1\linewidth]{figs/44.png}	         
			    		\caption{\tiny The different segments identified by different colors.}					 
			    	\end{subfigure}					 
				 \caption{\tiny The 2D grid maps used in this work.}				 
	         \end{figure}		         		
	\end{columns}		
	\footnotetext[16]{\tiny Maffei, Renan, et al. ``\textit{Fast monte carlo localization using spatial density information.}'', ICRA. 2015}
\end{frame}
%========================================== ############# ======================================================
%========================================== ############# ======================================================
%========================================== ############# ======================================================
\begin{frame}[noframenumbering]{Map Segmentation module}
	\begin{columns}[T]
		\column{.5\textwidth}
		\begin{itemize}
			\item \textbf{Divides} the map based on the \textbf{free space}	
			\item \textbf{Simultaneously} to the \textbf{Mapping module}
			\item Kernel Density Estimation (\textbf{KDE})\footnotemark[16]
			\item An \textbf{abstraction} of the \textbf{2D grid map} for \textbf{grouping} the \textbf{detected door signs}
		\end{itemize}
		\column{.5\textwidth}	
			\centering	
			\begin{figure}
			    \begin{subfigure}{.4\columnwidth}		
					 \includegraphics[width=1\linewidth]{figs/33.png}
     				 \caption{\tiny The 2D grid map segmented.}			 		 
			    	\end{subfigure}
			    \begin{subfigure}{.4\columnwidth}
			    		\includegraphics[width=1\linewidth]{figs/44.png}	         
			    		\caption{\tiny The different segments identified by different colors.}					 
			    	\end{subfigure}					 
				 \caption{\tiny The 2D grid maps used in this work.}				 
	         \end{figure}	         		
	\end{columns}		
	\footnotetext[16]{\tiny Maffei, Renan, et al. ``\textit{Fast monte carlo localization using spatial density information.}'', ICRA. 2015}	
\end{frame}
%========================================== ############# ======================================================
%========================================== ############# ======================================================
%========================================== ############# ======================================================
\begin{frame}[noframenumbering]{Image Processing module}
	\begin{columns}[T]
		\column{0.5\textwidth}
		\begin{itemize}
			\item Optical character recognition (\textbf{OCR})
		\end{itemize}
		\column{0.5\textwidth}
			\centering
			\begin{figure}
				 \includegraphics[width=.9\linewidth]{figs/image_processing.png}
				 \caption{\tiny Illustration of the Image Processing module detecting a door label.}			 		 
	        \end{figure}			
	\end{columns}	
\end{frame}
%========================================== ############# ======================================================
%========================================== ############# ======================================================
%========================================== ############# ======================================================
\begin{frame}[noframenumbering]{Image Processing module}
	\begin{columns}[T]
		\column{0.5\textwidth}
		\begin{itemize}
			\item Optical character recognition (\textbf{OCR})
			\item \textbf{Recognize} the number of \textbf{door signs} in \textbf{RGB images}
		\end{itemize}
		\column{0.5\textwidth}
			\centering
			\begin{figure}
				 \includegraphics[width=.9\linewidth]{figs/image_processing.png}
				 \caption{\tiny Illustration of the Image Processing module detecting a door label.}		
	        \end{figure}			
	\end{columns}	
\end{frame}
%========================================== ############# ======================================================
%========================================== ############# ======================================================
%========================================== ############# ======================================================
\begin{frame}[noframenumbering]{Image Processing module}
	\begin{columns}[T]
		\column{0.5\textwidth}
		\begin{itemize}
			\item Optical character recognition (\textbf{OCR})
			\item \textbf{Recognize} the number of \textbf{door signs} in \textbf{RGB images}
			\item We use the work proposed by Neumann and Matas (2012)\footnotemark[9]
		\end{itemize}
		\column{0.5\textwidth}
			\centering
			\begin{figure}
				 \includegraphics[width=.9\linewidth]{figs/image_processing.png}
				 \caption{\tiny Illustration of the Image Processing module detecting a door label.}				 		 
	        \end{figure}			
	\end{columns}	
	\footnotetext[9]{\tiny Neumann, Lukáš, and Jiří Matas. ``\textit{Real-time scene text localization and recognition}.'',Conference on Computer Vision and Pattern Recognition. 2012.}	
\end{frame}
%========================================== ############# ======================================================
%========================================== ############# ======================================================
%========================================== ############# ======================================================
\begin{frame}[noframenumbering]{Image Processing module}
	\begin{columns}[T]
		\column{0.5\textwidth}
		\begin{itemize}
			\item Optical character recognition (\textbf{OCR})
			\item \textbf{Recognize} the number of \textbf{door signs} in \textbf{RGB images}
			\item We use the work proposed by Neumann and Matas (2012)\footnotemark[9]
			\item The \textbf{recognized numbers} are \textbf{associated} to the \textbf{map segments}
		\end{itemize}
		\column{0.5\textwidth}
			\centering
			\begin{figure}
				 \includegraphics[width=.9\linewidth]{figs/image_processing.png}
				 \caption{\tiny Illustration of the Image Processing module detecting a door label.}		
	        \end{figure}			
	\end{columns}	
	\footnotetext[9]{\tiny Neumann, Lukáš, and Jiří Matas. ``\textit{Real-time scene text localization and recognition}.'',Conference on Computer Vision and Pattern Recognition. 2012.}		
\end{frame}
%========================================== ############# ======================================================
%========================================== ############# ======================================================
%========================================== ############# ======================================================
\begin{frame}[noframenumbering]{Equations - Semantic OS System based on text}
	\begin{itemize}
		\item Map segmentation: $\Psi({c}_k) = \sum_{c}^\mathbf{T} Q({c})K(\left \| {c} - {c}_k \right \|)$\\~~~~~~~~~~~~~~~~~~~~~~~~~~$Q({c}) = 
\begin{cases}
1 & \text{, if ${c}$ is a free cell} \\ 
0 & \text{, otherwise.} 
\end{cases}$\\~~~~~~~~~~~~~~~~~~~~~~~~~~$K(d) = \begin{cases}
a & \text{, if $d \leq r$} \\ 
0 & \text{, otherwise,} 
\end{cases}$\\~~~~~~~~~~~~~~~~~~~~~~~~~~$\Upsilon({c}_k) = \left \lfloor  \Psi({c}_k) / \delta  \right \rfloor$
	\end{itemize}
\end{frame}
%========================================== ############# ======================================================
%========================================== ############# ======================================================
%========================================== ############# ======================================================
\begin{frame}[noframenumbering]{Equations - Semantic OS System based on text}
	\begin{itemize}
		\item Growing Direction factor: $\zeta\big(S({c})\big) = \frac{\Big(L^<\big(S({c})\big) - L^>\big(S({c})\big)\Big)}{\max\Big(L^<\big(S({c})\big)+L^>\big(S({c})\big), w_g\Big)},$\\~~~~~~~~~~~~~~~~~~~~~~~~~~~~~~~~~$\gamma(\theta_f({c})) = 1.0 + \left | \frac{\theta_f({c}) - \theta_i\big(S({c})\big)}{\pi} \right | * - 2.0$\\~~~~~~~~~~~~~~~~~~~~~~~~~~~~~~~~~~~~~$\varphi_g({c}) = \frac{\zeta\big(S({c})\big) * \gamma\big(\theta_f({c})\big) + 1.0}{2.0}$
	\end{itemize}
\end{frame}
%========================================== ############# ======================================================
%========================================== ############# ======================================================
%========================================== ############# ======================================================
\begin{frame}[noframenumbering]{Equations - Semantic OS System based on text}
	\begin{itemize}
		\item Parity factor: $\varphi_p({c}) = 0.5 + \frac{L^=\big(S({c})\big) - L^{\neq}\big(S({c})\big)}{\max\Big(L^=\big(S({c})\big) + L^{\neq}\big(S({c})\big), w_p\Big)}*0.5$
	\end{itemize}
\end{frame}
%========================================== ############# ======================================================
%========================================== ############# ======================================================
%========================================== ############# ======================================================
\begin{frame}[noframenumbering]{Equations - Semantic OS System based on text}
	\begin{figure}
		\includegraphics[width=.35\linewidth]{figs/RESULTS_NORMAL_NEW_T-eps-converted-to}		
		\caption{\tiny Normal map.}
	\end{figure}
\end{frame}
%========================================== ############# ======================================================
%========================================== ############# ======================================================
%========================================== ############# ======================================================
\begin{frame}[noframenumbering]{Equations - Semantic OS System based on text}
	\begin{figure}
		\includegraphics[width=.35\linewidth]{figs/RESULTS_INVERSE_NEW_T-eps-converted-to}		
		\caption{\tiny Inverse map.}
	\end{figure}
\end{frame}
%========================================== ############# ======================================================
%========================================== ############# ======================================================
%========================================== ############# ======================================================
\begin{frame}[noframenumbering]{Equations - Semantic OS System based on text}
	\begin{figure}
		\includegraphics[width=.35\linewidth]{figs/RESULTS_HOTEL_NEW_T-eps-converted-to}		
		\caption{\tiny Hotel map.}
	\end{figure}
\end{frame}
%========================================== ############# ======================================================
%========================================== ############# ======================================================
%========================================== ############# ======================================================
\begin{frame}[noframenumbering]{Equations - Semantic OS System based on text}
	\begin{figure}
		\includegraphics[width=.35\linewidth]{figs/RESULTS_KTH_NEW_T-eps-converted-to}		
		\caption{\tiny KTH map.}
	\end{figure}
\end{frame}





%%========================================== ############# ======================================================
%%========================================== ############# ======================================================
%%========================================== ############# ======================================================
%\begin{frame}[noframenumbering]{First years of mobile robotics}
%	\begin{columns}[T]
%		\column{0.48\textwidth}
%		\begin{itemize}
%			\item Ages of mobile robotics\footnotemark[1]:
%			\begin{itemize}
%				\item Classical age (1986-2004):
%				\begin{itemize}
%					\item {\scriptsize \textbf{Introduction} of the \textbf{main probabilistic} formulations for \textbf{SLAM}}
%					\item {\scriptsize \textbf{Lidar} and \textbf{sonar} sensors}					
%				\end{itemize}
%				
%			\end{itemize}
%
%		\end{itemize}
%		\column{0.52\textwidth}
%			\centering
%			\begin{figure}
%			    \begin{subfigure}[b]{.5\columnwidth}
%			    		\centering			
%			    		\includegraphics[width=.9\linewidth]{figs/mapping_thrun.png}
%			    		\caption{\tiny Online mapping\footnotemark[2]}
%			    	\end{subfigure}~
%			    \begin{subfigure}[b]{.5\columnwidth}
%			    		\centering					    	
%			    		\includegraphics[width=.85\linewidth]{figs/CML_newman.png}	         
%			    		\caption{\tiny Real Time CML\footnotemark[3]}					 
%			    	\end{subfigure}					 
%				 \caption{\tiny Initial works on SLAM}			 
%	         \end{figure}
%		
%	\end{columns}	
%	\footnotetext[1]{\tiny Cesar, Cadena, et al. ``\textit{Simultaneous Localization And Mapping: Present Future and the Robust-Perception Age.}'' arXiv preprint arXiv: 1606.05830. 2016.}
%	\footnotetext[2]{\tiny Thrun, Sebastian. ``\textit{An Online Mapping Algorithm for Teams of Mobile Robots}''. Carnegie-Mellon Univ Pittsburgh PA School of Computer Science, 2000.}
%	\footnotetext[3]{\tiny Newman, Paul, et al. ``\textit{Explore and return: Experimental validation of real-time concurrent mapping and localization.}'' ICRA, 2002}	
%\end{frame}
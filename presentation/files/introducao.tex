%\begin{frame}{First years of mobile robotics}
%	\begin{columns}[T]
%		\column{0.48\textwidth}
%		\begin{itemize}
%			\item Ages of mobile robotics:
%			\begin{itemize}
%				\item Classical age (1986-2004)
%			\end{itemize}
%
%		\end{itemize}
%		\column{0.52\textwidth}
%			\centering
%			\begin{figure}
%			    \begin{subfigure}[b]{.5\columnwidth}
%			    		\centering			
%			    		\includegraphics[width=.9\linewidth]{figs/CML_newman.png}
%			    		\caption{\tiny Real Time CML\footnotemark[1]}
%			    	\end{subfigure}~
%			    \begin{subfigure}[b]{.5\columnwidth}
%			    		\centering					    	
%			    		\includegraphics[width=1\linewidth]{figs/mapping_thrun.png}	         
%			    		\caption{\tiny Online mapping\footnotemark[2]}					 
%			    	\end{subfigure}					 
%				 \caption{\tiny Initial works on SLAM}			 
%	         \end{figure}
%		
%	\end{columns}	
%	\footnotetext[1]{\tiny Newman, Paul, et al. ``\textit{Explore and return: Experimental validation of real-time concurrent mapping and localization.}'' ICRA, 2002}
%	\footnotetext[2]{\tiny Thrun, Sebastian. ``\textit{An Online Mapping Algorithm for Teams of Mobile Robots}''. Carnegie-Mellon Univ Pittsburgh PA School of Computer Science, 2000.}
%\end{frame}
%========================================== ############# ======================================================
%========================================== ############# ======================================================
%========================================== ############# ======================================================
\begin{frame}{Introducing Service Robots}
	\begin{columns}[T]
		\column{0.52\textwidth}
		\begin{itemize}
			\item \textbf{Robots} are \textbf{grouped} into different \textbf{classes} depending on \textbf{their use}%\footnotemark[1]:
			\item \textbf{Service Robots} (SRs)\footnotemark[1]:
			\begin{itemize}
				\item {\scriptsize \textbf{Professional} SRs (PSRs)}
			\end{itemize}
		\end{itemize}
		\column{0.48\textwidth}
			\centering
			\begin{figure}
			    \begin{subfigure}[b]{.7\columnwidth}
			    		\centering			
			    		\includegraphics[width=\linewidth]{figs/professional_robots.png}
			    	\end{subfigure}~
				 \caption{\tiny Example of Professional SRs\footnotemark[2].}			 
	         \end{figure}
	\end{columns}	
	\footnotetext[1]{\tiny LITZENBERGER, G. Service robots. \textit{``International Federation of Robotics''}, 2018. Image extracted from: \url{<https://ifr.org/service-robots>}.}
	\footnotetext[2]{\tiny GARCIA-HARO, J. M. et al. \textit{``Service robots in catering applications: A review and future
challenges''}. Electronics, MDPI, v. 10, n. 1, p. 47, 2020.}
\end{frame}
%========================================== ############# ======================================================
%========================================== ############# ======================================================
%========================================== ############# ======================================================
\begin{frame}[noframenumbering]{Introducing Service Robots}
	\begin{columns}[T]
		\column{0.52\textwidth}
		\begin{itemize}
			\item \textbf{Robots} are \textbf{grouped} into different \textbf{classes} depending on \textbf{their use}%\footnotemark[1]:
			\item \textbf{Service Robots} (SRs)\footnotemark[1]:
			\begin{itemize}
				\item {\scriptsize \textbf{Professional} SRs (PSRs)}
				\item {\scriptsize \textbf{Domestic} SRs (DSRs)}
			\end{itemize}
		\end{itemize}
		\column{0.48\textwidth}
			\centering
			\begin{figure}
			    \begin{subfigure}[b]{.7\columnwidth}
			    		\centering			
			    		\includegraphics[width=.8\linewidth]{figs/personal_robots.png}
			    	\end{subfigure}~
				 \caption{\tiny Example of Domestic SRs\footnotemark[2].}			 
	         \end{figure}
	\end{columns}	
	\footnotetext[1]{\tiny LITZENBERGER, G. Service robots. \textit{``International Federation of Robotics''}, 2018. Image extracted from: \url{<https://ifr.org/service-robots>}.}
	\footnotetext[2]{\tiny GARCIA-HARO, J. M. et al. \textit{``Service robots in catering applications: A review and future
challenges''}. Electronics, MDPI, v. 10, n. 1, p. 47, 2020.}
\end{frame} 
%========================================== ############# ======================================================
%========================================== ############# ======================================================
%========================================== ############# ======================================================
\begin{frame}[noframenumbering]{Introducing Service Robots}
	\begin{columns}[T]
		\column{0.52\textwidth}
		\begin{itemize}
			\item \textbf{Robots} are \textbf{grouped} into different \textbf{classes} depending on \textbf{their use}%\footnotemark[1]:
			\item \textbf{Service Robots} (SRs)\footnotemark[1]:
			\begin{itemize}
				\item {\scriptsize \textbf{Professional} SRs (PSRs)}
				\item {\scriptsize \textbf{Domestic} SRs (DSRs)}
			\end{itemize}
			\item \textbf{DSRs} and \textbf{older people}
		\end{itemize}
		\column{0.48\textwidth}
			\centering
			\begin{figure}
			    \begin{subfigure}[b]{.7\columnwidth}
			    		\centering			
			    		\includegraphics[width=.8\linewidth]{figs/elderly_robot.jpg}
			    	\end{subfigure}~
				 \caption{\tiny Example of SRs helping an old adult\footnotemark[3].}			 
	         \end{figure}
	\end{columns}	
	\footnotetext[1]{\tiny LITZENBERGER, G. Service robots. \textit{``International Federation of Robotics''}, 2018. Image extracted from: \url{<https://ifr.org/service-robots>}.}
	\footnotetext[3]{\tiny Image extracted from:\url{<https://www.bloomberg.com/news/articles/2016-03-17/europe-bets-on-robots-to-help-care-for-seniors>}.}
\end{frame} 
%========================================== ############# ======================================================
%========================================== ############# ======================================================
%========================================== ############# ======================================================
\begin{frame}[noframenumbering]{Introducing Service Robots}
	\begin{columns}[T]
		\column{0.52\textwidth}
		\begin{itemize}
			\item \textbf{Robots} are \textbf{grouped} into different \textbf{classes} depending on \textbf{their use}%\footnotemark[1]:
			\item \textbf{Service Robots} (SRs)\footnotemark[1]:
			\begin{itemize}
				\item {\scriptsize \textbf{Professional} SRs (PSRs)}
				\item {\scriptsize \textbf{Domestic} SRs (DSRs)}
			\end{itemize}
			\item \textbf{DSRs} and \textbf{older people}
			\item \textbf{SRs} and \textbf{COVID-19}, \textbf{outbreaks}, \textbf{hospital infection}
		\end{itemize}
		\column{0.48\textwidth}
			\centering
			\begin{figure}
			    \begin{subfigure}[b]{.7\columnwidth}
			    		\centering			
			    		\includegraphics[width=.45\linewidth]{figs/Jaci_ON.png}
			    	\end{subfigure}~
				 \caption{\tiny Example of a SRs performing environment disinfection\footnotemark[4].}			 
	         \end{figure}
	\end{columns}	
	\footnotetext[1]{\tiny LITZENBERGER, G. Service robots. \textit{``International Federation of Robotics''}, 2018. Image extracted from: \url{<https://ifr.org/service-robots>}.}
	\footnotetext[4]{\tiny MANTELLI, M. F. et al. \textit{``Autonomous environment disinfection based on dynamic uv-c
irradiation map''}. IEEE Robotics and Automation Letters, IEEE, 2022..}
\end{frame}
%========================================== ############# ======================================================
%========================================== ############# ======================================================
%========================================== ############# ======================================================
\begin{frame}[noframenumbering]{Introducing Service Robots}
	\begin{columns}[T]
		\column{0.52\textwidth}
		\begin{itemize}
			\item \textbf{Robots} are \textbf{grouped} into different \textbf{classes} depending on \textbf{their use}%\footnotemark[1]:
			\item \textbf{Service Robots} (SRs)\footnotemark[1]:
			\begin{itemize}
				\item {\scriptsize \textbf{Professional} SRs (PSRs)}
				\item {\scriptsize \textbf{Domestic} SRs (DSRs)}
			\end{itemize}
			\item \textbf{DSRs} and \textbf{older people}
			\item \textbf{SRs} and \textbf{COVID-19}, \textbf{outbreaks}, \textbf{hospital infection}
			\item \textbf{SRs} performing \textbf{object search} (OS) tasks
		\end{itemize}
		\column{0.48\textwidth}
			\centering
			\begin{figure}
			    \begin{subfigure}[b]{.7\columnwidth}
			    		\centering			
			    		\includegraphics[width=.45\linewidth]{figs/Jaci_ON.png}
			    	\end{subfigure}~
				 \caption{\tiny Example of a SRs performing environment disinfection\footnotemark[4].}			 
	         \end{figure}
	\end{columns}	
	\footnotetext[1]{\tiny LITZENBERGER, G. Service robots. \textit{``International Federation of Robotics''}, 2018. Image extracted from: \url{<https://ifr.org/service-robots>}.}
	\footnotetext[4]{\tiny MANTELLI, M. F. et al. \textit{``Autonomous environment disinfection based on dynamic uv-c
irradiation map''}. IEEE Robotics and Automation Letters, IEEE, 2022..}
\end{frame}
%========================================== ############# ======================================================
%========================================== ############# ======================================================
%========================================== ############# ======================================================
\begin{frame}{Object search problem}
	\begin{itemize}[<+->]
		\item \textbf{Challenging} and unsolved \textbf{task for robots}
		\item \textbf{Goal:} estimate the \textbf{target object's} position in an \textbf{environment} and \textbf{move the robot} to \textbf{find} it
		\item {Computation of the \textbf{optimal solution} \textit{vs} Estimation that the current goal \textbf{is not promising}}
		\item The robot's \textbf{perception} is \textbf{crucial}
		\item \textbf{Brute Force} \textit{vs} \textbf{Searching strategy}
		\item \textbf{Human-like} behavior
	\end{itemize}
\end{frame}
%========================================== ############# ======================================================
%========================================== ############# ======================================================
%========================================== ############# ======================================================
\begin{frame}{Searching for objects in organised environments}
	\begin{columns}[T]
		\column{0.52\textwidth}
		\begin{itemize}[<+->]
			\item \textbf{We claim} that, {in general}, \textbf{our society} is \textbf{not randomly organised}
			\item \textbf{Humans} can perform \textbf{searching tasks more efficiently} in \textbf{organised environments}
			\item \textbf{SRs} take advantage of the \textbf{organisation of the environment} to \textbf{improve} their \textbf{performance} in \textbf{OS tasks}
		\end{itemize}
		\column{0.48\textwidth}
			\centering
			\begin{figure}
			    \begin{subfigure}[b]{.4\columnwidth}
			    		\centering			
			    		\includegraphics[width=.57\linewidth]{figs/dictionary.jpg}
			    		\caption{\tiny Dictionary\footnotemark[5]}
			    	\end{subfigure}
			    \begin{subfigure}[b]{.55\columnwidth}
			    		\centering					    	
			    		\includegraphics[width=1\linewidth]{figs/supermarket.jpg}	         
			    		\caption{\tiny Supermarket shelves\footnotemark[6]}					 
			    	\end{subfigure}					 
				 \caption{\tiny Examples of how organisation can help in searching tasks.}			 
	         \end{figure}
	\end{columns}	
	\footnotetext[5]{\tiny Image extracted from:\url{<https://en.wikipedia.org/wiki/Oxford_Dictionary_of_English/>}.}
	\footnotetext[6]{\tiny Image extracted from:\url{<http://www.xingda-shelf.com/en/productlist.php?classid=22&btype=14&22>}.}
\end{frame}
%========================================== ############# ======================================================
%========================================== ############# ======================================================
%========================================== ############# ======================================================
\begin{frame}{Objective}
\centering

\textcolor{red}{\textit{We aim to \textbf{exploit} the \textbf{organisation} of both the \textbf{environment and objects} to
infer \textbf{semantic search cues} to address the \textbf{OS problem}}}
\end{frame}
%========================================== ############# ======================================================
%========================================== ############# ======================================================
%========================================== ############# ======================================================
%\begin{frame}{Geometric perception}
%	\begin{columns}[T]
%		\column{0.6\textwidth}
%			\begin{itemize}[<+->]
%				\item Raw sensor readings
%				\item Useful for many \textbf{robotic tasks}			
%				\item Efficient for \textbf{building maps} and \textbf{state estimation}
%				\item Suitable for \textbf{path-planning} and \textbf{obstacle avoidance}
%				\item \textbf{Limitations}:
%				\begin{itemize}[<+->]	
%					\item Type of maps (\textcolor{cyan}{free}, \textcolor{darkgray}{occupied}, \textcolor{lightgray}{unknown})
%					\item Does not distinguish obstacles
%					\item Insufficient for high-level tasks
%				\end{itemize}				
%				\item How to \textbf{overcome} these \textbf{limitations}?
%			\end{itemize}
%		
%		\column{0.4\textwidth}	
%			\centering
%			\begin{figure}
%				 \includegraphics[width=.9\linewidth]{figs/vacuum_cleaner.png}
%				 \caption{\tiny Vacuum cleaner robot in operation.\footnotemark[6]}			 		 
%	         \end{figure}			
%	\end{columns}	
%	\footnotetext[6]{\tiny Extracted from \url{youtube.com/watch?v=5O8VmDiab3w}}
%\end{frame}
%========================================== ############# ======================================================
%========================================== ############# ======================================================
%========================================== ############# ======================================================
%\begin{frame}{Expand the geometric perception}
%	\begin{columns}[T]
%		\column{0.55\textwidth}
%		\vspace{.7cm}
%		\begin{itemize}
%			\item Ages of mobile robotics\footnotemark[1]: 
%			\begin{itemize}
%				\item Robust-perception age (2015-now):
%				\begin{itemize}[<+->]
%					\item \textbf{Understand} the \textbf{concepts} of parts of the \textbf{sensor readings} ({\small {Semantic} information})		
%					\item \textbf{Associate them} to the \textbf{map} ({\small {Semantic} mapping})
%					\item Enhance robot's autonomy and robustness, \textbf{facilitate} more \textbf{complex tasks}
%					\item Essential for \textbf{high-level reasoning} and \textbf{human-robot interaction}
%				\end{itemize}
%			\end{itemize}
%		\end{itemize}
%		\column{0.45\textwidth}	
%			\centering
%			\begin{figure}
%			    \begin{subfigure}[b]{.6\columnwidth}
%			    		\centering				
%			    		\includegraphics[width=.9\linewidth]{figs/firefighters.png}
%			    		\caption{\tiny The sire of the fire truck.}
%			    	\end{subfigure}\\
%			    \begin{subfigure}[b]{.6\columnwidth}
%			    		\centering			    	
%			    		\includegraphics[width=.9\linewidth]{figs/door_open.png}
%			    		\caption{\tiny The car door.}
%			    	\end{subfigure}

%				 \caption{\tiny Self-Driving System of an autonomous driving car.\footnotemark[7]}			 		 
%	         \end{figure}				
%	\end{columns}
%	\footnotetext[1]{\tiny Cesar, Cadena, et al. ``\textit{Simultaneous Localization And Mapping: Present Future and the Robust-Perception Age.}'' arXiv preprint arXiv: 1606.05830. 2016.}
%	\footnotetext[7]{\tiny Extracted from \url{youtube.com/watch?v=BVRMh9NO9Cs}}
%\end{frame}
%========================================== ############# ======================================================
%========================================== ############# ======================================================
%========================================== ############# ======================================================
%\begin{frame}{Semantic in the robotics field}
%	\textbf{Amount of publications} returned by \textbf{Google Scholar} for the keywords ``\textbf{Semantic}, \textbf{robotics}'':
%\end{frame}
%========================================== ############# ======================================================
%========================================== ############# ======================================================
%========================================== ############# ======================================================
%\begin{frame}{Semantic information within mobile robotics}
%	\begin{itemize}[<+->]
%		\item Three questions:
%		\begin{itemize}[<+->]
%			\item \textbf{Which} type of \textbf{semantic information} is \textbf{relevant} to the \textbf{task}?	
%			\item \textbf{How} to perform the \textbf{inference/estimation} of the semantic information?	
%			\item \textbf{How} to \textbf{use} the \textbf{semantic information} to improve the \textbf{robot's performance}?	
%		\end{itemize}
%		\item We \textbf{investigate} these questions in the context of a \textbf{high-level task}: \textbf{object search} (OS)		
%	\end{itemize}		
%\end{frame}